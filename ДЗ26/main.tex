\documentclass[a4paper]{article}
\usepackage{setspace}
\usepackage[utf8]{inputenc}
\usepackage[russian]{babel}
\usepackage[12pt]{extsizes}
\usepackage{mathtools}
\usepackage{graphicx}
\usepackage{fancyhdr}
\usepackage{amssymb}
\usepackage{amsmath, amsfonts, amssymb, amsthm, mathtools}
\usepackage{tikz}

\usetikzlibrary{positioning}
\setstretch{1.3}

\newcommand{\mat}[1]{\begin{pmatrix} #1 \end{pmatrix}}
\newcommand{\vmat}[1]{\begin{vmatrix} #1 \end{vmatrix}}
\renewcommand{\f}[2]{\frac{#1}{#2}} 
\newcommand{\dspace}{\space\space}
\newcommand{\s}[2]{\sum\limits_{#1}^{#2}}
\newcommand{\mul}[2]{\prod_{#1}^{#2}}
\newcommand{\sq}[1]{\left[ {#1} \right]}
\newcommand{\gath}[1]{\left[ \begin{array}{@{}l@{}} #1 \end{array} \right.}
\newcommand{\case}[1]{\begin{cases} #1 \end{cases}}
\newcommand{\ts}{\text{\space}}
\newcommand{\lm}[1]{\underset{#1}{\lim}}
\newcommand{\suplm}[1]{\underset{#1}{\overline{\lim}}}
\newcommand{\inflm}[1]{\underset{#1}{\underline{\lim}}} 
\newcommand{\Ker}[1]{\operatorname{Ker}}

\renewcommand{\phi}{\varphi}
\newcommand{\lr}{\Leftrightarrow}
\renewcommand{\l}{\left(}
\renewcommand{\r}{\right)}
\newcommand{\rr}{\rightarrow}
\renewcommand{\geq}{\geqslant}
\renewcommand{\leq}{\leqslant}
\newcommand{\RR}{\mathbb{R}}
\newcommand{\CC}{\mathbb{C}}
\newcommand{\QQ}{\mathbb{Q}}
\newcommand{\ZZ}{\mathbb{Z}}
\newcommand{\VV}{\mathbb{V}}
\newcommand{\NN}{\mathbb{N}}
\newcommand{\OO}{\underline{O}}
\newcommand{\oo}{\overline{o}}
\renewcommand{\Ker}{\operatorname{Ker}}
\renewcommand{\Im}{\operatorname{Im}}
\newcommand{\vol}{\text{vol}}
\newcommand{\Vol}{\text{Vol}}

\DeclarePairedDelimiter\abs{\lvert}{\rvert} %
\makeatletter                               % \abs{}
\let\oldabs\abs                             %
\def\abs{\@ifstar{\oldabs}{\oldabs*}}       %

\begin{document}

\section*{Домашнее задание на 23.05 (Математический анализ)}
{\large Емельянов Владимир, ПМИ гр №247}\\\\
\begin{enumerate}
  \item[\textbf{№1}]\begin{enumerate}
    \item[(a)]Вычислим:
    \[\displaystyle\int_{0}^{+\infty} e^{-ax} \sin(bx) \, dx, \quad a > 0\]
    Применим дважды формулу интегрирования по частям (Формула 5). Пусть:
    \[
    u = \sin(bx), \quad dv = e^{-ax} dx \implies du = b \cos(bx) dx, 
    \quad v = -\frac{1}{a} e^{-ax}
    \]
    По формуле интегрирования по частям:
    \[
    \int_{0}^{\infty} e^{-ax} \sin(bx) dx = 
    \left[ -\frac{1}{a} e^{-ax} \sin(bx) \right]_0^{\infty}
     + \frac{b}{a} \int_{0}^{\infty} e^{-ax} \cos(bx) dx
    \]
    Пределы при \(x \to \infty\) стремятся к нулю, при \(x=0\) первый член равен нулю. Остаётся:
    \[
    \frac{b}{a} \int_{0}^{\infty} e^{-ax} \cos(bx) dx
    \]
    Повторим интегрирование по частям для нового интеграла. Пусть:
    \[
    u = \cos(bx), \quad dv = e^{-ax} dx \implies du = -b 
    \sin(bx) dx, \quad v = -\frac{1}{a} e^{-ax}
    \]
    Тогда:
    \[
    \frac{b}{a} \left( \left[ -\frac{1}{a} e^{-ax} \cos(bx) 
    \right]_0^{\infty} + \frac{b}{a} \int_{0}^{\infty} e^{-ax} \sin(bx) dx \right)
    \]
    Пределы обрабатываются аналогично, получаем:
    \[
    \frac{b}{a} \left( \frac{1}{a} + \frac{b}{a} I \right),
    \]
    где \(I\) — исходный интеграл. Решая уравнение:
    \[
    I = \frac{b}{a^2} + \frac{b^2}{a^2} I \implies I = \frac{b}{a^2 + b^2}
    \]
    \textbf{Ответ:} \(\dfrac{b}{a^2 + b^2}\)\\

    \item[(b)]Вычислим:
    \[
    \int_{0}^{1} \frac{x^3}{\sqrt[5]{1-x}} \, dx
    \] 
    Выберем:  
    \[
    f'(x) = \frac{1}{\sqrt[5]{1-x}} \implies f(x) = -\frac{5}{4}(1-x)^{4/5},
     \quad g(x) = x^3 \implies g'(x) = 3x^2
    \]  
    Применяем формулу интегрирования по частям:  
    \[
    \int_{0}^{1} \frac{x^3}{\sqrt[5]{1-x}} \, dx =
     \lim_{x \to 1-} \left[ -\frac{5}{4}x^3(1-x)^{4/5} \right] - 0 -
      \int_{0}^{1} \left( -\frac{5}{4}(1-x)^{4/5} \cdot 3x^2 \right) dx
    \]  
    Пределы обращаются в ноль, остается:  
    \[
    \frac{15}{4} \int_{0}^{1} x^2(1-x)^{4/5} \, dx
    \]
    Выберем:  
    \[
    f'(x) = (1-x)^{4/5} \implies f(x) = -\frac{5}{9}(1-x)^{9/5}, 
    \quad g(x) = x^2 \implies g'(x) = 2x
    \]  
    После аналогичных вычислений:  
    \[
    \frac{15}{4} \cdot \frac{10}{9} \int_{0}^{1} x(1-x)^{9/5} \,
     dx = \frac{25}{6} \int_{0}^{1} x(1-x)^{9/5} \, dx
    \]
    Выберем:  
    \[
    f'(x) = (1-x)^{9/5} \implies f(x) = -\frac{5}{14}(1-x)^{14/5}, \quad g(x) 
    = x \implies g'(x) = 1
    \]  
    Получаем:  
    \[
    \frac{25}{6} \cdot \frac{5}{14} \int_{0}^{1} (1-x)^{14/5} \, dx
     = \frac{125}{84} \int_{0}^{1} (1-x)^{14/5} \, dx
    \]
    \[
    \int_{0}^{1} (1-x)^{14/5} \, dx = \left[ -\frac{5}{19}(1-x)^{19/5} 
    \right]_{0}^{1} = \frac{5}{19}
    \]
    \[
    \frac{125}{84} \cdot \frac{5}{19} = \frac{625}{1596}
    \]
    \textbf{Ответ: } $\frac{625}{1596}$\\

    \item[(c)]Вычислим:
    \[
    \int_{2}^{+\infty} \frac{e^{\frac{1}{x}}}{x^3} \, dx
    \] 
    Положим \( t = \dfrac{1}{x} \). Тогда:  
    \[
    dt = -\dfrac{1}{x^2} dx \implies dx = -\dfrac{1}{t^2} dt, \quad x = \dfrac{1}{t}
    \]  
    Пределы интегрирования:  

    При \( x = 2 \): \[ t = \dfrac{1}{2} \]

    При \( x \to +\infty \): \[ t \to 0+ \]

    Интеграл преобразуется:  
    \[
    \int_{2}^{+\infty} \frac{e^{\frac{1}{x}}}{x^3} \, dx =
     \int_{1/2}^{0} \frac{e^{t}}{\left(\dfrac{1}{t}\right)^3} \cdot
      \left(-\dfrac{1}{t^2}\right) dt = \int_{0}^{1/2} e^{t} \cdot t \, dt
    \]

    Пусть:  
    \[
    u = t \implies du = dt, \quad dv = e^{t} dt \implies v = e^{t}
    \]  
    Применяем формулу интегрирования по частям:  
    \[
    \int_{0}^{1/2} t e^{t} \, dt = \left[ t e^{t} \right]_{0}^{1/2} - 
    \int_{0}^{1/2} e^{t} \, dt = \left( \dfrac{1}{2} e^{1/2} - 0 \right) -
     \left[ e^{t} \right]_{0}^{1/2}
    \]  
    Вычисляем:  
    \[
    = \dfrac{1}{2} e^{1/2} - \left( e^{1/2} - 1 \right) = -\dfrac{1}{2} e^{1/2} + 1
    \]
    \textbf{Ответ: } $1 - \dfrac{\sqrt{e}}{2}$\\
    
    \item[(d)]Вычислим:
    \[
    \int_{-1}^1 \frac{|\arcsin x|}{\sqrt{1 - x^2}} \, dx
    \]  
    Функция \(|\arcsin x|\) чётная, так как \(|\arcsin(-x)| = 
    |-\arcsin x| = |\arcsin x|\). Поэтому интеграл можно преобразовать:  
    \[
    \int_{-1}^1 \frac{|\arcsin x|}{\sqrt{1 - x^2}} \,
     dx = 2 \int_{0}^1 \frac{\arcsin x}{\sqrt{1 - x^2}} \, dx
    \]
    Положим \(t = \arcsin x\). Тогда:  
    \[
    x = \sin t, \quad dx = \cos t \, dt, \quad \sqrt{1 - x^2} = \cos t. 
    \]  
    Пределы интегрирования:  
    
    При \(x = 0\): \[t = 0\]  
    При \(x = 1\): \[t = \dfrac{\pi}{2}\]  

    Интеграл преобразуется:  
    \[
    2 \int_{0}^{\pi/2} \frac{t}{\cos t} \cdot \cos t \, dt = 2 \int_{0}^{\pi/2} t \, dt
    \]
    Найдём
    \[
    2 \int_{0}^{\pi/2} t \, dt = 2 \left[ \frac{t^2}{2} \right]_{0}^{\pi/2} =
     2 \cdot \frac{(\pi/2)^2}{2} = \frac{\pi^2}{4}
    \]
    \textbf{Ответ: } $\frac{\pi^2}{4}$\\

    \item[(e)]Вычислим:
    \[
    \int_{0}^{1} \frac{1}{(4 - 3x)\sqrt{x - x^2}} \, dx
    \]  
    замена: \(x = \dfrac{1}{t}\)

    \(dx = -\dfrac{1}{t^2} dt\)

    Пределы: \(x = 0 \to t \to +\infty\), \(x = 1 \to t = 1\)

    Подстановка:  
    \[
    \int_{+\infty}^{1} \frac{1}{\left(4 - \dfrac{3}{t}\right)\sqrt{\dfrac{1}{t} 
    - \dfrac{1}{t^2}}} \cdot \left(-\dfrac{1}{t^2}\right) dt =
     \int_{1}^{+\infty} \frac{1}{(4t - 3)\sqrt{t - 1}} \, dt
    \]
    замена: \(t = y + 1\)
    
    \(dt = dy\).  
   
    Пределы: \(t = 1 \to y = 0\), \(t \to +\infty \to y \to +\infty\)
   
    Подстановка:  
    \[
    \int_{0}^{+\infty} \frac{1}{(4y + 1)\sqrt{y}} \, dy
    \]
    замена: \(y = z^2\)

    \(dy = 2z dz\).  
   
    Пределы: \(y = 0 \to z = 0\), \(y \to +\infty \to z \to +\infty\).
   
    Подстановка:  
     \[
     \int_{0}^{+\infty} \frac{2z}{(4z^2 + 1)z} \, dz = 2 \int_{0}^{+\infty} 
     \frac{1}{4z^2 + 1} \, dz
     \]
    Найдём
    \[
    2 \cdot \frac{1}{2} \arctg(2z) \Big|_{0}^{+\infty} = 
    \arctg(2z) \Big|_{0}^{+\infty} = \frac{\pi}{2} - 0 = \frac{\pi}{2}
    \]
    \textbf{Ответ: } $\frac{\pi}{2}$\\

    \item[(f)]Вычислим:
    \[
    \int_{0}^{\pi} \frac{1}{2 - \cos x} \, dx
    \]  
    Положим \(t = \tg \dfrac{x}{2}\). Тогда:  
    \[
    x = 2 \arctg t, \quad dx = \dfrac{2}{1 + t^2} \, dt,
     \quad \cos x = \dfrac{1 - t^2}{1 + t^2}
    \]  
    Пределы интегрирования:  
    
    При \(x = 0\): \(t = 0\) 
    
    При \(x \to \pi-\): \(t \to +\infty\)

    Подставляем замену в интеграл:  
    \[
    \int_{0}^{\pi} \frac{1}{2 - \cos x} \, dx = \int_{0}^{+\infty} \frac{1}{2 - \dfrac{1 - t^2}{1 + t^2}} \cdot \dfrac{2}{1 + t^2} \, dt.
    \]  
    Упрощаем знаменатель:  
    \[
    2 - \dfrac{1 - t^2}{1 + t^2} = \dfrac{2(1 + t^2) - (1 - t^2)}{1 + t^2} =
     \dfrac{1 + 3t^2}{1 + t^2}
    \]  
    Интеграл принимает вид:  
    \[
    \int_{0}^{+\infty} \dfrac{2}{1 + 3t^2} \, dt
    \]
    
    Используем стандартный результат:  
    \[
    \int_{0}^{+\infty} \dfrac{1}{a^2 + t^2} \, dt = \dfrac{\pi}{2a}
    \]  
    Для \(a = \dfrac{1}{\sqrt{3}}\):  
    \[
    \int_{0}^{+\infty} \dfrac{2}{1 + 3t^2} \, dt = 2 \cdot 
    \dfrac{\pi}{2 \cdot \dfrac{1}{\sqrt{3}}} = \dfrac{\pi}{\sqrt{3}}
    \]

    \item[(g)]Вычислим:
    \[
    \int_{0}^{1} \cos^2(\ln x) \, dx
    \] 
    Используем тождество \(\cos^2 \theta = \dfrac{1 + \cos(2\theta)}{2}\):  
    \[
    \int_{0}^{1} \cos^2(\ln x) \, dx = \dfrac{1}{2} \int_{0}^{1} 
    \left(1 + \cos(2\ln x)\right) \, dx
    \]
    \[
    = \dfrac{1}{2} \left( \int_{0}^{1} 1 \, dx + \int_{0}^{1} \cos(2\ln x) \, dx \right)
    \]
    Первый интеграл:  
    \[
    \int_{0}^{1} 1 \, dx = 1
    \]
    Положим \(t = \ln x\), тогда \(x = e^t\), \(dx = e^t dt\). Пределы:  
    
    \(x = 0 \to t \to -\infty\),  
  
    \(x = 1 \to t = 0\)
    
    Интеграл преобразуется:  
    \[
    \int_{0}^{1} \cos(2\ln x) \, dx = \int_{-\infty}^{0} \cos(2t) \cdot e^t \, dt
    \]
    Пусть \(u = \cos(2t)\), \(dv = e^t dt\). Тогда \(du = -2\sin(2t) dt\), \(v = e^t\):  
    \[
    \int \cos(2t) \cdot e^t \, dt = e^t \cos(2t) + 2 \int e^t \sin(2t) \, dt
    \]  
    Для \(\int e^t \sin(2t) \, dt\) снова интегрируем по частям:  
    Пусть \(u = \sin(2t)\), \(dv = e^t dt\). Тогда \(du = 2\cos(2t) dt\), \(v = e^t\):  
    \[
    \int e^t \sin(2t) \, dt = e^t \sin(2t) - 2 \int e^t \cos(2t) \, dt
    \]  
    Подставляем обратно:  
    \[
    \int \cos(2t) \cdot e^t \, dt = e^t \cos(2t) + 2e^t \sin(2t) - 
    4 \int e^t \cos(2t) \, dt
    \]  
    Переносим интеграл в левую часть:  
    \[
    5 \int e^t \cos(2t) \, dt = e^t \left(\cos(2t) + 2\sin(2t)\right)
    \]  
    Отсюда:  
    \[
    \int e^t \cos(2t) \, dt = \dfrac{e^t \left(\cos(2t) + 2\sin(2t)\right)}{5} + C
    \]
    
    Подставляем пределы \(-\infty\) и \(0\):  
    
    При \(t \to -\infty\): \(e^t \to 0\), поэтому слагаемое стремится к 0.  
    
    При \(t = 0\):  
    \[
    \dfrac{e^0 \left(\cos(0) + 2\sin(0)\right)}{5} = \dfrac{1 \cdot (1 + 0)}{5}
     = \dfrac{1}{5}
    \]  
    Таким образом:  
    \[
    \int_{-\infty}^{0} \cos(2t) \cdot e^t \, dt = \dfrac{1}{5}
    \]
    Собираем все части:  
    \[
    \int_{0}^{1} \cos^2(\ln x) \, dx = \dfrac{1}{2} \left(1 + \dfrac{1}{5}\right)
     = \dfrac{1}{2} \cdot \dfrac{6}{5} = \dfrac{3}{5}
    \]
    \textbf{Ответ: } $\f{3}{5}$\\
  \end{enumerate}

  \item[\textbf{№2}]\begin{enumerate}
    \item[(a)]Найдём коэффициенты \( \alpha \) и \( \beta \)
    \[
    \int_{0}^{\pi/2} \ln \sin x \, dx = \alpha \cdot \int_{0}^{\pi/2} \ln 
    \sin(2x) \, dx + \beta
    \]
    Используем тождество \(\sin(2x) = 2 \sin x \cos x\):
    \[
    \ln \sin(2x) = \ln 2 + \ln \sin x + \ln \cos x.
    \]
    Подставляем
    \[
    \int_{0}^{\pi/2} \ln \sin x \, dx = \alpha \cdot \int_{0}^{\pi/2}
     \left(\ln 2 + \ln \sin x + \ln \cos x\right) dx + \beta
    \]
    Обозначим \( I = \int_{0}^{\pi/2} \ln \sin x \, dx \). Тогда:  
    \[
    I = \alpha \left(\ln 2 \cdot \dfrac{\pi}{2} + I + I\right) + \beta.
    \]
    \[
    I = \alpha \left(\dfrac{\pi \ln 2}{2} + 2I\right) + \beta \implies 
    I = \dfrac{\alpha \pi \ln 2}{2} + 2\alpha I + \beta
    \]  
    Переносим члены с \(I\):  
    \[
    I - 2\alpha I = \dfrac{\alpha \pi \ln 2}{2} + \beta 
    \implies I(1 - 2\alpha) = \dfrac{\alpha \pi \ln 2}{2} + \beta
    \]  
    Чтобы равенство выполнялось для всех \(I\), коэффициенты при \(I\) и 
    свободные члены должны совпадать:  
    \[
    1 - 2\alpha = 0 \implies \alpha = \dfrac{1}{2}, \quad 
    \dfrac{\alpha \pi \ln 2}{2} + \beta = 0 \implies \beta = -\dfrac{\pi \ln 2}{4}
    \]
    Получаем
    \[
    \alpha = \dfrac{1}{2}, \quad \beta = -\dfrac{\pi \ln 2}{4}
    \]

    \item[(b)]Замена переменной \(2x = y\):
    \[
    x = \dfrac{y}{2}, \quad dx = \dfrac{dy}{2}, \quad \text{пределы: }
     x=0 \to y=0, \quad x=\pi/2 \to y=\pi
    \]
    Преобразуем
    \[
    \int_{0}^{\pi/2} \ln \sin(2x) \, dx = \dfrac{1}{2} \int_{0}^{\pi} \ln \sin y \, dy
    \]
    В силу симметрии
    \[
    \int_{0}^{\pi} \ln \sin y \, dy = 2 \int_{0}^{\pi/2} \ln \sin y \, dy = 2I
    \]
    подставим
    \[
    \int_{0}^{\pi/2} \ln \sin(2x) \, dx = \dfrac{1}{2} \cdot 2I = I \implies \gamma = 1
    \]
    \textbf{Ответ: }$\gamma = 1$

    \item[(c)]Подставляем \(\alpha = \dfrac{1}{2}\), \(\beta = 
    -\dfrac{\pi \ln 2}{4}\), \(\gamma = 1\) в уравнение из предыдущих пунктов:
    \[
    I = \dfrac{1}{2} \cdot I - \dfrac{\pi \ln 2}{4}
    \]
    Решаем уравнение относительно \(I\):
    \[
    I - \dfrac{1}{2}I = -\dfrac{\pi \ln 2}{4} \implies \dfrac{1}{2}I =
     -\dfrac{\pi \ln 2}{4} \implies I = -\dfrac{\pi \ln 2}{2}
    \]
    \textbf{Ответ: } $-\dfrac{\pi \ln 2}{2}$\\
  \end{enumerate}

  \item[\textbf{№3}]Рассмотрим тело, 
  образованное вращением кривой \( y = \dfrac{1}{x} \) 
  вокруг оси \( x \) на интервале \( [1, +\infty) \). 
  
  Объем тела вращения вычисляется по формуле:  
  \[
  V = \pi \int_{1}^{\infty} \left( \dfrac{1}{x} \right)^2 dx = \pi \int_{1}^{\infty} \dfrac{1}{x^2} dx.
  \]  
  Интеграл сходится:  
  \[
  V = \pi \left[ -\dfrac{1}{x} \right]_{1}^{\infty} = \pi \left( 0 - (-1) \right) = \pi.
  \]
  Площадь поверхности тела вращения вычисляется по формуле:  
  \[
  S = 2\pi \int_{1}^{\infty} \dfrac{1}{x} \sqrt{1 + \left( -\dfrac{1}{x^2} \right)^2 } dx.
  \]  
  Упрощаем подынтегральное выражение:  
  \[
  \sqrt{1 + \dfrac{1}{x^4}} \approx 1 \quad \text{при } x \to \infty.
  \]  
  Таким образом, интеграл ведет себя как:  
  \[
  S \approx 2\pi \int_{1}^{\infty} \dfrac{1}{x} dx,
  \]  
  который расходится. Следовательно, площадь поверхности бесконечна.

\end{enumerate}
\end{document}