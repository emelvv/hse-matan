\documentclass[a4paper]{article}
\usepackage{setspace}
\usepackage[utf8]{inputenc}
\usepackage[russian]{babel}
\usepackage[12pt]{extsizes}
\usepackage{mathtools}
\usepackage{graphicx}
\usepackage{fancyhdr}
\usepackage{amssymb}
\usepackage{amsmath, amsfonts, amssymb, amsthm, mathtools}
\usepackage{tikz}

\usetikzlibrary{positioning}
\setstretch{1.3}

\newcommand{\mat}[1]{\begin{pmatrix} #1 \end{pmatrix}}
\newcommand{\vmat}[1]{\begin{vmatrix} #1 \end{vmatrix}}
\renewcommand{\f}[2]{\frac{#1}{#2}} 
\newcommand{\dspace}{\space\space}
\newcommand{\s}[2]{\sum\limits_{#1}^{#2}}
\newcommand{\mul}[2]{\prod_{#1}^{#2}}
\newcommand{\sq}[1]{\left[ {#1} \right]}
\newcommand{\gath}[1]{\left[ \begin{array}{@{}l@{}} #1 \end{array} \right.}
\newcommand{\case}[1]{\begin{cases} #1 \end{cases}}
\newcommand{\ts}{\text{\space}}
\newcommand{\lm}[1]{\underset{#1}{\lim}}
\newcommand{\suplm}[1]{\underset{#1}{\overline{\lim}}}
\newcommand{\inflm}[1]{\underset{#1}{\underline{\lim}}} 
\newcommand{\Ker}[1]{\operatorname{Ker}}

\renewcommand{\phi}{\varphi}
\newcommand{\lr}{\Leftrightarrow}
\renewcommand{\l}{\left(}
\renewcommand{\r}{\right)}
\newcommand{\rr}{\rightarrow}
\renewcommand{\geq}{\geqslant}
\renewcommand{\leq}{\leqslant}
\newcommand{\RR}{\mathbb{R}}
\newcommand{\CC}{\mathbb{C}}
\newcommand{\QQ}{\mathbb{Q}}
\newcommand{\ZZ}{\mathbb{Z}}
\newcommand{\VV}{\mathbb{V}}
\newcommand{\NN}{\mathbb{N}}
\newcommand{\OO}{\underline{O}}
\newcommand{\oo}{\overline{o}}
\renewcommand{\Ker}{\operatorname{Ker}}
\renewcommand{\Im}{\operatorname{Im}}
\newcommand{\vol}{\text{vol}}
\newcommand{\Vol}{\text{Vol}}

\DeclarePairedDelimiter\abs{\lvert}{\rvert} %
\makeatletter                               % \abs{}
\let\oldabs\abs                             %
\def\abs{\@ifstar{\oldabs}{\oldabs*}}       %

\begin{document}

\section*{Домашнее задание на 23.05 (Математический анализ)}
{\large Емельянов Владимир, ПМИ гр №247}\\\\
\begin{enumerate}
  \item[\textbf{№1}]\begin{enumerate}
    \item[(a)]Вычислим:
    \[\displaystyle\int_{0}^{+\infty} e^{-ax} \sin(bx) \, dx, \quad a > 0\]
    Применим дважды формулу интегрирования по частям (Формула 5). Пусть:
    \[
    u = \sin(bx), \quad dv = e^{-ax} dx \implies du = b \cos(bx) dx, 
    \quad v = -\frac{1}{a} e^{-ax}
    \]
    По формуле интегрирования по частям:
    \[
    \int_{0}^{\infty} e^{-ax} \sin(bx) dx = 
    \left[ -\frac{1}{a} e^{-ax} \sin(bx) \right]_0^{\infty}
     + \frac{b}{a} \int_{0}^{\infty} e^{-ax} \cos(bx) dx
    \]
    Пределы при \(x \to \infty\) стремятся к нулю, при \(x=0\) первый член равен нулю. Остаётся:
    \[
    \frac{b}{a} \int_{0}^{\infty} e^{-ax} \cos(bx) dx
    \]
    Повторим интегрирование по частям для нового интеграла. Пусть:
    \[
    u = \cos(bx), \quad dv = e^{-ax} dx \implies du = -b 
    \sin(bx) dx, \quad v = -\frac{1}{a} e^{-ax}
    \]
    Тогда:
    \[
    \frac{b}{a} \left( \left[ -\frac{1}{a} e^{-ax} \cos(bx) 
    \right]_0^{\infty} + \frac{b}{a} \int_{0}^{\infty} e^{-ax} \sin(bx) dx \right)
    \]
    Пределы обрабатываются аналогично, получаем:
    \[
    \frac{b}{a} \left( \frac{1}{a} + \frac{b}{a} I \right),
    \]
    где \(I\) — исходный интеграл. Решая уравнение:
    \[
    I = \frac{b}{a^2} + \frac{b^2}{a^2} I \implies I = \frac{b}{a^2 + b^2}
    \]
    \textbf{Ответ:} \(\dfrac{b}{a^2 + b^2}\)

    \item[(b)]Вычислим:
    
  \end{enumerate}
\end{enumerate}
\end{document}