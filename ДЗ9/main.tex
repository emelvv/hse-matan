\documentclass[a4paper]{article}
\usepackage{setspace}
\usepackage[T2A]{fontenc} %
\usepackage[utf8]{inputenc} % подключение русского языка
\usepackage[russian]{babel} %
\usepackage[12pt]{extsizes}
\usepackage{mathtools}
\usepackage{graphicx}
\usepackage{fancyhdr}
\usepackage{amssymb}
\usepackage{amsmath, amsfonts, amssymb, amsthm, mathtools}
\usepackage{tikz}

\usetikzlibrary{positioning}
\setstretch{1.3}

\newcommand{\mat}[1]{\begin{pmatrix} #1 \end{pmatrix}}
\renewcommand{\det}[1]{\begin{vmatrix} #1 \end{vmatrix}}
\renewcommand{\f}[2]{\frac{#1}{#2}}
\newcommand{\dspace}{\space\space}
\newcommand{\s}[2]{\sum\limits_{#1}^{#2}}
\newcommand{\mul}[2]{\prod_{#1}^{#2}}
\newcommand{\sq}[1]{\left[ {#1} \right]}
\newcommand{\gath}[1]{\left[ \begin{array}{@{}l@{}} #1 \end{array} \right.}
\newcommand{\case}[1]{\begin{cases} #1 \end{cases}}
\newcommand{\ts}{\text{\space}}
\newcommand{\lm}[1]{\underset{#1}{\lim}}
\newcommand{\suplm}[1]{\underset{#1}{\overline{\lim}}}
\newcommand{\inflm}[1]{\underset{#1}{\underline{\lim}}}

\renewcommand{\phi}{\varphi}
\newcommand{\lr}{\Leftrightarrow}
\renewcommand{\r}{\Rightarrow}
\newcommand{\rr}{\rightarrow}
\renewcommand{\geq}{\geqslant}
\renewcommand{\leq}{\leqslant}
\newcommand{\RR}{\mathbb{R}}
\newcommand{\CC}{\mathbb{C}}
\newcommand{\QQ}{\mathbb{Q}}
\newcommand{\ZZ}{\mathbb{Z}}
\newcommand{\VV}{\mathbb{V}}
\newcommand{\NN}{\mathbb{N}}
\newcommand{\OO}{\underline{O}}
\newcommand{\oo}{\overline{o}}


\DeclarePairedDelimiter\abs{\lvert}{\rvert} %
\makeatletter                               % \abs{}
\let\oldabs\abs                             %
\def\abs{\@ifstar{\oldabs}{\oldabs*}}       %

\begin{document}

\section*{Домашнее задание на 20.11 (Линейная алгебра)}
 {\large Емельянов Владимир, ПМИ гр №247}\\\\
\begin{enumerate}
    \item[\textbf{1.}]
    \begin{enumerate}
        \item[(a)]$f(x) = \begin{cases}
            2^{x^{-1}}, & x \neq 0, \\
            -3,         & x = 0.
        \end{cases} = f(x) = \begin{cases}
            2^{\f{1}{x}}, & x \neq 0\\
            -3, & x = 0
        \end{cases}$\\\\
        Найдем предел функции при $ x \to 0 $:
        $$
        \lim_{x \to 0} f(x) = \lim_{x \to 0} 2^{x^{-1}}.
        $$
        Когда $ x \to 0^+ $, $ x^{-1} \to +\infty $, и $ 2^{x^{-1}} \to +\infty $. Когда $ x \to 0^- $, $ x^{-1} \to -\infty $, и $ 2^{x^{-1}} \to 0 $\\
        Предела не существует, так как:
        $$
        \lim_{x \to 0^+} f(x) = +\infty \quad \text{и} \quad \lim_{x \to 0^-} f(x) = 0.
        $$
        Поскольку $ \lim_{x \to 0^+} f(x) $ не существует, точка $ x = 0 $ является разрывом второго рода.\\
    
        \item[(b)]$f(x) = \begin{cases}
            \arctan \frac{1}{x}, & x \neq 0, \\
            0, & x = 0.
        \end{cases}$\\
        Найдем предел функции при $ x \to 0 $:
        $$
        \lim_{x \to 0} f(x) = \lim_{x \to 0} \arctan \frac{1}{x}.
        $$
        Когда $ x \to 0^+ $, $ \frac{1}{x} \to +\infty $ и $ \arctan \frac{1}{x} \to \frac{\pi}{2} $. Когда $ x \to 0^- $, $ \frac{1}{x} \to -\infty $ и $ \arctan \frac{1}{x} \to -\frac{\pi}{2} $.\\
        Предела не существует, так как:
        $$
        \lim_{x \to 0^+} f(x) = \frac{\pi}{2} \quad \text{и} \quad \lim_{x \to 0^-} f(x) = -\frac{\pi}{2}.
        $$
        Поскольку $ \lim_{x \to 0} f(x) $ не существует, но существуют два оносторонних предела, точка $ x = 0 $ является разрывом первого рода.\\

        \item[(c)]$f(x) = \begin{cases}
            x, & x \in \mathbb{Q}, \\
            0, & x \in \mathbb{R} \backslash \mathbb{Q}.
        \end{cases}$\\
        Найдем предел функции при $ x \to 0 $:
        $$
        \lim_{x \to 0} f(x).
        $$
        Для рациональных чисел $ x_1 \to 0 $ (где $ f(x_1) = x_1 $), мы имеем:
        $$
        \lim_{x_1 \to \infty} f(x_1) = 0.
        $$
        Для иррациональных чисел $ x_2 \to 0 $ (где $ f(x_2) = 0 $), мы имеем:
        $$
        \lim_{x_2 \to \infty} f(x_2) = 0.
        $$
        Значение функции в точке $ x = 0 $ равно 0, то есть:
        $$
        f(0) = 0.
        $$
        Таким образом, функция непрерывна в точке $ x = 0 $.\\
    \end{enumerate}
    \item[\textbf{2.}]$f(x) = \case{
        \frac{\cos(\sin(3x)) - 1}{x^2},x \neq 0, \\
        \lambda, x=0 
    }$\\\\
    Чтобы функция $ f(x) $ была непрерывной в точке $ x=0 $, необходимо, чтобы выполнялось следующее условие:
    $$
    \lim_{x \to 0} f(x) = f(0) = \lambda.
    $$
    Сначала найдем предел $ \lim_{x \to 0} f(x) $ для $ x \neq 0 $:
    $$
    f(x) = \frac{\cos(\sin(3x)) - 1}{x^2}.
    $$
    $$\lim_{x \to 0}\frac{\cos(\sin(3x)) - 1}{x^2} = \lim_{x \to 0}\f{1 - \f{1}{2}\sin^2(x) + \oo(sin^2(3x)) -1}{x^2} = $$
    $$=\lim_{x \to 0}\f{-\f{1}{2}\sin^2(3x) + \oo(sin^2(3x))}{x^2} = \lim_{x \to 0}\f{\sin^2(3x)(-\f{1}{2} + \oo(1))}{x^2} =$$ 
    $$=\lim_{x \to 0}\f{\sin^2(3x)(\oo(1)-\f{1}{2})}{x^2} =\lim_{x \to 0}\f{(3x+\oo(9x^2))^2(\oo(1)-\f{1}{2})}{x^2} = $$
    $$=\lim_{x \to 0}(3+\oo(9x))^2(\oo(1)-\f{1}{2})=\lim_{x \to 0}(9 + 6\oo(9x)+\oo(81x^2))(\oo(1)-\f{1}{2})=$$
    $$=\lim_{x \to 0}(9 + 6\oo(9x)+\oo(81x^2))(\oo(1)-\f{1}{2})=\lim_{x \to 0}(-\f{9}{2}+\oo(1)) = -\f{9}{2}$$
    Для непрерывности функции в точке $ x=0 $ необходимо, чтобы:
    $$
    \lambda = \lim_{x \to 0} f(x) = -\frac{9}{2}.
    $$
    \textbf{Ответ: }-$\f{9}{2}$

    \item[\textbf{3.}]\begin{enumerate}
        \item[(a)]$f(x) = \frac{2 + x^2}{\sqrt{1 + x^4}} $
        $$f'(x) = \frac{(2 + x^2)' \cdot \sqrt{1 + x^4} - (2 + x^2) \cdot (\sqrt{1 + x^4})'}{(1 + x^4)} = $$
        Находим производные:
        $$(2 + x^2)' = 2x,$$
        $$(\sqrt{1 + x^4})' = \frac{1}{2\sqrt{1 + x^4}} \cdot (4x^3) = \frac{2x^3}{\sqrt{1 + x^4}}.$$
        Теперь подставим в формулу:
        $$
        f'(x) = \frac{2x \cdot \sqrt{1 + x^4} - (2 + x^2) \cdot \frac{2x^3}{\sqrt{1 + x^4}}}{1 + x^4}= $$
        $$=\frac{2x(1 + x^4) - 2x^3(2 + x^2)}{(1 + x^4)\sqrt{1 + x^4}} =$$
        $$= \frac{2x + 2x^5 - 4x^3 - 2x^5}{(1 + x^4)\sqrt{1 + x^4}} = \frac{2x - 4x^3}{(1 + x^4)\sqrt{1 + x^4}}.
        $$\\

        \item[(b)]$f(x) = \arcsin(5^{x^2})$\\
        Область определения:
        $$-1 \leq 5^{x^2} \leq 1 \r x^2 <= 0 \r x = 0$$
        Значит функция определена только в точке 0. Производная в точке 0:
        $$\lm{x \to 0}\f{f(x)-f(0)}{x-0} = \lm{x \to 0}\f{\arcsin(5^{x^2})-\arcsin(1)}{x} =$$
        $$=\lm{x \to 0}\f{\arcsin(5^{x^2})-\f{\pi}{2}}{x} $$
        

        \item[(c)]$ f(x) = (2 + \cos(3x))^{\ln x}$
        $$f'(x) = \left( (2 + \cos(3x))^{\ln x} \right)' =$$
        $$={e}^{\ln\left(x\right)\,\ln\left(\cos\left(3\,x\right)+2\right)}\cdot \left(\ln\left(x\right)\,\ln\left(\cos\left(3\,x\right)+2\right)\right)' = $$
        $$= {e}^{\ln\left(x\right)\,\ln\left(\cos\left(3\,x\right)+2\right)}\cdot \left(\left(\ln\left(x\right)\right)'\cdot \ln\left(\cos\left(3\,x\right)+2\right)+\left(\ln\left(\cos\left(3\,x\right)+2\right)\right)'\cdot \ln\left(x\right)\right) = $$
        $$= {e}^{\ln\left(x\right)\,\ln\left(\cos\left(3\,x\right)+2\right)}\cdot \left(\dfrac{\ln\left(\cos\left(3\,x\right)+2\right)}{x}+\dfrac{\ln\left(x\right)}{\cos\left(3\,x\right)+2}\cdot \left(\cos\left(3\,x\right)+2\right)'\right) =$$
        $$= \left({\cos\left(3\,x\right)+2}\right)^{\ln\left(x\right)}\,\left(\dfrac{\ln\left(\cos\left(3\,x\right)+2\right)}{x}-\dfrac{3\,\ln\left(x\right)\,\sin\left(3\,x\right)}{\cos\left(3\,x\right)+2}\right)$$\\
        
        \item[(d)]$ f(x) = 2^{\arctan(\sqrt{1 + x^2})}$
        $$f'(x) = 2^{\arctan(\sqrt{1 + x^2})} \cdot \ln(2) \cdot (\arctan(\sqrt{1 + x^2}))'$$
        Находим производную $ \arctan(u) $:
        $$
        (\arctan(u))' = \frac{u'}{1 + u^2}.
        $$
        Где $ u = \sqrt{1 + x^2} $:
        $$
        u' = \frac{x}{\sqrt{1 + x^2}}.
        $$
        Теперь подставим:
        $$
        f'(x) = 2^{\arctan(\sqrt{1 + x^2})} \cdot \ln(2) \cdot \frac{\frac{x}{\sqrt{1 + x^2}}}{1 + (1 + x^2)} = 2^{\arctan(\sqrt{1 + x^2})} \cdot \ln(2) \cdot \frac{x}{\sqrt{1 + x^2}(2 + x^2)}=
        $$
        $$=\dfrac{\ln\left(2\right)\,x\,{2}^{\operatorname{arctan}\left(\sqrt{{x}^{2}+1}\right)}}{\sqrt{{x}^{2}+1}\,\left({x}^{2}+2\right)}$$\\

        \item[(e)] $ f(x) = x^{a^a} + a^{x^a} + a^{a^x} $
        
        Находим производные по отдельности:

        1. $ (x^{a^a})' = a^a x^{a^a - 1} $.\\
        2. $ (a^{x^a})' = a^{x^a} \cdot \ln(a) \cdot (x^a)' = a^{x^a} \cdot \ln(a) \cdot a x^{a - 1} $.\\
        3. $ (a^{a^x})' = a^{a^x} \cdot \ln(a) \cdot (a^x)' = a^{a^x} \cdot \ln(a) \cdot a^x \ln(a) $.\\

        Теперь объединяем:

        $$
        f'(x) = a^a x^{a^a - 1} + a^{x^a+1} \cdot \ln(a) \cdot x^{a - 1} + a^{a^x + x} \cdot \ln^2(a)
        $$\\
    \end{enumerate}

    \item[\textbf{4.}]$f(x) = \case{
        x^2\sin\f{1}{x}, \; x \neq 0 \\
        0, \; x = 0
    }$\\\\
    \begin{enumerate} 
        \item[(a)]
        Чтобы найти производную функции $ f(x) $ в точке $ x = 0 $, воспользуемся определением производной:
        $$
        f'(0) = \lim_{h \to 0} \frac{f(h) - f(0)}{h}
        $$
        
        Значение функции в нуле:
        $$
        f(0) = 0
        $$
        Теперь подставим $ f(h) $ для $ h \neq 0 $:
        $$
        f(h) = h^2 \sin \frac{1}{h}
        $$
        Тогда производная в нуле будет:
        $$
        f'(0) = \lim_{h \to 0} \frac{h^2 \sin \frac{1}{h}}{h} = \lim_{h \to 0} h \sin \frac{1}{h} = 0
        $$
        
        Теперь найдем производную $ f'(x) $ для $ x \neq 0 $ с использованием правила произведения:

        $$
        f'(x) = (x^2 \sin \frac{1}{x})' = 2x \sin \frac{1}{x} + x^2 \cos \frac{1}{x} \cdot \left(-\frac{1}{x^2}\right) = 
        $$
        $$=2x \sin \frac{1}{x} - \cos \frac{1}{x}$$
        
        Таким образом, производная функции $ f(x) $ будет:

        $$
        f'(x) = \left\{\begin{array}{ll}
        2x \sin \frac{1}{x} - \cos \frac{1}{x}, & x \neq 0 \\
        0, & x = 0
        \end{array}\right.
        $$

        \item[(b)]Теперь установим род каждой точки разрыва полученной производной $ f'(x) $.

        Для $ x \neq 0 $ функция $ f'(x) $ непрерывна, так как $ \sin \frac{1}{x} $ и $ \cos \frac{1}{x} $ являются непрерывными функциями. Однако, необходимо проверить непрерывность в точке $ x = 0 $:
        
        $$
        \lim_{x \to 0} f'(x) = \lim_{x \to 0} \left(2x \sin \frac{1}{x} - \cos \frac{1}{x}\right)
        $$
        
        Поскольку $ \lm{x \to 0} 2x \sin \frac{1}{x} = 0 $, а $ \cos \frac{1}{x} $ не имеет предела при $ x \to 0 $ (колеблется между -1 и 1), то:
        
        $$
        \lim_{x \to 0} f'(x) \text{ не существует}
        $$
        
        Таким образом, в точке $ x = 0 $ производная имеет разрыв второго рода.

    \end{enumerate}
    
    \item[\textbf{5.}]
    Для матрицы $ A(x) = \begin{pmatrix} a(x) & b(x) \\ c(x) & d(x) \end{pmatrix} $ мы сначала найдем определитель $ \operatorname{det} A(x) $:
    $$
    \operatorname{det} A(x) = a(x)d(x) - b(x)c(x)
    $$
    
    Теперь вычислим производную определителя по $ x $:

    $$
    \frac{d}{dx} \operatorname{det} A(x) = \frac{d}{dx} (a(x)d(x) - b(x)c(x)) =
    $$
    
    $$
    = \frac{d a(x)}{dx} d(x) + a(x) \frac{d d(x)}{dx} - \left( \frac{d b(x)}{dx} c(x) + b(x) \frac{d c(x)}{dx} \right) =
    $$
    
    $$
    = \frac{d a(x)}{dx} d(x) + a(x) \frac{d d(x)}{dx} - \frac{d b(x)}{dx} c(x) - b(x) \frac{d c(x)}{dx}
    $$

    Теперь найдем присоединённую матрицу $ \operatorname{adj}(A(x)) $:

    $$
    \operatorname{adj}(A(x)) = \begin{pmatrix} d(x) & -b(x) \\ -c(x) & a(x) \end{pmatrix}
    $$
    
    Теперь вычислим матрицу $ \frac{d A(x)}{d x} $:

    $$
    \frac{d A(x)}{d x} = \begin{pmatrix} \frac{d a(x)}{dx} & \frac{d b(x)}{dx} \\ \frac{d c(x)}{dx} & \frac{d d(x)}{dx} \end{pmatrix}
    $$
    Справа в следе получаем:
    $$
    \operatorname{adj}(A(x)) \cdot \frac{d A(x)}{d x} = \begin{pmatrix} d(x) & -b(x) \\ -c(x) & a(x) \end{pmatrix} \begin{pmatrix} \frac{d a(x)}{dx} & \frac{d b(x)}{dx} \\ \frac{d c(x)}{dx} & \frac{d d(x)}{dx} \end{pmatrix}.
    $$
    
    Вычислим элементы этого произведения:

    1. Первый элемент:
    $$
    d(x) \frac{d a(x)}{dx} - b(x) \frac{d c(x)}{dx}.
    $$

    2. Второй элемент:
    $$
    d(x) \frac{d b(x)}{dx} - b(x) \frac{d d(x)}{dx}.
    $$

    3. Третий элемент:
    $$
    -c(x) \frac{d a(x)}{dx} + a(x) \frac{d c(x)}{dx}.
    $$

    4. Четвертый элемент:
    $$
    -c(x) \frac{d b(x)}{dx} + a(x) \frac{d d(x)}{dx}.
    $$
        
    Теперь найдем след:

    $$
    \operatorname{tr}\left(\operatorname{adj}(A(x)) \cdot \frac{d A(x)}{d x}\right) = \left(d(x) \frac{d a(x)}{dx} - b(x) \frac{d c(x)}{dx}\right) + \left(-c(x) \frac{d b(x)}{dx} + a(x) \frac{d d(x)}{dx}\right).
    $$
    
    Упростим:

    $$
    \operatorname{tr}\left(\operatorname{adj}(A(x)) \cdot \frac{d A(x)}{d x}\right) = d(x) \frac{d a(x)}{dx} + a(x) \frac{d d(x)}{dx} - b(x) \frac{d c(x)}{dx} - c(x) \frac{d b(x)}{dx}
    $$

    Получаем:
    $$
    \case{
        \operatorname{tr}\left(\operatorname{adj}(A(x)) \cdot \frac{d A(x)}{d x}\right) = d(x) \frac{d a(x)}{dx} + a(x) \frac{d d(x)}{dx} - b(x) \frac{d c(x)}{dx} - c(x) \frac{d b(x)}{dx} \\
        \frac{d}{dx} \operatorname{det} A(x) = \frac{d a(x)}{dx} d(x) + a(x) \frac{d d(x)}{dx} - \frac{d b(x)}{dx} c(x) - b(x) \frac{d c(x)}{dx}
    }
    $$
    \textbf{Следовательно, $\operatorname{tr}\left(\operatorname{adj}(A(x)) \cdot \frac{d A(x)}{d x}\right) = \frac{d}{dx} \operatorname{det} A(x)$}



\end{enumerate}
\end{document}