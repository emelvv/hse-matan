\documentclass[a4paper]{article}
\usepackage{setspace}
\usepackage[T2A]{fontenc} %
\usepackage[utf8]{inputenc} % подключение русского языка
\usepackage[russian]{babel} %
\usepackage[12pt]{extsizes}
\usepackage{mathtools}
\usepackage{graphicx}
\usepackage{fancyhdr}
\usepackage{amssymb}
\usepackage{amsmath, amsfonts, amssymb, amsthm, mathtools}
\usepackage{tikz}

\usetikzlibrary{positioning}
\setstretch{1.3}

\newcommand{\mat}[1]{\begin{pmatrix} #1 \end{pmatrix}}
\newcommand{\matsq}[1]{\begin{bmatrix} #1 \end{bmatrix}}
\newcommand{\vmat}[1]{\begin{vmatrix} #1 \end{vmatrix}}
\renewcommand{\f}[2]{\frac{#1}{#2}}
\newcommand{\dspace}{\space\space}
\newcommand{\s}[2]{\sum\limits_{#1}^{#2}}
\newcommand{\mul}[2]{\prod_{#1}^{#2}}
\newcommand{\sq}[1]{\left[ {#1} \right]}
\newcommand{\gath}[1]{\left[ \begin{array}{@{}l@{}} #1 \end{array} \right.}
\newcommand{\case}[1]{\begin{cases} #1 \end{cases}}
\newcommand{\ts}{\text{\space}}
\newcommand{\lm}[1]{\underset{#1}{\lim}}
\newcommand{\suplm}[1]{\underset{#1}{\overline{\lim}}}
\newcommand{\inflm}[1]{\underset{#1}{\underline{\lim}}}
\newcommand{\Ker}[1]{\operatorname{Ker}}

\renewcommand{\phi}{\varphi}
\newcommand{\lr}{\Leftrightarrow}
\renewcommand{\l}{\left(}
\renewcommand{\r}{\right)}
\newcommand{\rr}{\rightarrow}
\renewcommand{\geq}{\geqslant}
\renewcommand{\leq}{\leqslant}
\newcommand{\RR}{\mathbb{R}}
\newcommand{\CC}{\mathbb{C}}
\newcommand{\QQ}{\mathbb{Q}}
\newcommand{\ZZ}{\mathbb{Z}}
\newcommand{\VV}{\mathbb{V}}
\newcommand{\NN}{\mathbb{N}}
\newcommand{\OO}{\underline{O}}
\newcommand{\oo}{\overline{o}}
\renewcommand{\Ker}{\operatorname{Ker}}
\renewcommand{\Im}{\operatorname{Im}}
\newcommand{\vol}{\text{vol}}
\newcommand{\Vol}{\text{Vol}}
\renewcommand{\d}{\;\text{d}}
\newcommand{\dx}{\;\text{dx}}

\DeclarePairedDelimiter\abs{\lvert}{\rvert} %
\makeatletter                               % \abs{}
\let\oldabs\abs                             %
\def\abs{\@ifstar{\oldabs}{\oldabs*}}       %

\begin{document}

\section*{Домашнее задание на 27.04 (Математический анализ)}
 {\large Емельянов Владимир, ПМИ гр №247}\\\\
\begin{enumerate}
    \item[\textbf{№1}]
    \begin{enumerate}
        \item[(a)]
        Функция \( f(x) = \ln x \) при \( x \in (0, 1] \) не ограничена в окрестности точки \( x = 0 \), так как \( \ln x \to -\infty \) при \( x \to 0^+ \). 
        Следовательно, по факту 1, она не принадлежит \( R([0, 1]) \).\\
        \textbf{Ответ: } не принадлежит

        \item[(b)]Функция \( g(x) \) является ступенчатой и монотонно убывающей на \([0, 1]\).
        Монотонные функции на отрезке ограничены и имеют не более чем счётное число точек разрыва (в данном случае — в точках \( x = \frac{1}{2n-1} \), где \( n \in \mathbb{N} \)).
        По Факту 2, такие функции интегрируемы по Риману.\\
        \textbf{Ответ: } принадлежит

        \item[(c)]Функция \( h(x) \) разрывна во всех точках отрезка \([0, 1]\), кроме \( x = 0 \).
        Множество точек разрыва несчётно (так как \([0, 1] \setminus \{0\}\) несчётно). 
        По факту 2 функция с несчётным числом точек разрыва не интегрируема по Риману.\\
        \textbf{Ответ: } не принадлежит
    \end{enumerate}

    \item[\textbf{№2}]
    \begin{enumerate}
        \item[(a)] Каждое слагаемое можно записать в виде:  
        \[
        \frac{1}{\sqrt{4n^2 + k^2}} = \frac{1}{n \sqrt{4 + \left(\frac{k}{n}\right)^2}}
        \]  
        То есть:
        \[
        \sum_{k=1}^{n} \frac{1}{n} \cdot \frac{1}{\sqrt{4 + \left(\frac{k}{n}\right)^2}}
        \]
        Это интегральная сумма Римана для функции \( f(x) = \frac{1}{\sqrt{4 + x^2}} \) на интервале \([0, 1]\) с разбиением \( x_k = \frac{k}{n} \) и шагом \( \Delta x = \frac{1}{n} \). 
    
        При \( n \to \infty \), сумма стремится к интегралу:  
        \[
        \int_{0}^{1} \frac{1}{\sqrt{4 + x^2}} \, dx
        \]  
        Используя формулу для интеграла \( \int \frac{dx}{\sqrt{a^2 + x^2}} = \ln(x + \sqrt{x^2 + a^2}) \):  
        \[
        \int_{0}^{1} \frac{1}{\sqrt{4 + x^2}} \, dx = \ln\left(1 + \sqrt{1 + 4}\right) - \ln(2) = \ln(1 + \sqrt{5}) - \ln(2) = \ln\left(\frac{1 + \sqrt{5}}{2}\right)
        \]
        
        \textbf{Ответ: } $\ln\left(\frac{1 + \sqrt{5}}{2}\right)$\\

        \item[(b)]Рассмотрим логарифм предела:  
        \[
        \ln L = \lim_{n \to \infty} \frac{1}{n} \sum_{k=1}^{n} \ln(n + k)
        \]  
        Выразим \( \ln(n + k) \) как:  
        \[
        \ln(n + k) = \ln\left(n \left(1 + \frac{k}{n}\right)\right) = \ln n + \ln\left(1 + \frac{k}{n}\right)
        \]  
        Тогда сумма примет вид:  
        \[
        \frac{1}{n} \sum_{k=1}^{n} \ln n + \frac{1}{n} \sum_{k=1}^{n} \ln\left(1 + \frac{k}{n}\right)
        \]  
        Первая сумма равна \( \ln n \), а вторая стремится к интегралу:  
        \[
        \int_{0}^{1} \ln(1 + x) \, dx
        \]  
        Интегрируем по частям:  
        \[
        \int_{0}^{1} \ln(1 + x) \, dx = (x \ln(1 + x))\bigg|_{0}^{1} - \int_{0}^{1} \frac{x}{1 + x} \, dx = \ln 2 - \left(1 - \ln 2\right) = 2 \ln 2 - 1
        \]  
        Однако исходное выражение требует учета множителя \( n \) в произведении:  
        \[
        \prod_{k=1}^{n} (n + k) = n^n \prod_{k=1}^{n} \left(1 + \frac{k}{n}\right)
        \]  
        Тогда:  
        \[
        \sqrt[n]{\prod_{k=1}^{n} (n + k)} = n \cdot \sqrt[n]{\prod_{k=1}^{n} \left(1 + \frac{k}{n}\right)}
        \]  
        Логарифм этого выражения:  
        \[
        \ln L = \ln n + \frac{1}{n} \sum_{k=1}^{n} \ln\left(1 + \frac{k}{n}\right)
        \]  
        При \( n \to \infty \), сумма \( \frac{1}{n} \sum_{k=1}^{n} \ln\left(1 + \frac{k}{n}\right) \) стремится к \( 2 \ln 2 - 1 \), а \( \ln n \) компенсируется множителем \( n \) в исходном выражении.  

        \textbf{Ответ: } $ L = e^{2 \ln 2 - 1} = \frac{4}{e} $\\
    \end{enumerate}


\end{enumerate}
\end{document}