\documentclass[a4paper]{article}
\usepackage{setspace}
\usepackage[T2A]{fontenc} %
\usepackage[utf8]{inputenc} % подключение русского языка
\usepackage[russian]{babel} %
\usepackage[12pt]{extsizes}
\usepackage{mathtools}
\usepackage{graphicx}
\usepackage{fancyhdr}
\usepackage{amssymb}
\usepackage{amsmath, amsfonts, amssymb, amsthm, mathtools}
\usepackage{tikz}

\usetikzlibrary{positioning}
\setstretch{1.3}

\newcommand{\mat}[1]{\begin{pmatrix} #1 \end{pmatrix}}
\newcommand{\matsq}[1]{\begin{bmatrix} #1 \end{bmatrix}}
\renewcommand{\det}[1]{\begin{vmatrix} #1 \end{vmatrix}}
\renewcommand{\f}[2]{\frac{#1}{#2}}
\newcommand{\dspace}{\space\space}
\newcommand{\s}[2]{\sum\limits_{#1}^{#2}}
\newcommand{\mul}[2]{\prod_{#1}^{#2}}
\newcommand{\sq}[1]{\left[ {#1} \right]}
\newcommand{\gath}[1]{\left[ \begin{array}{@{}l@{}} #1 \end{array} \right.}
\newcommand{\case}[1]{\begin{cases} #1 \end{cases}}
\newcommand{\ts}{\text{\space}}
\newcommand{\lm}[1]{\underset{#1}{\lim}}
\newcommand{\suplm}[1]{\underset{#1}{\overline{\lim}}}
\newcommand{\inflm}[1]{\underset{#1}{\underline{\lim}}}

\renewcommand{\phi}{\varphi}
\newcommand{\lr}{\Leftrightarrow}
\newcommand{\rr}{\rightarrow}
\renewcommand{\geq}{\geqslant}
\renewcommand{\leq}{\leqslant}
\newcommand{\RR}{\mathbb{R}}
\newcommand{\CC}{\mathbb{C}}
\newcommand{\QQ}{\mathbb{Q}}
\newcommand{\ZZ}{\mathbb{Z}}
\newcommand{\VV}{\mathbb{V}}
\newcommand{\NN}{\mathbb{N}}
\newcommand{\OO}{\underline{O}}
\newcommand{\oo}{\overline{o}}
\newcommand{\p}{\partial}
\renewcommand{\l}{\left(}
\renewcommand{\r}{\right)}


\DeclarePairedDelimiter\abs{\lvert}{\rvert} %
\makeatletter                               % \abs{}
\let\oldabs\abs                             %
\def\abs{\@ifstar{\oldabs}{\oldabs*}}       %

\begin{document}

\section*{Домашнее задание на 06.03 (Математический анализ)}
 {\large Емельянов Владимир, ПМИ гр №247}\\\\
\begin{enumerate}
    \item[\textbf{№1}]
    \begin{enumerate}
        \item[(a)]
        $f(x, y) = x^3 + 3xy^2 - 39x - 36y + 26$
        
        Найдём стационарные точки:
        $$\case{
            f_x'(x, y) =3\,\left({x}^{2}+{y}^{2}-13\right) =0\\
            f_y'(x, y) = 6\,\left(x\,y-6\right)=0\\
        }\implies \begin{aligned}
            A = (-3, -2), &\quad B= (-2, -3),\\
            C = (2, 3), & \quad D = (3, 2) 
        \end{aligned}$$
        Найдём матрицу Гессе:
        $$H_f = \mat{
            f_{x^2}'' & f_{xy}''\\
            f_{yx}'' & f_{y^2}''
        } = \mat{
            6x & 6y\\
            6y& 6x
        } \implies d^2f_{(x, y)}\left(\matsq{h_1\\h_2}\right) = 12xh_1^2+12yh_1h_2$$
    
        Найдём нормальный вид матрицы второго дифференциала для точек $A, B, C$ и $D$
        \begin{enumerate}
            \item[1)] $A = (-3, -2)$
            $$H_f = \mat{
            -18 & -12\\
            -12 & -18}$$
            $$\delta_1 = -18 < 0, \quad \delta_2  = 18^2-12^2 > 0$$
            Следовательно, 
            $$d^2f_{A}\left(\matsq{h_1\\h_2}\right) < 0$$
            Значит $A$ - точка локального максимума
    
            \item[2)] $B = (-2, -3)$
            $$H_f = \mat{
            -12 & -18\\
            -18 & -12}$$
            $$\delta_1 = -12 < 0, \quad \delta_2  = 12^2-18^2=-180 < 0$$
            Следовательно, 
            $$d^2f_{B}\left(\matsq{h_1\\h_2}\right) = -12h_1^2+15h_2^2$$
            $$\case{
                d^2f_{B}\left(\matsq{1\\0}\right) = -12 < 0\\
                d^2f_{B}\left(\matsq{0\\1}\right) = 15 > 0
            } \implies B \text{ - не точка локального экстремума}$$
    
            \item[3)] $C = (2, 3)$
            $$H_f = \mat{
                12 & 18\\
                18 & 12
            }$$
            $$\delta_1 = 12 > 0, \quad \delta_2  = 12^2-18^2 = -180< 0$$
            Следовательно, 
            $$d^2f_{B}\left(\matsq{h_1\\h_2}\right) = 12h_1^2-15h_2^2$$
            $$\case{
                d^2f_{B}\left(\matsq{1\\0}\right) = 12 > 0\\
                d^2f_{B}\left(\matsq{0\\1}\right) = -15 < 0
            } \implies C \text{ - не точка локального экстремума}$$
    
            \item[4)] $D = (3, 2)$
            $$H_f = \mat{
            18 & 12\\
            12 & 18}$$
            $$\delta_1 = 18 > 0, \quad \delta_2  = 18^2-12^2 > 0$$
            Следовательно, 
            $$d^2f_{D}\left(\matsq{h_1\\h_2}\right) > 0$$
            Значит $D$ - точка локального минимума       
            
        \end{enumerate}
        \textbf{Ответ: } $A$ - лок. мин., $D$ - лок. макс.\\

        \item[(b)] $f(x, y) = x^4 + y^4 - 2x^2$
        
        Найдём стационарные точки:
        $$\case{
            f_x'(x, y) =4\,\left({x}^{3}-x\right) =0\\
            f_y'(x, y) = 4\,{y}^{3}=0\\
        }\implies \begin{aligned}
            A = (0, 0), &\quad B= (-1, 0),\\
            C = (1, 0) 
        \end{aligned}$$
        Найдём матрицу Гессе:
        $$H_f = \mat{
            f_{x^2}'' & f_{xy}''\\
            f_{yx}'' & f_{y^2}''
        } = \mat{
            4\,\left(3\,{x}^{2}-1\right) & 0\\
            0 & 12\,{y}^{2}
        } $$
        $$\implies d^2f_{(x, y)}\left(\matsq{h_1\\h_2}\right) = 4\,\left(3\,{x}^{2}-1\right)h_1^2+12y^2h_2^2$$
        Подставим точки:
        \begin{enumerate}
            \item[1)]$A = (0, 0)$:
            $$\implies d^2f_{A}\left(\matsq{h_1\\h_2}\right)
             = -4h_1^2$$
             $$\implies d^2f_{A}\left(\matsq{h_1\\h_2}\right)<0$$
            Значит $A$ - точка локального максимума

            \item[2)]$B= (-1, 0)$:
            $$\implies d^2f_{B}\left(\matsq{h_1\\h_2}\right) =
             8h_1^2 > 0$$
             Значит $B$ - точка локального минимума

            \item[3)]$C = (1, 0) $:
            $$\implies d^2f_{С}\left(\matsq{h_1\\h_2}\right) =
            8h_1^2 > 0$$
            Значит $C$ - точка локального минимума
        \end{enumerate}
        \textbf{Ответ: } $B$ - лок. мин., $C$ - лок. мин.\\

        \item[(c)]$f(x, y, z) = 3x^3 + y^2 + z^2 + 6xy - 2z + 1$
        
        Найдём стационарные точки:
        $$\case{
            f_x'(x, y, z) =3\,\left(3\,{x}^{2}+2\,y\right) =0\\
            f_y'(x, y, z) = 2\,\left(y+3\,x\right)=0\\
            f_z'(x, y, z) =2\,\left(z-1\right) =0
        }\implies \begin{aligned}
            A = (0, 0, 1), &\quad B= (2, -6, 1)
        \end{aligned}$$
        Найдём матрицу Гессе:
        $$H_f = \mat{
            f_{x^2}'' & f_{xy}'' & f_{xz}''\\
            f_{yx}'' & f_{y^2}'' & f_{yz}''\\
            f_{zx}'' & f_{zy}'' & f_{z^2}''
        } = \mat{
            18x & 6 & 0\\
            6 & 2 & 0\\
            0 & 0 & 2
        } $$
        Приведём квадратичную форму к нормальному виду симметричным методом Гауса:
        $$\mat{
            18x & 6 & 0\\
            6 & 2 & 0\\
            0 & 0 & 2
        } \implies \mat{
            2 & 0 & 0\\
            0 & 18x-18 & 0\\
            0 & 0 & 2
        }$$
        Следовательно:
        $$d^2f_{(x, y, z)}\left(\matsq{h_1\\h_2\\h_3}\right) = 2h_1^2+(18x-18)h_2^2+2h_3^2$$
        Подставим точки:
        \begin{enumerate}
            \item[1)]$A = (0, 0, 1)$:
            $$d^2f_{A}\left(\matsq{h_1\\h_2\\h_3}\right) 
            = 2h_1^2-18h_2^2+2h_3^2$$
            $$\case{
                d^2f_{A}\left(\matsq{1\\0\\0}\right) >0\\
                d^2f_{A}\left(\matsq{0\\1\\0}\right) < 0 
            } \implies A \text{ - не точка локального экстремума}$$

            \item[2)]$B= (2, -6, 1)$:
            $$d^2f_{B}\left(\matsq{h_1\\h_2\\h_3}\right) 
            = 2h_1^2+18h_2^2+2h_3^2 >0 $$
            Следовательно, $B$ - точка локального минимума
            
        \end{enumerate}
        \textbf{Ответ: } $B$ - лок. мин.\\
    \end{enumerate}

    \item[\textbf{№2}]Если $(a, b)$ лежит на $x-y^2 = 6$, это значит, что:
    $$a = 6+b^2$$
    Если $(c, d)$ лежит на $-2x+y=0$ это значит, что:
    $$d = 2c$$
    Найдём расстояние между точками
    $$(6+b^2, b) \text{ и } (c, 2c)$$
    Подставим в формулу для нахождени расстояния в Евклидовом пространстве:
    $$\sqrt{(6+b^2-c)^2 + (b-2c)^2}$$
    Найдём точки минимума у функции:
    $$f(b, c) = (6+b^2-c)^2 + (b-2c)^2$$
    Найдём стационарные точки:
    $$\case{
        f_b' = 2\,\left(2\,{b}^{3}+\left(13-2\,c\right)\,b-2\,c\right)=0\\
        f_c' = 2\,\left(5\,c-{b}^{2}-2\,b-6\right) = 0
    } \implies (b, c) = \left(\f{1}{4}, \f{21}{16}\right)$$
    Найдём матрицу Гессе:
    $$H_f = \mat{
        f_{b^2}'' & f_{bc}''\\
        f_{cb}'' & f_{c^2}''
    } = \mat{
        2\,\left(6\,{b}^{2}-2\,c+13\right) & -4\,b-4\\
        -4\,b-4 & 10
    }$$
    Подставим $(b, c) = \left(\f{1}{4}, \f{21}{16}\right)$:
    $$\begin{bmatrix}
        \frac{43}{2} & -5 \\
        -5 & 10
    \end{bmatrix}$$
    Найдём угловые миноры:
    $$\delta_1 = \f{43}{2}, \quad \delta_2 = 190$$
    Следовательно, $(b, c) = \left(\f{1}{4}, \f{21}{16}\right)$ точка минимума функции. 
    
    Теперь найдём искомое минимальное расстояние между кривыми, подставив эту точку:
    $$\sqrt{(6+b^2-c)^2 + (b-2c)^2} = \frac{19\sqrt{5}}{8}$$
    \textbf{Ответ: } $\frac{19\sqrt{5}}{8}$\\

    \item[\textbf{№3}]Псевдорешение и значение функции невязки для несовместной СЛУ:

    \begin{enumerate}
        \item[1)] Нормальное уравнение:  
           \[
           A^T A \theta = A^T \beta
           \]  
           Решение системы:  
           \[
           \theta = \begin{bmatrix}
           -\dfrac{107}{30} \\
           -\dfrac{17}{5} \\
           0
           \end{bmatrix} \quad (\text{при } z = 0)
           \]
        
        \item[2)] Значение функции невязки:  
            \[
            \psi(\theta) = \sqrt{\dfrac{1}{30}} = \dfrac{\sqrt{30}}{30}
            \]
           
    \end{enumerate}
    \textbf{Ответ:}  
    Псевдорешение: \(\begin{bmatrix} -\dfrac{107}{30} \\ -\dfrac{17}{5} \\ 0 \end{bmatrix}\), значение функции невязки: \(\dfrac{\sqrt{30}}{30}\).


\end{enumerate}
\end{document}