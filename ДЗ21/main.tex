\documentclass[a4paper]{article}
\usepackage{setspace}
\usepackage[T2A]{fontenc} %
\usepackage[utf8]{inputenc} % подключение русского языка
\usepackage[russian]{babel} %
\usepackage[12pt]{extsizes}
\usepackage{mathtools}
\usepackage{graphicx}
\usepackage{fancyhdr}
\usepackage{amssymb}
\usepackage{amsmath, amsfonts, amssymb, amsthm, mathtools}
\usepackage{tikz}

\usetikzlibrary{positioning}
\setstretch{1.3}

\newcommand{\mat}[1]{\begin{pmatrix} #1 \end{pmatrix}}
\newcommand{\matsq}[1]{\begin{bmatrix} #1 \end{bmatrix}}
\newcommand{\vmat}[1]{\begin{vmatrix} #1 \end{vmatrix}}
\renewcommand{\f}[2]{\frac{#1}{#2}}
\newcommand{\dspace}{\space\space}
\newcommand{\s}[2]{\sum\limits_{#1}^{#2}}
\newcommand{\mul}[2]{\prod_{#1}^{#2}}
\newcommand{\sq}[1]{\left[ {#1} \right]}
\newcommand{\gath}[1]{\left[ \begin{array}{@{}l@{}} #1 \end{array} \right.}
\newcommand{\case}[1]{\begin{cases} #1 \end{cases}}
\newcommand{\ts}{\text{\space}}
\newcommand{\lm}[1]{\underset{#1}{\lim}}
\newcommand{\suplm}[1]{\underset{#1}{\overline{\lim}}}
\newcommand{\inflm}[1]{\underset{#1}{\underline{\lim}}}
\newcommand{\Ker}[1]{\operatorname{Ker}}

\renewcommand{\phi}{\varphi}
\newcommand{\lr}{\Leftrightarrow}
\renewcommand{\l}{\left(}
\renewcommand{\r}{\right)}
\newcommand{\rr}{\rightarrow}
\renewcommand{\geq}{\geqslant}
\renewcommand{\leq}{\leqslant}
\newcommand{\RR}{\mathbb{R}}
\newcommand{\CC}{\mathbb{C}}
\newcommand{\QQ}{\mathbb{Q}}
\newcommand{\ZZ}{\mathbb{Z}}
\newcommand{\VV}{\mathbb{V}}
\newcommand{\NN}{\mathbb{N}}
\newcommand{\OO}{\underline{O}}
\newcommand{\oo}{\overline{o}}
\renewcommand{\Ker}{\operatorname{Ker}}
\renewcommand{\Im}{\operatorname{Im}}
\newcommand{\vol}{\text{vol}}
\newcommand{\Vol}{\text{Vol}}
\renewcommand{\d}{\;\text{d}}
\newcommand{\dx}{\;\text{dx}}

\DeclarePairedDelimiter\abs{\lvert}{\rvert} %
\makeatletter                               % \abs{}
\let\oldabs\abs                             %
\def\abs{\@ifstar{\oldabs}{\oldabs*}}       %

\begin{document}

\section*{Домашнее задание на 10.04 (Математический анализ)}
 {\large Емельянов Владимир, ПМИ гр №247}\\\\
\begin{enumerate}
    \item[\textbf{№1}]Первое равенство неверно, так как:
    $$\int \f{1}{x} \d x - \int \f{1}{x} \d x= \ln(x) + C_1 - (\ln(x) + C_2) = C_1 - C_2$$
    Второе равенство верно. Третье равенство неверно - аналогично первому.

    \item[\textbf{№2}]\begin{enumerate}
        \item[(a)]Рассмотрим интеграл:
        $$\int \f{1}{\sqrt{1-x^2}} \dx = \arcsin{x} + C_1 = -\arccos{x} + C_2 \implies$$
        $$\implies \arcsin{x} + \arccos{x} = C_2 - C_1 = C$$
        Значит, в любой точке сумма константа:
        $$\arcsin{1} + \arccos{1} = \f{\pi}{2}$$
        \textbf{Ответ:} $\f{\pi}{2}$

        \item[(b)]Рассмотрим интеграл:
        $$\int \f{1}{\sqrt{1+x^2}} \dx = \arctg{x} + C_1 = -\arctg{\f{1}{x}} + C_2 \implies$$
        $$\implies \arctg{x} + \arctg{\f{1}{x}} = C_2 - C_1 = C$$
        Значит, в любой точке сумма константа:
        $$C_2(x) = \begin{cases} 
            \frac{\pi}{2}, & x > 0, \\
            -\frac{\pi}{2}, & x < 0.
        \end{cases}$$
    \end{enumerate}


    \item[\textbf{№3}]
    \begin{enumerate}
        \item[(a)]
        Разложим:
        \[
        \frac{1}{x^4 - 1} = \frac{1}{(x^2 - 1)(x^2 + 1)} = \frac{A}{x - 1} + \frac{B}{x + 1} + \frac{C}{x^2 + 1}
        \]
        \[
        =\frac{(A + B) x^3 + (A - B + C) x^2 + (A + B) x + (A - B - C)}{(x - 1)(x + 1)(x^2 + 1)}
        \]
        Решим систему:
        $$\case{
            A+B = 0\\
            A-B+C = 0\\
            A+B = 0\\
            A-B-C = 1
        } \implies A = \frac{1}{4}, \quad B = -\frac{1}{4}, \quad C = -\frac{1}{2}$$
        Следовательно:
        \[
        \int \frac{1}{x^4 - 1}\, \dx = \int \l \frac{\frac{1}{4}}{x - 1} + \frac{-\frac{1}{4}}{x + 1} + \frac{-\frac{1}{2}}{x^2 + 1}\r \dx =
        \]
        \[
        =\int \frac{1}{x^4 - 1} \, \dx = \frac{1}{4} \ln|x - 1| - \frac{1}{4} \ln|x + 1| - \frac{1}{2} \arctg x + C
        \]
        \textbf{Ответ: } $\frac{1}{4} \ln|x - 1| - \frac{1}{4} \ln|x + 1| - \frac{1}{2} \arctg x + C$\\
 
        \item[(b)]Найдём:
        $$\int \frac{1}{\sqrt{-8 - 12x - 4x^2}} \, dx = \int \frac{1}{\sqrt{1 - t^2}} \cdot \f{1}{2} \, dt
         = \arcsin(2x+3) + C$$
         \textbf{Ответ: } $\f{1}{2} \arcsin(2x+3) + C$

        \item[(c)]Найдём:
        $$\int \frac{1}{\sin x} \, \dx = \int \frac{\sin x}{1 - \cos^2 x} \, \dx$$
        Замена \(t = \cos x\), тогда \(dt = -\sin x \, dx \implies -dt = \sin x \, dx\)
        $$-\int \frac{dt}{1 - t^2} = -\frac{1}{2} \ln \left| \frac{1 + t}{1 - t} \right| + C$$
        \[
        -\frac{1}{2} \ln \left| \frac{1 + \cos x}{1 - \cos x} \right| + C = \ln \left| \tan \frac{x}{2} \right| + C.
        \] 
        \textbf{Ответ: } $\ln \left| \tan \frac{x}{2} \right| + C$
    \end{enumerate}
\end{enumerate}
\end{document}