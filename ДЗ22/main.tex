\documentclass[a4paper]{article}
\usepackage{setspace}
\usepackage[T2A]{fontenc} %
\usepackage[utf8]{inputenc} % подключение русского языка
\usepackage[russian]{babel} %
\usepackage[12pt]{extsizes}
\usepackage{mathtools}
\usepackage{graphicx}
\usepackage{fancyhdr}
\usepackage{amssymb}
\usepackage{amsmath, amsfonts, amssymb, amsthm, mathtools}
\usepackage{tikz}

\usetikzlibrary{positioning}
\setstretch{1.3}

\newcommand{\mat}[1]{\begin{pmatrix} #1 \end{pmatrix}}
\newcommand{\matsq}[1]{\begin{bmatrix} #1 \end{bmatrix}}
\newcommand{\vmat}[1]{\begin{vmatrix} #1 \end{vmatrix}}
\renewcommand{\f}[2]{\frac{#1}{#2}}
\newcommand{\dspace}{\space\space}
\newcommand{\s}[2]{\sum\limits_{#1}^{#2}}
\newcommand{\mul}[2]{\prod_{#1}^{#2}}
\newcommand{\sq}[1]{\left[ {#1} \right]}
\newcommand{\gath}[1]{\left[ \begin{array}{@{}l@{}} #1 \end{array} \right.}
\newcommand{\case}[1]{\begin{cases} #1 \end{cases}}
\newcommand{\ts}{\text{\space}}
\newcommand{\lm}[1]{\underset{#1}{\lim}}
\newcommand{\suplm}[1]{\underset{#1}{\overline{\lim}}}
\newcommand{\inflm}[1]{\underset{#1}{\underline{\lim}}}
\newcommand{\Ker}[1]{\operatorname{Ker}}

\renewcommand{\phi}{\varphi}
\newcommand{\lr}{\Leftrightarrow}
\renewcommand{\l}{\left(}
\renewcommand{\r}{\right)}
\newcommand{\rr}{\rightarrow}
\renewcommand{\geq}{\geqslant}
\renewcommand{\leq}{\leqslant}
\newcommand{\RR}{\mathbb{R}}
\newcommand{\CC}{\mathbb{C}}
\newcommand{\QQ}{\mathbb{Q}}
\newcommand{\ZZ}{\mathbb{Z}}
\newcommand{\VV}{\mathbb{V}}
\newcommand{\NN}{\mathbb{N}}
\newcommand{\OO}{\underline{O}}
\newcommand{\oo}{\overline{o}}
\renewcommand{\Ker}{\operatorname{Ker}}
\renewcommand{\Im}{\operatorname{Im}}
\newcommand{\vol}{\text{vol}}
\newcommand{\Vol}{\text{Vol}}
\renewcommand{\d}{\;\text{d}}
\newcommand{\dx}{\;\text{dx}}

\DeclarePairedDelimiter\abs{\lvert}{\rvert} %
\makeatletter                               % \abs{}
\let\oldabs\abs                             %
\def\abs{\@ifstar{\oldabs}{\oldabs*}}       %

\begin{document}

\section*{Домашнее задание на 17.04 (Математический анализ)}
 {\large Емельянов Владимир, ПМИ гр №247}\\\\
\begin{enumerate}
    \item[\textbf{№1}]Докажем формулу:
    \[
    J_k = \int \frac{1}{(y^2 + \alpha^2)^k} \, dy = \case{
        \frac{1}{2(k-1)\alpha^2} \cdot \frac{y}{(y^2 + \alpha^2)^{k-1}} + \frac{2k - 3}{2(k-1)\alpha^2} J_{k-1}, & k\geq 2\\
        \f{1}{\alpha}\arctg \f{y}{\alpha} + C, & k=1
    }  
    \]  
    Для \( k \geq 2 \):

    По формуле интегрирования по частям:  
    \[
    J_{k-1} = \frac{y}{(y^2 + \alpha^2)^{k-1}} + 2(k-1) \int \frac{y^2}{(y^2 + \alpha^2)^k} \, dy
    \]  
    Преобразуем \( y^2 = (y^2 + \alpha^2) - \alpha^2 \):  
    \[
    \int \frac{y^2}{(y^2 + \alpha^2)^k} \, dy = J_{k-1} - \alpha^2 J_k
    \]  
    Подставляем обратно:  
    \[
    J_{k-1} = \frac{y}{(y^2 + \alpha^2)^{k-1}} + 2(k-1)(J_{k-1} - \alpha^2 J_k)
    \]  
    Решаем относительно \( J_k \):  
    \[
    J_{k-1} - 2(k-1)J_{k-1} + 2(k-1)\alpha^2 J_k = \frac{y}{(y^2 + \alpha^2)^{k-1}} 
    \]  
    \[
    - (2k - 3) J_{k-1} + 2(k-1)\alpha^2 J_k = \frac{y}{(y^2 + \alpha^2)^{k-1}}  
    \]  
    \[
    J_k = \frac{1}{2(k-1)\alpha^2} \cdot \frac{y}{(y^2 + \alpha^2)^{k-1}} + \frac{2k - 3}{2(k-1)\alpha^2} J_{k-1}
    \]  
    Для \( k = 1 \):  
    \[
    J_1 = \int \frac{1}{y^2 + \alpha^2} \, dy = \frac{1}{\alpha} \arctg \frac{y}{\alpha} + C.
    \]  \\

    \item[\textbf{№2}]\begin{enumerate}
    \item[(a)]Найдём:
    $$\int \frac{x^4}{(x+1)^3} \, dx$$
    Пусть \( t = x + 1 \), тогда \( x = t - 1 \), \( dx = dt \). Подставляем в интеграл:
    \[
    \int \frac{(t - 1)^4}{t^3} \, dt
    \]
    Раскрываем \((t - 1)^4\):
    \[
    (t - 1)^4 = t^4 - 4t^3 + 6t^2 - 4t + 1
    \]
    Делим числитель на знаменатель:
    \[
    \frac{t^4 - 4t^3 + 6t^2 - 4t + 1}{t^3} = t - 4 + \frac{6}{t} - \frac{4}{t^2} + \frac{1}{t^3}
    \]
    Тогда:
    \[
    \int \left( t - 4 + \frac{6}{t} - \frac{4}{t^2} + \frac{1}{t^3} \right) dt = \frac{t^2}{2} - 4t + 6\ln|t| + \frac{4}{t} - \frac{1}{2t^2} + C
    \]
    Заменяем \( t = x + 1 \):
    \[
    \int \frac{x^4}{(x+1)^3} = \frac{(x + 1)^2}{2} - 4(x + 1) + 6\ln|x + 1| + \frac{4}{x + 1} - \frac{1}{2(x + 1)^2} + C
    \]
    \textbf{Ответ:} $\frac{(x + 1)^2}{2} - 4(x + 1) + 6\ln|x + 1| + \frac{4}{x + 1} - \frac{1}{2(x + 1)^2} + C$\\

    \item[(b)]Найдём:
    $$\int \frac{1}{x^4 + 1} \, dx $$
    Знаменатель:
    \[
    x^4 + 1 = (x^2 + \sqrt{2}x + 1)(x^2 - \sqrt{2}x + 1)
    \]
    Значит:
    \[
    \frac{1}{x^4 + 1} = \frac{Ax + B}{x^2 + \sqrt{2}x + 1} + \frac{Cx + D}{x^2 - \sqrt{2}x + 1}
    \]
    Решая систему уравнений, получаем:  
    \[
    A = \frac{1}{2\sqrt{2}}, \quad B = \frac{1}{2}, \quad C = -\frac{1}{2\sqrt{2}}, \quad D = \frac{1}{2}
    \]
    Первая дробь:
    \[
    \int \frac{\frac{1}{2\sqrt{2}}x + \frac{1}{2}}{x^2 + \sqrt{2}x + 1} dx = \frac{1}{4\sqrt{2}} \ln|x^2 + \sqrt{2}x + 1| + \frac{1}{2\sqrt{2}} \arctg(\sqrt{2}x + 1) + C
    \]
    Вторая дробь:
    \[
    \int \frac{-\frac{1}{2\sqrt{2}}x + \frac{1}{2}}{x^2 - \sqrt{2}x + 1} dx= -\frac{1}{4\sqrt{2}} \ln|x^2 - \sqrt{2}x + 1| + \frac{1}{2\sqrt{2}} \arctg(\sqrt{2}x - 1) + C
    \]
    Получаем:
    $$\int \frac{1}{x^4 + 1} \, dx = \frac{1}{2\sqrt{2}} \arctg(\sqrt{2}x + 1) + \frac{1}{2\sqrt{2}} \arctg(\sqrt{2}x - 1) + C$$
    \textbf{Ответ:} $\frac{1}{2\sqrt{2}} \arctg(\sqrt{2}x + 1) + \frac{1}{2\sqrt{2}} \arctg(\sqrt{2}x - 1) + C$\\
    
    \item[(с)]Найдём:
    $$\int \frac{4x^2 - 8x}{(x-1)^2(x^2 + 1)^2} \, dx$$
    Представим подынтегральную функцию в виде:  
    \[
    \frac{4x^2 - 8x}{(x-1)^2(x^2 + 1)^2} = \frac{A}{x - 1} + \frac{B}{(x - 1)^2} + \frac{Cx + D}{x^2 + 1} + \frac{Ex + F}{(x^2 + 1)^2}
    \]
    Умножаем обе части на \((x-1)^2(x^2 + 1)^2\) и подставляем \(x = 1\):  
    \[
    4(1)^2 - 8(1) = B(1^2 + 1)^2 \implies -4 = 4B \implies B = -1.
    \]  
    Далее раскрываем скобки и приравниваем коэффициенты при степенях \(x\):  
    \[
    \begin{cases} 
    A + C = 0, \\
    -2A + B + D = 4, \\
    A - 2C + E = -8, \\
    -A - 2D + F = 0, \\
    C - 2E = 0, \\
    D - 2F = 0.
    \end{cases}
    \]  
    Решение системы:  
    \[
    A = 1, \, B = -1, \, C = -1, \, D = 0, \, E = -1, \, F = 0.
    \]
    Разбиваем интеграл на части:  
    \small{\[
        \int \frac{4x^2 - 8x}{(x-1)^2(x^2 + 1)^2} \, dx= \int \frac{1}{x - 1} \, dx - \int \frac{1}{(x - 1)^2} \, dx - \int \frac{x}{x^2 + 1} \, dx - \int \frac{x}{(x^2 + 1)^2} \, dx
    \]}
    Вычисляем каждую часть:  
        \begin{itemize}
        \item \(\int \frac{1}{x - 1} \, dx = \ln|x - 1|\) 
        \item \(\int \frac{1}{(x - 1)^2} \, dx = -\frac{1}{x - 1}\)  
        \item \(\int \frac{x}{x^2 + 1} \, dx = \frac{1}{2} \ln(x^2 + 1)\)  
        \item \(\int \frac{x}{(x^2 + 1)^2} \, dx = -\frac{1}{2(x^2 + 1)}\) 
    \end{itemize}

    \[
    \ln|x - 1| + \frac{1}{x - 1} - \frac{1}{2} \ln(x^2 + 1) + \frac{1}{2(x^2 + 1)} + C
    \]  
    \textbf{Ответ:} $\ln|x - 1| + \frac{1}{x - 1} - \frac{1}{2} \ln(x^2 + 1) + \frac{1}{2(x^2 + 1)} + C$\\

    \item[(d)]Найдём:
    $$\int \frac{1}{3\sin x + 4\cos x + 10} \, dx$$
    Положим \( t = \tg\frac{x}{2} \). Тогда:  
    \[
    \sin x = \frac{2t}{1 + t^2}, \quad \cos x = \frac{1 - t^2}{1 + t^2}, \quad dx = \frac{2}{1 + t^2} \, dt
    \]
    Подставим:
    \[
        \int \frac{1}{3\sin x + 4\cos x + 10} \, dx = \int \frac{1}{3 \cdot \frac{2t}{1 + t^2} + 4 \cdot \frac{1 - t^2}{1 + t^2} + 10} \cdot \frac{2}{1 + t^2} \, dt
    \]
    Приводим к общему знаменателю \( 1 + t^2 \):  
    \[
        3 \cdot \frac{2t}{1 + t^2} + 4 \cdot \frac{1 - t^2}{1 + t^2} + 10 = \frac{6t + 4(1 - t^2) + 10(1 + t^2)}{1 + t^2} =$$
        $$= \frac{6t + 4 - 4t^2 + 10 + 10t^2}{1 + t^2} = \frac{6t^2 + 6t + 14}{1 + t^2}
    \]
    Подставим:
    \[
    \int \frac{1 + t^2}{6t^2 + 6t + 14} \cdot \frac{2}{1 + t^2} \, dt = \int \frac{2}{6t^2 + 6t + 14} \, dt = \frac{1}{3} \int \frac{1}{t^2 + t + \frac{14}{6}} \, dt
    \]
    Выделим полный квадрат:
    \[
    t^2 + t + \frac{7}{3} = \left(t + \frac{1}{2}\right)^2 + \left(\sqrt{\frac{7}{3} - \frac{1}{4}}\right) = \left(t + \frac{1}{2}\right)^2 + \frac{21}{12} = \left(t + \frac{1}{2}\right)^2 + \left(\frac{\sqrt{21}}{2\sqrt{3}}\right)^2
    \]
    Следовательно:
    \[
        \frac{1}{3} \int \frac{1}{t^2 + t + \frac{14}{6}} \, dt = \frac{1}{3} \int \frac{1}{\left(t + \frac{1}{2}\right)^2 + \left(\frac{\sqrt{21}}{2\sqrt{3}}\right)^2} \, dt = $$
        $$=\frac{1}{3} \cdot \frac{2\sqrt{3}}{\sqrt{21}} \arctg\left( \frac{t + \frac{1}{2}}{\frac{\sqrt{21}}{2\sqrt{3}}} \right) + C = \frac{2}{\sqrt{21}} \arctg\left( \frac{2\sqrt{3}(t + \frac{1}{2})}{\sqrt{21}} \right) + C
    \]
    Подставляем \( t = \tg\frac{x}{2} \):  
    \[
    \frac{2}{\sqrt{21}} \arctg\left( \frac{2\tg\frac{x}{2} + 1}{\sqrt{21}} \right) + C
    \]
    \textbf{Ответ:} $\frac{2}{\sqrt{21}} \arctg\left( \frac{2\tg\frac{x}{2} + 1}{\sqrt{21}} \right) + C$\\

    \item[(e)]Найдём:
    $$\int \frac{1}{\sqrt[3]{(x+1)^2(x-1)^7}} \, dx = \int (x+1)^{-2/3} (x-1)^{-7/3} \, dx$$
    Пусть
    \[
    t = \left(\frac{x+1}{x-1}\right)^{1/3} \Rightarrow t^3 = \frac{x+1}{x-1}
    \]
    Решая относительно \(x\):
    \[
    x = \frac{t^3 + 1}{t^3 - 1}, \quad dx = \frac{-6t^2}{(t^3 - 1)^2} dt
    \]
    Подставим:
    \[
    \int t^{-2} \cdot (x-1)^{-3} \cdot \frac{-6t^2}{(t^3 - 1)^2} dt
    \]
    Упрощаем \((x-1)^{-3}\):
    \[
    (x-1)^{-3} = \left(\frac{2}{t^3 - 1}\right)^{-3} = \frac{(t^3 - 1)^3}{8}
    \]
    После сокращений:
    \[
    \frac{-3}{4} \int (t^3 - 1) dt
    \]
    Найдём:
    \[
        \frac{-3}{4} \int (t^3 - 1) dt = \frac{-3}{4} \left( \frac{t^4}{4} - t \right) + C = \frac{-3(t^4 - 4t)}{16} + C
    \]
    Подставляем \(t = \left(\frac{x+1}{x-1}\right)^{1/3}\) и упрощаем:
    \[
    \frac{-3}{16} \left( \left(\frac{x+1}{x-1}\right)^{4/3} - 4\left(\frac{x+1}{x-1}\right)^{1/3} \right) + C
    \]
    Выносим общий множитель и преобразуем:
    \[
    \frac{3(1 + x)^{1/3}(-5 + 3x)}{16(x - 1)^{4/3}} + C
    \]
    \textbf{Ответ: } $\frac{3(1 + x)^{1/3}(-5 + 3x)}{16(x - 1)^{4/3}} + C$\\

    \item[(f)]Найдём:
    $$\int \frac{1 - \sqrt{x^2 + x + 1}}{x\sqrt{x^2 + x + 1}} \, dx$$
    Положим \(\sqrt{x^2 + x + 1} = x t + 1\).  
    Возведем в квадрат:  
    \[
    x^2 + x + 1 = x^2 t^2 + 2x t + 1 \implies x = \frac{2t - 1}{1 - t^2}
    \]
    \[
    dx = \frac{2(t^2 - t + 1)}{(1 - t^2)^2} \, dt
    \]
    Разобьем интеграл на два слагаемых:  
    \[
    \int \frac{1}{x\sqrt{x^2 + x + 1}} \, dx - \int \frac{1}{x} \, dx
    \]
    Первый интеграл: 
    \[
    \int \frac{2}{2t - 1} \, dt = \ln|2t - 1| + C
    \]
    Второй интеграл:  
    \[  
    \int \frac{1}{x} \, dx = \ln|x| + C = \ln\left|\frac{2t - 1}{1 - t^2}\right| + C
    \]
    Вместе:
    \[
    \ln|2t - 1| - \ln\left|\frac{2t - 1}{1 - t^2}\right| + C = \ln|1 - t^2| + C
    \]
    Из замены \(t = \frac{\sqrt{x^2 + x + 1} - 1}{x}\):  
    \[
    1 - t^2 = \frac{2\sqrt{x^2 + x + 1} - x - 2}{x^2}
    \]
    Получаем:
    \[
    \int \frac{1 - \sqrt{x^2 + x + 1}}{x\sqrt{x^2 + x + 1}} \, dx = \ln|2\sqrt{x^2 + x + 1} - x - 2| - 2\ln|x| + C
    \]

    \textbf{Ответ: } $\ln|2\sqrt{x^2 + x + 1} - x - 2| - 2\ln|x| + C$\\
 
\end{enumerate}


\end{enumerate}
\end{document}