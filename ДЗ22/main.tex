\documentclass[a4paper]{article}
\usepackage{setspace}
\usepackage[T2A]{fontenc} %
\usepackage[utf8]{inputenc} % подключение русского языка
\usepackage[russian]{babel} %
\usepackage[12pt]{extsizes}
\usepackage{mathtools}
\usepackage{graphicx}
\usepackage{fancyhdr}
\usepackage{amssymb}
\usepackage{amsmath, amsfonts, amssymb, amsthm, mathtools}
\usepackage{tikz}

\usetikzlibrary{positioning}
\setstretch{1.3}

\newcommand{\mat}[1]{\begin{pmatrix} #1 \end{pmatrix}}
\newcommand{\matsq}[1]{\begin{bmatrix} #1 \end{bmatrix}}
\newcommand{\vmat}[1]{\begin{vmatrix} #1 \end{vmatrix}}
\renewcommand{\f}[2]{\frac{#1}{#2}}
\newcommand{\dspace}{\space\space}
\newcommand{\s}[2]{\sum\limits_{#1}^{#2}}
\newcommand{\mul}[2]{\prod_{#1}^{#2}}
\newcommand{\sq}[1]{\left[ {#1} \right]}
\newcommand{\gath}[1]{\left[ \begin{array}{@{}l@{}} #1 \end{array} \right.}
\newcommand{\case}[1]{\begin{cases} #1 \end{cases}}
\newcommand{\ts}{\text{\space}}
\newcommand{\lm}[1]{\underset{#1}{\lim}}
\newcommand{\suplm}[1]{\underset{#1}{\overline{\lim}}}
\newcommand{\inflm}[1]{\underset{#1}{\underline{\lim}}}
\newcommand{\Ker}[1]{\operatorname{Ker}}

\renewcommand{\phi}{\varphi}
\newcommand{\lr}{\Leftrightarrow}
\renewcommand{\l}{\left(}
\renewcommand{\r}{\right)}
\newcommand{\rr}{\rightarrow}
\renewcommand{\geq}{\geqslant}
\renewcommand{\leq}{\leqslant}
\newcommand{\RR}{\mathbb{R}}
\newcommand{\CC}{\mathbb{C}}
\newcommand{\QQ}{\mathbb{Q}}
\newcommand{\ZZ}{\mathbb{Z}}
\newcommand{\VV}{\mathbb{V}}
\newcommand{\NN}{\mathbb{N}}
\newcommand{\OO}{\underline{O}}
\newcommand{\oo}{\overline{o}}
\renewcommand{\Ker}{\operatorname{Ker}}
\renewcommand{\Im}{\operatorname{Im}}
\newcommand{\vol}{\text{vol}}
\newcommand{\Vol}{\text{Vol}}
\renewcommand{\d}{\;\text{d}}
\newcommand{\dx}{\;\text{dx}}

\DeclarePairedDelimiter\abs{\lvert}{\rvert} %
\makeatletter                               % \abs{}
\let\oldabs\abs                             %
\def\abs{\@ifstar{\oldabs}{\oldabs*}}       %

\begin{document}

\section*{Домашнее задание на 17.04 (Математический анализ)}
 {\large Емельянов Владимир, ПМИ гр №247}\\\\
\begin{enumerate}
    \item[\textbf{№1}]Докажем формулу:
    \[
    J_k = \int \frac{1}{(y^2 + \alpha^2)^k} \, dy = \case{
        \frac{1}{2(k-1)\alpha^2} \cdot \frac{y}{(y^2 + \alpha^2)^{k-1}} + \frac{2k - 3}{2(k-1)\alpha^2} J_{k-1}, & k\geq 2\\
        \f{1}{\alpha}\arctg \f{y}{\alpha} + C, & k=1
    }  
    \]  
    Для \( k \geq 2 \):

    По формуле интегрирования по частям:  
    \[
    J_{k-1} = \frac{y}{(y^2 + \alpha^2)^{k-1}} + 2(k-1) \int \frac{y^2}{(y^2 + \alpha^2)^k} \, dy
    \]  
    Преобразуем \( y^2 = (y^2 + \alpha^2) - \alpha^2 \):  
    \[
    \int \frac{y^2}{(y^2 + \alpha^2)^k} \, dy = J_{k-1} - \alpha^2 J_k
    \]  
    Подставляем обратно:  
    \[
    J_{k-1} = \frac{y}{(y^2 + \alpha^2)^{k-1}} + 2(k-1)(J_{k-1} - \alpha^2 J_k)
    \]  
    Решаем относительно \( J_k \):  
    \[
    J_{k-1} - 2(k-1)J_{k-1} + 2(k-1)\alpha^2 J_k = \frac{y}{(y^2 + \alpha^2)^{k-1}} 
    \]  
    \[
    - (2k - 3) J_{k-1} + 2(k-1)\alpha^2 J_k = \frac{y}{(y^2 + \alpha^2)^{k-1}}  
    \]  
    \[
    J_k = \frac{1}{2(k-1)\alpha^2} \cdot \frac{y}{(y^2 + \alpha^2)^{k-1}} + \frac{2k - 3}{2(k-1)\alpha^2} J_{k-1}
    \]  
    Для \( k = 1 \):  
    \[
    J_1 = \int \frac{1}{y^2 + \alpha^2} \, dy = \frac{1}{\alpha} \arctg \frac{y}{\alpha} + C.
    \]  \\

    \item[\textbf{№2}]\begin{enumerate}
    \item[(a)]Найдём:
    $$\int \frac{x^4}{(x+1)^3} \, dx$$
    Пусть \( t = x + 1 \), тогда \( x = t - 1 \), \( dx = dt \). Подставляем в интеграл:
    \[
    \int \frac{(t - 1)^4}{t^3} \, dt
    \]
    Раскрываем \((t - 1)^4\):
    \[
    (t - 1)^4 = t^4 - 4t^3 + 6t^2 - 4t + 1
    \]
    Делим числитель на знаменатель:
    \[
    \frac{t^4 - 4t^3 + 6t^2 - 4t + 1}{t^3} = t - 4 + \frac{6}{t} - \frac{4}{t^2} + \frac{1}{t^3}
    \]
    Тогда:
    \[
    \int \left( t - 4 + \frac{6}{t} - \frac{4}{t^2} + \frac{1}{t^3} \right) dt = \frac{t^2}{2} - 4t + 6\ln|t| + \frac{4}{t} - \frac{1}{2t^2} + C
    \]
    Заменяем \( t = x + 1 \):
    \[
    \int \frac{x^4}{(x+1)^3} = \frac{(x + 1)^2}{2} - 4(x + 1) + 6\ln|x + 1| + \frac{4}{x + 1} - \frac{1}{2(x + 1)^2} + C
    \]
    \textbf{Ответ:} $\frac{(x + 1)^2}{2} - 4(x + 1) + 6\ln|x + 1| + \frac{4}{x + 1} - \frac{1}{2(x + 1)^2} + C$\\

    \item[(b)]Найдём:
    $$\int \frac{1}{x^4 + 1} \, dx $$
    Знаменатель:
    \[
    x^4 + 1 = (x^2 + \sqrt{2}x + 1)(x^2 - \sqrt{2}x + 1)
    \]
    Значит:
    \[
    \frac{1}{x^4 + 1} = \frac{Ax + B}{x^2 + \sqrt{2}x + 1} + \frac{Cx + D}{x^2 - \sqrt{2}x + 1}
    \]
    Решая систему уравнений, получаем:  
    \[
    A = \frac{1}{2\sqrt{2}}, \quad B = \frac{1}{2}, \quad C = -\frac{1}{2\sqrt{2}}, \quad D = \frac{1}{2}
    \]
    Первая дробь:
    \[
    \int \frac{\frac{1}{2\sqrt{2}}x + \frac{1}{2}}{x^2 + \sqrt{2}x + 1} dx = \frac{1}{4\sqrt{2}} \ln|x^2 + \sqrt{2}x + 1| + \frac{1}{2\sqrt{2}} \arctg(\sqrt{2}x + 1) + C
    \]
    Вторая дробь:
    \[
    \int \frac{-\frac{1}{2\sqrt{2}}x + \frac{1}{2}}{x^2 - \sqrt{2}x + 1} dx= -\frac{1}{4\sqrt{2}} \ln|x^2 - \sqrt{2}x + 1| + \frac{1}{2\sqrt{2}} \arctg(\sqrt{2}x - 1) + C
    \]
    Получаем:
    $$\int \frac{1}{x^4 + 1} \, dx = \frac{1}{2\sqrt{2}} \arctg(\sqrt{2}x + 1) + \frac{1}{2\sqrt{2}} \arctg(\sqrt{2}x - 1) + C$$
    \textbf{Ответ:} $\frac{1}{2\sqrt{2}} \arctg(\sqrt{2}x + 1) + \frac{1}{2\sqrt{2}} \arctg(\sqrt{2}x - 1) + C$
    \end{enumerate}


\end{enumerate}
\end{document}