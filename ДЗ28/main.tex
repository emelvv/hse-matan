\documentclass[a4paper]{article}
\usepackage{setspace}
\usepackage[utf8]{inputenc}
\usepackage[russian]{babel}
\usepackage[12pt]{extsizes}
\usepackage{mathtools}
\usepackage{graphicx}
\usepackage{fancyhdr}
\usepackage{amssymb}
\usepackage{amsmath, amsfonts, amssymb, amsthm, mathtools}
\usepackage{tikz}

\usetikzlibrary{positioning}
\setstretch{1.3}

\newcommand{\mat}[1]{\begin{pmatrix} #1 \end{pmatrix}}
\newcommand{\vmat}[1]{\begin{vmatrix} #1 \end{vmatrix}}
\renewcommand{\f}[2]{\frac{#1}{#2}} 
\newcommand{\dspace}{\space\space}
\newcommand{\s}[2]{\sum\limits_{#1}^{#2}}
\newcommand{\mul}[2]{\prod_{#1}^{#2}}
\newcommand{\sq}[1]{\left[ {#1} \right]}
\newcommand{\gath}[1]{\left[ \begin{array}{@{}l@{}} #1 \end{array} \right.}
\newcommand{\case}[1]{\begin{cases} #1 \end{cases}}
\newcommand{\ts}{\text{\space}}
\newcommand{\lm}[1]{\underset{#1}{\lim}}
\newcommand{\suplm}[1]{\underset{#1}{\overline{\lim}}}
\newcommand{\inflm}[1]{\underset{#1}{\underline{\lim}}} 
\newcommand{\Ker}[1]{\operatorname{Ker}}

\renewcommand{\phi}{\varphi}
\newcommand{\lr}{\Leftrightarrow}
\renewcommand{\l}{\left(}
\renewcommand{\r}{\right)}
\newcommand{\rr}{\rightarrow}
\renewcommand{\geq}{\geqslant}
\renewcommand{\leq}{\leqslant}
\newcommand{\RR}{\mathbb{R}}
\newcommand{\CC}{\mathbb{C}}
\newcommand{\QQ}{\mathbb{Q}}
\newcommand{\ZZ}{\mathbb{Z}}
\newcommand{\VV}{\mathbb{V}}
\newcommand{\NN}{\mathbb{N}}
\newcommand{\OO}{\underline{O}}
\newcommand{\oo}{\overline{o}}
\renewcommand{\Ker}{\operatorname{Ker}}
\renewcommand{\Im}{\operatorname{Im}}
\newcommand{\vol}{\text{vol}}
\newcommand{\Vol}{\text{Vol}}

\DeclarePairedDelimiter\abs{\lvert}{\rvert} %
\makeatletter                               % \abs{}
\let\oldabs\abs                             %
\def\abs{\@ifstar{\oldabs}{\oldabs*}}       %

\begin{document}
\section*{
\scalebox{0.9}{Домашнее задание на 08.06 (Математический анализ)}
}
{\large Емельянов Владимир, ПМИ гр №247}\\\\
\begin{enumerate}
  \item[\textbf{(a)}]Дано
  $$
  \sum_{k=2}^{\infty} \frac{\sin\bigl(\sqrt{k!}\bigr)}{\,k\,\ln^2 k\,}.
  $$
  Пусть
  $$
  a_k \;=\;\frac{\sin\bigl(\sqrt{k!}\bigr)}{\,k\,\ln^2 k\,}\,,\quad k\ge2.
  $$
  проверим, сходится ли
  $$
  \sum_{k=2}^\infty |a_k| \;=\;\sum_{k=2}^\infty \frac{\bigl|\sin(\sqrt{k!})\bigr|}{\,k\ln^2 k\,}.
  $$
  для любого $x$: $\bigl|\sin x\bigr|\le 1$, имеем
  $$
  |a_k| \;=\;\frac{\bigl|\sin(\sqrt{k!})\bigr|}{\,k\,\ln^2 k\,}\;\le\;\frac{1}{\,k\,\ln^2 k\,}.
  $$
  Известно, что ряд
  $$
  \sum_{k=2}^\infty \frac{1}{\,k\,\ln^2 k\,}
  $$
  сходится. Это следует, например, из интегрального признака:
  $$\displaystyle\int_2^\infty\frac{dx}{\,x\,\ln^2 x\,}<\infty$$
  Следовательно,
  $$
  \sum_{k=2}^\infty |a_k|
  \;\le\;
  \sum_{k=2}^\infty \frac{1}{\,k\,\ln^2 k\,}
  \;<\;+\infty.
  $$ 
  \textbf{Ответ: } сходится абсолютно\\

  \item[\textbf{(b)}]Дано
  $$\sum_{k=1}^{\infty} \int_1^\f{1}{k^2} \frac{(-1)^{k(k+1)/2}}{1 + x + x^{27}} \, dx$$
  Пусть
  $$
  a_k \;=\;\int_{1}^{\frac{1}{k^2}}\;\frac{(-1)^{\frac{k(k+1)}{2}}}{\,1 + x + x^{27}\,}\;dx.
  $$
  По определению
  $$
  |a_k|
  =
  \left|\int_{1}^{\frac{1}{k^2}} \frac{(-1)^{\frac{k(k+1)}{2}}}{1 + x + x^{27}}\;dx \right|
  \;=\;
  \int_{1}^{\frac{1}{k^2}}\; \frac{1}{\,1 + x + x^{27}\,}\;dx
  $$
  Заметим, что для любого $x\in[0,1]$ имеет место неравенство
  $$
  1 + x + x^{27} \;\le\; 1 + 1 + 1 \;=\; 3,
  $$
  поэтому
  $$
  \frac{1}{\,1 + x + x^{27}\,}\;\ge\;\frac{1}{3}
  \quad\text{для всех }x\in[0,1].
  $$
  Поскольку при больших $k$ число $\tfrac{1}{k^2}$ лежит в отрезке $\bigl[0,1\bigr]$, то на всём интегрируемом отрезке $\bigl[\tfrac{1}{k^2},\,1\bigr]$ справедливо
  $$
  \frac{1}{\,1 + x + x^{27}\,}\;\ge\;\frac{1}{3}.
  $$
  Однако в выражении $|a_k|$ мы интегрируем в обратном порядке (из $1$ в $1/k^2$), поэтому
  $$
  |a_k|
  =
  \int_{1}^{\frac{1}{k^2}} \frac{1}{\,1 + x + x^{27}\,}\;dx
  \;=\;
  \int_{\frac{1}{k^2}}^{1} \frac{1}{\,1 + x + x^{27}\,}\;dx
  $$
  Следовательно, для всех $k\ge1$
  $$
  |a_k|
  \;=\;
  \int_{\frac{1}{k^2}}^{1} \frac{1}{\,1 + x + x^{27}\,}\;dx
  \;\ge\;
  \int_{\frac{1}{k^2}}^{1} \frac{1}{3}\;dx
  \;=\;
  \frac{1}{3}\,\Bigl(1 - \tfrac{1}{k^2}\Bigr).
  $$
  Из этого сразу видно, что при $k \to \infty$
  $$
  |a_k| \;\ge\; \frac{1}{3}\Bigl(1 - \tfrac{1}{k^2}\Bigr)\;\longrightarrow\;\frac{1}{3} \;>\;0.
  $$
  В частности, $\displaystyle\lim_{k\to\infty} |a_k|\neq 0$. 
  Но для сходимости любого числового ряда необходимо условие $\lim_{k\to\infty} 
  |a_k| = 0$. Значит, искомый ряд не сходится

  \textbf{Ответ: } расходится\\



  \item[\textbf{(c)}]Дано
  $$\sum_{k=2}^{\infty} \frac{(-1)^k}{k \cdot \ln k \cdot \ln \ln k} $$
  Рассмотрим ряд
  \[
  \sum_{k=2}^{\infty} \left| \frac{(-1)^k}{k \cdot \ln k \cdot \ln \ln k} \right| 
  = \sum_{k=2}^{\infty} \frac{1}{k \cdot \ln k \cdot \ln \ln k}
  \]
  Для исследования его сходимости применим интегральный признак. Проверим, является ли функция
  \[
  f(x) = \frac{1}{x \cdot \ln x \cdot \ln \ln x}
  \]
  монотонно убывающей на $[2; +\infty)$. Вычислим производную:
  \[
  f'(x) = -\frac{1 + \ln(\ln x)}{x^2 \cdot (\ln x)^2 \cdot (\ln \ln x)^2}
  \]
  Знаменатель положителен, а числитель $1 + \ln(\ln x)$:
  \begin{itemize}
    \item Для $x \geq e^e \approx 15.15$, $\ln(\ln x) \geq 1$, следовательно, числитель положителен, и $f'(x) < 0$.
    \item Для $x \in [e; e^e)$, $\ln(\ln x) \in [0; 1)$, числитель положителен, и $f'(x) < 0$.
    \item Для $x \in [2; e)$, функция $f(x)$ не определена 
  \end{itemize}
  Таким образом, $f(x)$ убывает на $[e; +\infty)$. Применим интегральный признак с 
  начальной точкой $x = e$:
  \[
  \int_{e}^{\infty} \frac{1}{x \cdot \ln x \cdot \ln \ln x} \, dx
  \]
  Сделаем замену $t = \ln \ln x$, $dt = \frac{1}{x \ln x} dx$. Интеграл преобразуется:
  \[
  \int_{t(e)}^{\infty} \frac{1}{t} \, dt = \int_{0}^{\infty} \frac{1}{t} \, dt,
  \]
  который расходится. Следовательно, ряд
   $\sum_{k=2}^{\infty} \frac{1}{k \cdot \ln k \cdot \ln \ln k}$ расходится.
    Значит, исходный ряд не сходится абсолютно.

  Ряд имеет вид:
  \[
  \sum_{k=2}^{\infty} (-1)^k a_k, \quad \text{где } a_k = 
  \frac{1}{k \cdot \ln k \cdot \ln \ln k}
  \]
  Применим признак Лейбница (Факт 3):
  \begin{enumerate}
    \item[1)] Убывание $a_k$: Для $k \geq 2$, $a_{k+1} = \frac{1}{(k+1) 
    \cdot \ln(k+1) \cdot \ln \ln(k+1)} < a_k$, так как все
    множители в знаменателе увеличиваются.
    \item[2)] Предел $a_k$: $\lim_{k \to \infty} a_k = 0$.
  \end{enumerate}

  Оба условия выполнены. Следовательно, ряд сходится условно.
  
  \textbf{Ответ: } сходится условно, но не абсолютно\\

\end{enumerate}
\end{document}