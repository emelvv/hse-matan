\documentclass[a4paper]{article}
\usepackage{setspace}
\usepackage[T2A]{fontenc} %
\usepackage[utf8]{inputenc} % подключение русского языка
\usepackage[russian]{babel} %
\usepackage[12pt]{extsizes}
\usepackage{mathtools}
\usepackage{graphicx}
\usepackage{fancyhdr}
\usepackage{amssymb}
\usepackage{amsmath, amsfonts, amssymb, amsthm, mathtools}
\usepackage{tikz}

\usetikzlibrary{positioning}
\setstretch{1.3}

\newcommand{\mat}[1]{\begin{pmatrix} #1 \end{pmatrix}}
\renewcommand{\det}[1]{\begin{vmatrix} #1 \end{vmatrix}}
\renewcommand{\f}[2]{\frac{#1}{#2}}
\newcommand{\dspace}{\space\space}
\newcommand{\s}[2]{\sum\limits_{#1}^{#2}}
\newcommand{\mul}[2]{\prod_{#1}^{#2}}
\newcommand{\sq}[1]{\left[ {#1} \right]}
\newcommand{\gath}[1]{\left[ \begin{array}{@{}l@{}} #1 \end{array} \right.}
\newcommand{\case}[1]{\begin{cases} #1 \end{cases}}
\newcommand{\ts}{\text{\space}}
\newcommand{\lm}[1]{\underset{#1}{\lim}}
\newcommand{\suplm}[1]{\underset{#1}{\overline{\lim}}}
\newcommand{\inflm}[1]{\underset{#1}{\underline{\lim}}}

\renewcommand{\phi}{\varphi}
\newcommand{\lr}{\Leftrightarrow}
\renewcommand{\r}{\Rightarrow}
\newcommand{\rr}{\rightarrow}
\renewcommand{\geq}{\geqslant}
\renewcommand{\leq}{\leqslant}
\newcommand{\RR}{\mathbb{R}}
\newcommand{\CC}{\mathbb{C}}
\newcommand{\QQ}{\mathbb{Q}}
\newcommand{\ZZ}{\mathbb{Z}}
\newcommand{\VV}{\mathbb{V}}
\newcommand{\NN}{\mathbb{N}}
\newcommand{\OO}{\underline{O}}
\newcommand{\oo}{\overline{o}}


\DeclarePairedDelimiter\abs{\lvert}{\rvert} %
\makeatletter                               % \abs{}
\let\oldabs\abs                             %
\def\abs{\@ifstar{\oldabs}{\oldabs*}}       %

\begin{document}

\section*{Домашнее задание на 06.02 (Математический анализ)}
 {\large Емельянов Владимир, ПМИ гр №247}\\\\
\begin{enumerate}
    \item[\textbf{№1}]Для функции $ f(x, y) = e^{x^2 y} + x y^2 + 1 $
     найдем частную производную по переменной $ y $ в точке $ (a, b) $ двумя способами.
    
    Согласно формуле, частная производная функции $ f $ по переменной $ y $ 
    в точке $ (a, b) $ определяется как:
    $$
    \frac{\partial f}{\partial y}(a, b) = \lim_{t \to 0} \frac{f(a, b + t) - f(a, b)}{t}
    $$
    Сначала вычислим $ f(a, b + t) $:
    $$
    f(a, b + t) = e^{a^2 (b + t)} + a (b + t)^2 + 1
    $$
    Теперь подставим это в формулу:
    $$
    f(a, b + t) - f(a, b) = \left( e^{a^2 (b + t)} - e^{a^2 b} \right) + a (b + t)^2 - a b^2
    $$
    $$
    a (b + t)^2 - a b^2 = a (b^2 + 2bt + t^2) - a b^2 = 2ab t + at^2
    $$
    $$
    f(a, b + t) - f(a, b) = \left( e^{a^2 (b + t)} - e^{a^2 b} \right) + 2ab t + at^2
    $$
    $$
    \frac{f(a, b + t) - f(a, b)}{t} = \frac{e^{a^2 (b + t)} - e^{a^2 b}}{t} + 2ab + at
    $$
    $$
    \lim_{t \to 0} \frac{e^{a^2 (b + t)} - e^{a^2 b}}{t} = \frac{d}{dt} e^{a^2 (b + t)} \bigg|_{t=0} = a^2 e^{a^2 b}
    $$
    Таким образом, получаем:
    $$
    \frac{\partial f}{\partial y}(a, b) = a^2 e^{a^2 b} + 2ab
    $$
    Теперь найдем частную производную $ f $ по $ y $, считая $ x $ константой:
    $$
    \frac{\partial f}{\partial y} = \frac{\partial}{\partial y} \left( e^{x^2 y} + x y^2 + 1 \right)
    $$
    $$
    \frac{\partial f}{\partial y} = x^2 e^{x^2 y} + 2xy
    $$
    Подставим $ x = a $ и $ y = b $:
    $$
    \frac{\partial f}{\partial y}(a, b) = a^2 e^{a^2 b} + 2ab
    $$
    \textbf{Ответ:} $ a^2 e^{a^2 b} + 2ab$

    \item[\textbf{№2}]\begin{enumerate}
        \item[(a)]$ f(x, y, z) = (x + 3y)^{2x + z} $
        
        Частная производная по $ x $:
        \[
        \ln f(x, y, z) = (2x + z) \ln(x + 3y)
        \]
        \[
        \frac{1}{f(x, y, z)} \cdot \frac{\partial f}{\partial x} = 2 \ln(x + 3y) + (2x + z) \cdot \frac{1}{x + 3y}
        \]
        \[
        \frac{\partial f}{\partial x} = f(x, y, z) \left( 2 \ln(x + 3y) + \frac{2x + z}{x + 3y} \right)
        \]
        \[
        \frac{\partial f}{\partial x} = (x + 3y)^{2x + z} \left( 2 \ln(x + 3y) + \frac{2x + z}{x + 3y} \right)
        \]

        Частная производная по $ y $:
        \[
        \frac{\partial f}{\partial y} = (2x + z) \cdot (x + 3y)^{2x + z - 1} \cdot 3
        \]
        \[
        \frac{\partial f}{\partial y} = 3(2x + z)(x + 3y)^{2x + z - 1}
        \]
        Частная производная по $ z $:
        \[
        \frac{\partial f}{\partial z} = (x + 3y)^{2x + z} \cdot \ln(x + 3y)
        \]
        \textbf{Ответ: } 
        $\begin{aligned}
        \frac{\partial f}{\partial x} &= (x + 3y)^{2x + z} \left( 2 \ln(x + 3y) + \frac{2x + z}{x + 3y} \right), \\
        \frac{\partial f}{\partial y} &= 3(2x + z)(x + 3y)^{2x + z - 1}, \\
        \frac{\partial f}{\partial z} &= (x + 3y)^{2x + z} \ln(x + 3y)
        \end{aligned}$\\

        \item[(b)]$ f(x, y, z) = z\cdot \arctg(x+\ln(y)) $
        
        Частная производная по \( x \):
        \[
        \frac{\partial f}{\partial x} = z \cdot \frac{1}{1 + (x + \ln(y))^2} \cdot \frac{\partial}{\partial x}(x + \ln(y))
        \]
        \[
        \frac{\partial f}{\partial x} = \frac{z}{1 + (x + \ln(y))^2}
        \]
        Частная производная по \( y \):
        \[
        \frac{\partial f}{\partial y} = z \cdot \frac{1}{1 + (x + \ln(y))^2} \cdot \frac{\partial}{\partial y}(x + \ln(y))
        \]
        \[
        \frac{\partial f}{\partial y} = \frac{z}{y \left(1 + (x + \ln(y))^2\right)}
        \]
        Частная производная по \( z \):
        \[
        \frac{\partial f}{\partial z} = \arctg(x + \ln(y))
        \]
        \textbf{Ответ: } $\begin{aligned}
            \frac{\partial f}{\partial x} &= \frac{z}{1 + (x + \ln(y))^2}, \\
            \frac{\partial f}{\partial y} &= \frac{z}{y \left(1 + (x + \ln(y))^2\right)}, \\
            \frac{\partial f}{\partial z} &= \arctg(x + \ln(y)).
            \end{aligned}$
    \end{enumerate}

    \item[\textbf{№3}]Рассмотрим функцию \( f(x, y) \), заданную следующим образом:
    \[
    f(x, y) = 
    \begin{cases}
    \frac{xy}{x^2 + y^2}, & (x, y) \neq (0, 0); \\
    0, & (x, y) = (0, 0)
    \end{cases}
    \]
    \begin{enumerate}
        \item[(a)]
        Для нахождения частных производных \( f_x'(0, 0) \) и \( f_y'(0, 0) \)
         воспользуемся определением производной по направлению:
        \[
        \frac{\partial f}{\partial x}(0, 0) = \lim_{t \to 0} 
        \frac{f(0 + t, 0) - f(0, 0)}{t}
        \]
        Подставим \( f(0, 0) = 0 \) и \( f(t, 0) = \frac{t \cdot 0}{t^2 + 0^2} = 0 \):
        \[
        \frac{\partial f}{\partial x}(0, 0) = \lim_{t \to 0} \frac{0 - 0}{t} = 0
        \]
        Аналогично для \( \frac{\partial f}{\partial y}(0, 0) \):
        \[
        \frac{\partial f}{\partial y}(0, 0) = 
        \lim_{t \to 0} \frac{f(0, 0 + t) - f(0, 0)}{t}
        \]
        Подставим \( f(0, 0) = 0 \) и \( f(0, t) = \frac{0 \cdot t}{0^2 + t^2} = 0 \):
        \[
        \frac{\partial f}{\partial y}(0, 0) = \lim_{t \to 0} \frac{0 - 0}{t} = 0
        \]
        Таким образом, обе частные производные в точке \((0, 0)\) существуют и равны нулю:
        \[
        f_x'(0, 0) = 0, \quad f_y'(0, 0) = 0
        \]
        \item[(b)]
        Чтобы показать, что функция \( f(x, y) \) разрывна в точке \((0, 0)\), 
        рассмотрим предел функции при приближении к \((0, 0)\) вдоль различных 
        направлений.

        Пусть \( y = kx \), где \( k \neq 0 \), тогда:
        \[
        f(x, kx) = \frac{x \cdot kx}{x^2 + (kx)^2} = \frac{kx^2}{x^2(1 + k^2)} = \frac{k}{1 + k^2}
        \]
        Предел при \( x \to 0 \):
        \[
        \lim_{x \to 0} f(x, kx) = \frac{k}{1 + k^2}
        \]
        Этот предел зависит от \( k \) и не равен нулю, если \( k \neq 0 \).
        
        Таким образом, предел функции \( f(x, y) \) при \((x, y) \to (0, 0)\) 
        зависит от направления приближения и не существует. Следовательно, функция
         \( f(x, y) \) разрывна в точке \((0, 0)\).\\
            
    \end{enumerate}

    \item[\textbf{№4}]\begin{enumerate}
        \item[(a)]\( f(x, y) = 3x^2 + 5y^2 \), \( \bar{a} = (2, -3) \), 
        \( \bar{v} = (3, -4) \)
        
        Производная по направлению вычисляется для единичного вектора \( \bar{w} \),
         сонаправленного с \( \bar{v} \). Найдем \( \bar{w} \):
        \[
        \|\bar{v}\| = \sqrt{3^2 + (-4)^2} = \sqrt{9 + 16} = 5.
        \]
        Тогда единичный вектор:
        \[
        \bar{w} = \frac{1}{\|\bar{v}\|} \bar{v} = \left( \frac{3}{5}, -\frac{4}{5} \right)
        \]
        Вычислим частные производные функции \( f(x, y) = 3x^2 + 5y^2 \):
        \[
        \frac{\partial f}{\partial x} = 6x, \quad \frac{\partial f}{\partial y} = 10y
        \]
        Вычисляем частные производные в точке \( \bar{a} = (2, -3) \)
        
        Подставляем \( \bar{a} = (2, -3) \):
        \[
        \frac{\partial f}{\partial x}(2, -3) = 6 \cdot 2 = 12, \quad \frac{\partial f}{\partial y}(2, -3) = 10 \cdot (-3) = -30
        \]
        Находим производную по направлению \( \bar{w} \)
        \[
        \frac{\partial f}{\partial \bar{w}}(2, -3) = w_1 \cdot \frac{\partial f}{\partial x}(2, -3) + w_2 \cdot \frac{\partial f}{\partial y}(2, -3)
        \]
        Подставляем \( \bar{w} = \left( \frac{3}{5}, -\frac{4}{5} \right) \):
        \[
        \frac{\partial f}{\partial \bar{w}}(2, -3) = \frac{3}{5} \cdot 12 + \left( -\frac{4}{5} \right) \cdot (-30) = \frac{36}{5} + \frac{120}{5} = \frac{156}{5}
        \]
        \textbf{Ответ: } $\frac{\partial f}{\partial \bar{w}}(2, -3) = \frac{156}{5}$\\

        \item[(b)]\( f(x, y, z) = z \sin(x + 2y) \), \( \bar{a} = \left( \frac{\pi}{4}, \frac{\pi}{8}, -2 \right) \), \( \bar{v} = (1, -1, 2) \)
        Найдем единичный вектор \( \bar{w} \), сонаправленный с \( \bar{v} \):
        \[
        \|\bar{v}\| = \sqrt{1^2 + (-1)^2 + 2^2} = \sqrt{1 + 1 + 4} = \sqrt{6}.
        \]
        Тогда единичный вектор:
        \[
        \bar{w} = \frac{1}{\|\bar{v}\|} \bar{v} = \left( \frac{1}{\sqrt{6}}, 
        -\frac{1}{\sqrt{6}}, \frac{2}{\sqrt{6}} \right)
        \]
        Вычислим частные производные функции \( f(x, y, z) = z \sin(x + 2y) \):
        \[
        \frac{\partial f}{\partial x} = z \cos(x + 2y), 
        \quad \frac{\partial f}{\partial y} = 2z \cos(x + 2y), 
        \quad \frac{\partial f}{\partial z} = \sin(x + 2y)
        \]
        Подставляем \( \bar{a} = \left( \frac{\pi}{4}, \frac{\pi}{8}, -2 \right) \):
        \[
        x + 2y = \frac{\pi}{4} + 2 \cdot \frac{\pi}{8} = \frac{\pi}{4} + \frac{\pi}{4}
         = \frac{\pi}{2}
        \]
        Тогда:
        \[
        \frac{\partial f}{\partial x}\left( \frac{\pi}{4}, \frac{\pi}{8}, -2 \right)
         = -2 \cdot \cos\left( \frac{\pi}{2} \right) = -2 \cdot 0 = 0,
        \]
        \[
        \frac{\partial f}{\partial y}\left( \frac{\pi}{4}, \frac{\pi}{8}, -2 \right)
         = 2 \cdot (-2) \cdot \cos\left( \frac{\pi}{2} \right) = -4 \cdot 0 = 0,
        \]
        \[
        \frac{\partial f}{\partial z}\left( \frac{\pi}{4}, \frac{\pi}{8}, -2 \right) 
        = \sin\left( \frac{\pi}{2} \right) = 1
        \]    
        \[
        \frac{\partial f}{\partial \bar{w}}\left( \frac{\pi}{4}, \frac{\pi}{8},
         -2 \right) = w_1 \cdot \frac{\partial f}{\partial x} + w_2 \cdot 
         \frac{\partial f}{\partial y} + w_3 \cdot \frac{\partial f}{\partial z}
        \]
        Подставляем \( \bar{w} = \left( \frac{1}{\sqrt{6}}, -\frac{1}{\sqrt{6}}, 
        \frac{2}{\sqrt{6}} \right) \):
        \[
        \frac{\partial f}{\partial \bar{w}}\left( \frac{\pi}{4},
         \frac{\pi}{8}, -2 \right) = \frac{1}{\sqrt{6}} \cdot 0 + 
         \left( -\frac{1}{\sqrt{6}} \right) \cdot 0 + \frac{2}{\sqrt{6}} \cdot 1 =
          \frac{2}{\sqrt{6}}
        \]
        \textbf{Ответ:} $\frac{\sqrt{6}}{3}$\\
    \end{enumerate}

    \item[\textbf{№5}]Рассмотрим функцию \( f(x, y) = x^2 - xy + y^2 \) 
    и найдем её производную в точке \( (1, 1) \) по направлению вектора 
    \( \bar{w}_{\lambda} = (\cos \lambda, \sin \lambda) \),
     где \( \lambda \in [0, 2\pi) \).
    
    Вычислим частные производные функции \( f(x, y)\):
    \[
    \frac{\partial f}{\partial x} = 2x - y, \quad \frac{\partial f}{\partial y} =
     -x + 2y.
    \]
    Подставляем \( (x, y) = (1, 1) \):
    \[
    \frac{\partial f}{\partial x}(1, 1) = 2 \cdot 1 - 1 = 1,
     \quad \frac{\partial f}{\partial y}(1, 1) = -1 + 2 \cdot 1 = 1.
    \]
    Используем формулу для производной по направлению:
    \[
    \frac{\partial f}{\partial \bar{w}_{\lambda}}(1, 1) = 
    w_1 \cdot \frac{\partial f}{\partial x}(1, 1) + 
    w_2 \cdot \frac{\partial f}{\partial y}(1, 1)
    \]
    Подставляем \( \bar{w}_{\lambda} = (\cos \lambda, \sin \lambda) \):
    \[
    \frac{\partial f}{\partial \bar{w}_{\lambda}}(1, 1) =
     \cos \lambda \cdot 1 + \sin \lambda \cdot 1 = \cos \lambda + \sin \lambda
    \]
    Таким образом, производная по направлению \( \bar{w}_{\lambda} \) равна:
    \[
    \frac{\partial f}{\partial \bar{w}_{\lambda}}(1, 1) = \cos \lambda + \sin \lambda
    \]
    Упростим выражение:
    \[
    \cos \lambda + \sin \lambda = \sqrt{2} \sin\left( \lambda + \frac{\pi}{4} \right).
    \]
    Максимум \( \sin \theta \) равен \( 1 \), поэтому максимум 
    \( \cos \lambda + \sin \lambda \) равен \( \sqrt{2} \). Это достигается при:
    \[
    \lambda + \frac{\pi}{4} = \frac{\pi}{2} + 2\pi k, \quad k \in \mathbb{Z}.
    \]
    Решаем для \( \lambda \):
    \[
    \lambda = \frac{\pi}{2} - \frac{\pi}{4} + 2\pi k = \frac{\pi}{4} + 2\pi k.
    \]
    В интервале \( [0, 2\pi) \) это \( \lambda = \frac{\pi}{4} \)

    Минимум \( \sin \theta \) равен \( -1 \), поэтому минимум 
    \( \cos \lambda + \sin \lambda \) равен \( -\sqrt{2} \). Это достигается при:
    \[
    \lambda + \frac{\pi}{4} = \frac{3\pi}{2} + 2\pi k, \quad k \in \mathbb{Z}.
    \]
    Решаем для \( \lambda \):
    \[
    \lambda = \frac{3\pi}{2} - \frac{\pi}{4} + 2\pi k = \frac{5\pi}{4} + 2\pi k.
    \]
    В интервале \( [0, 2\pi) \) это \( \lambda = \frac{5\pi}{4} \).

    Производная равна нулю, когда:
    \[
    \cos \lambda + \sin \lambda = 0
    \]
    Решаем уравнение:
    \[
    \cos \lambda = -\sin \lambda \implies \tan \lambda = -1
    \]
    \[
    \lambda = \frac{3\pi}{4} + \pi k, \quad k \in \mathbb{Z}.
    \]
    В интервале \( [0, 2\pi) \) это \( \lambda = \frac{3\pi}{4} \)
     и \( \lambda = \frac{7\pi}{4} \).

    \textbf{Ответ: }
    $\begin{aligned}
        \text{максимум: } & \lambda = \frac{\pi}{4}\\
        \text{минимум: } & \lambda = \frac{5\pi}{4}\\
        \text{ноль: } &\lambda = \frac{3\pi}{4} \quad \lambda = \frac{7\pi}{4}\\
    \end{aligned}$\\

    \item[\textbf{№6}]Рассмотрим поверхность, заданную неявным уравнением:
    \[
    \sqrt{x^2 + y^2 + z^2} = x + y + z - 4.
    \]
    Перепишем уравнение в виде \( F(x, y, z) = 0 \):
    \[
    F(x, y, z) = \sqrt{x^2 + y^2 + z^2} - x - y - z + 4 = 0.
    \]
    Точка \( (2, 3, 6) \) лежит на поверхности, так как:
    \[
    \sqrt{2^2 + 3^2 + 6^2} = \sqrt{4 + 9 + 36} = \sqrt{49} = 7,
    \]
    \[
    2 + 3 + 6 - 4 = 7.
    \]
    Таким образом, \( F(2, 3, 6) = 0 \)
    \begin{enumerate}
        \item[(a)]Найдём уравнение касательной плоскости, по формуле:
        \[
        F_x'(a, b, c) \cdot (x - a) + F_y'(a, b, c) \cdot (y - b) + F_z'(a, b, c) \cdot (z - c) = 0.
        \]
        Вычислим частные производные функции
         \( F(x, y, z)\):
        \[
        F_x' = \frac{x}{\sqrt{x^2 + y^2 + z^2}} - 1
        \]
        \[
        F_y' = \frac{y}{\sqrt{x^2 + y^2 + z^2}} - 1
        \]
        \[
        F_z' = \frac{z}{\sqrt{x^2 + y^2 + z^2}} - 1
        \]
        Подставляем \( (x, y, z) = (2, 3, 6) \):
        \[
        F_x'(2, 3, 6) = \frac{2}{\sqrt{2^2 + 3^2 + 6^2}} - 1 = \frac{2}{7} - 1 = -\frac{5}{7},
        \]
        \[
        F_y'(2, 3, 6) = \frac{3}{\sqrt{2^2 + 3^2 + 6^2}} - 1 = \frac{3}{7} - 1 = -\frac{4}{7},
        \]
        \[
        F_z'(2, 3, 6) = \frac{6}{\sqrt{2^2 + 3^2 + 6^2}} - 1 = \frac{6}{7} - 1 = -\frac{1}{7}.
        \]
        Подставляем \( (a, b, c) = (2, 3, 6) \) и найденные значения частных производных:
        \[
        -\frac{5}{7} \cdot (x - 2) - \frac{4}{7} \cdot (y - 3) - \frac{1}{7} \cdot (z - 6) = 0.
        \]
        \[
        -5(x - 2) - 4(y - 3) - (z - 6) = 0.
        \]
        \[
        -5x + 10 - 4y + 12 - z + 6 = 0.
        \]
        \[
        -5x - 4y - z + 28 = 0.
        \]
        \[
        5x + 4y + z = 28
        \]
        \textbf{Ответ: } $5x + 4y + z = 28$

        \item[(b)]
        Найдём уравнение нормали, по формуле:
        \[
        \frac{x - a}{F_x'(a, b, c)} = \frac{y - b}{F_y'(a, b, c)} = \frac{z - c}{F_z'(a, b, c)}.
        \]
        Подставляем \( (a, b, c) = (2, 3, 6) \) и найденные значения частных производных:
        \[
        \frac{x - 2}{-\frac{5}{7}} = \frac{y - 3}{-\frac{4}{7}} = \frac{z - 6}{-\frac{1}{7}}.
        \]
        \[
        \frac{x - 2}{-5} = \frac{y - 3}{-4} = \frac{z - 6}{-1}.
        \]
        \[
        \frac{x - 2}{5} = \frac{y - 3}{4} = \frac{z - 6}{1}.
        \]
        \textbf{Ответ: }$\frac{x - 2}{5} = \frac{y - 3}{4} = \frac{z - 6}{1}$

    \end{enumerate}

\end{enumerate}
\end{document}