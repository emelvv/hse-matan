\documentclass[a4paper]{article}
\usepackage{setspace}
\usepackage[utf8]{inputenc}
\usepackage[russian]{babel}
\usepackage[12pt]{extsizes}
\usepackage{mathtools}
\usepackage{graphicx}
\usepackage{fancyhdr}
\usepackage{amssymb}
\usepackage{amsmath, amsfonts, amssymb, amsthm, mathtools}
\usepackage{tikz}

\usetikzlibrary{positioning}
\setstretch{1.3}

\newcommand{\mat}[1]{\begin{pmatrix} #1 \end{pmatrix}}
\newcommand{\vmat}[1]{\begin{vmatrix} #1 \end{vmatrix}}
\renewcommand{\f}[2]{\frac{#1}{#2}} 
\newcommand{\dspace}{\space\space}
\newcommand{\s}[2]{\sum\limits_{#1}^{#2}}
\newcommand{\mul}[2]{\prod_{#1}^{#2}}
\newcommand{\sq}[1]{\left[ {#1} \right]}
\newcommand{\gath}[1]{\left[ \begin{array}{@{}l@{}} #1 \end{array} \right.}
\newcommand{\case}[1]{\begin{cases} #1 \end{cases}}
\newcommand{\ts}{\text{\space}}
\newcommand{\lm}[1]{\underset{#1}{\lim}}
\newcommand{\suplm}[1]{\underset{#1}{\overline{\lim}}}
\newcommand{\inflm}[1]{\underset{#1}{\underline{\lim}}} 
\newcommand{\Ker}[1]{\operatorname{Ker}}

\renewcommand{\phi}{\varphi}
\newcommand{\lr}{\Leftrightarrow}
\renewcommand{\l}{\left(}
\renewcommand{\r}{\right)}
\newcommand{\rr}{\rightarrow}
\renewcommand{\geq}{\geqslant}
\renewcommand{\leq}{\leqslant}
\newcommand{\RR}{\mathbb{R}}
\newcommand{\CC}{\mathbb{C}}
\newcommand{\QQ}{\mathbb{Q}}
\newcommand{\ZZ}{\mathbb{Z}}
\newcommand{\VV}{\mathbb{V}}
\newcommand{\NN}{\mathbb{N}}
\newcommand{\OO}{\underline{O}}
\newcommand{\oo}{\overline{o}}
\renewcommand{\Ker}{\operatorname{Ker}}
\renewcommand{\Im}{\operatorname{Im}}
\newcommand{\vol}{\text{vol}}
\newcommand{\Vol}{\text{Vol}}

\DeclarePairedDelimiter\abs{\lvert}{\rvert} %
\makeatletter                               % \abs{}
\let\oldabs\abs                             %
\def\abs{\@ifstar{\oldabs}{\oldabs*}}       %

\begin{document}

\section*{Домашнее задание на 30.05 (Математический анализ)}
{\large Емельянов Владимир, ПМИ гр №247}\\\\
\begin{enumerate}
  \item[\textbf{№1}]\begin{enumerate}
    \item[(a)]
    Выберем:  
     \[
     f'(x) = \cos(Tx) \implies f(x) = \dfrac{\sin(Tx)}{T} $$
     $$ g(x) =
      \dfrac{1}{\sqrt[3]{x^2 + 1}} \implies g'(x) = -\dfrac{2x}{3(x^2 + 1)^{4/3}}
     \]  
     Применяем формулу интегрирования по частям:  
     \[
     \int_{0}^{+\infty} \dfrac{\cos(Tx)}{\sqrt[3]{x^2 + 1}} dx = 
     \left. \dfrac{\sin(Tx)}{T \cdot \sqrt[3]{x^2 + 1}} \right|_{0}^{+\infty} +
      \dfrac{2}{3T} \int_{0}^{+\infty} \dfrac{x \sin(Tx)}{(x^2 + 1)^{4/3}} dx
     \]  
     Первое слагаемое обращается в ноль, так как:  
     \[
     \lim_{x \to +\infty} \dfrac{\sin(Tx)}{\sqrt[3]{x^2 + 1}} =
      0 \quad (\text{числитель ограничен, знаменатель растёт})
     \]
     Рассмотрим интеграл от модуля:  
     \[
     \int_{0}^{+\infty} \left| \dfrac{x \sin(Tx)}{(x^2 + 1)^{4/3}}
      \right| dx \leq \int_{0}^{+\infty} \dfrac{x}{(x^2 + 1)^{4/3}} dx
     \]  
     Для \(x \geq 1\):  
     \[
     \dfrac{x}{(x^2 + 1)^{4/3}} \sim \dfrac{x}{x^{8/3}} = \dfrac{1}{x^{5/3}}
     \]  
     Интеграл \(\int_{1}^{+\infty} \dfrac{1}{x^{5/3}} dx\) сходится, так как \(5/3 > 1\).
  
      Поскольку:  
     \[
     \dfrac{x}{(x^2 + 1)^{4/3}} \leq \dfrac{C}{x^{5/3}} \quad 
     \text{для некоторой константы } C,
     \]  
     интеграл \(\int_{0}^{+\infty} \dfrac{x}{(x^2 + 1)^{4/3}} dx\) сходится абсолютно. 
     Следовательно, исходный интеграл также сходится абсолютно
    
     \textbf{Ответ: } сходится
  \end{enumerate}
  
\end{enumerate}
\end{document}