\documentclass[a4paper]{article}
\usepackage{setspace}
\usepackage[utf8]{inputenc}
\usepackage[russian]{babel}
\usepackage[12pt]{extsizes}
\usepackage{mathtools}
\usepackage{graphicx}
\usepackage{fancyhdr}
\usepackage{amssymb}
\usepackage{amsmath, amsfonts, amssymb, amsthm, mathtools}
\usepackage{tikz}

\usetikzlibrary{positioning}
\setstretch{1.3}

\newcommand{\mat}[1]{\begin{pmatrix} #1 \end{pmatrix}}
\newcommand{\vmat}[1]{\begin{vmatrix} #1 \end{vmatrix}}
\renewcommand{\f}[2]{\frac{#1}{#2}} 
\newcommand{\dspace}{\space\space}
\newcommand{\s}[2]{\sum\limits_{#1}^{#2}}
\newcommand{\mul}[2]{\prod_{#1}^{#2}}
\newcommand{\sq}[1]{\left[ {#1} \right]}
\newcommand{\gath}[1]{\left[ \begin{array}{@{}l@{}} #1 \end{array} \right.}
\newcommand{\case}[1]{\begin{cases} #1 \end{cases}}
\newcommand{\ts}{\text{\space}}
\newcommand{\lm}[1]{\underset{#1}{\lim}}
\newcommand{\suplm}[1]{\underset{#1}{\overline{\lim}}}
\newcommand{\inflm}[1]{\underset{#1}{\underline{\lim}}} 
\newcommand{\Ker}[1]{\operatorname{Ker}}

\renewcommand{\phi}{\varphi}
\newcommand{\lr}{\Leftrightarrow}
\renewcommand{\l}{\left(}
\renewcommand{\r}{\right)}
\newcommand{\rr}{\rightarrow}
\renewcommand{\geq}{\geqslant}
\renewcommand{\leq}{\leqslant}
\newcommand{\RR}{\mathbb{R}}
\newcommand{\CC}{\mathbb{C}}
\newcommand{\QQ}{\mathbb{Q}}
\newcommand{\ZZ}{\mathbb{Z}}
\newcommand{\VV}{\mathbb{V}}
\newcommand{\NN}{\mathbb{N}}
\newcommand{\OO}{\underline{O}}
\newcommand{\oo}{\overline{o}}
\renewcommand{\Ker}{\operatorname{Ker}}
\renewcommand{\Im}{\operatorname{Im}}
\newcommand{\vol}{\text{vol}}
\newcommand{\Vol}{\text{Vol}}

\DeclarePairedDelimiter\abs{\lvert}{\rvert} %
\makeatletter                               % \abs{}
\let\oldabs\abs                             %
\def\abs{\@ifstar{\oldabs}{\oldabs*}}       %

\begin{document}

\section*{Домашнее задание на 30.05 (Математический анализ)}
{\large Емельянов Владимир, ПМИ гр №247}\\\\
\begin{enumerate}
  \item[\textbf{№1}]\begin{enumerate}
    \item[(a)]
    Выберем:  
     \[
     f'(x) = \cos(Tx) \implies f(x) = \dfrac{\sin(Tx)}{T} $$
     $$ g(x) =
      \dfrac{1}{\sqrt[3]{x^2 + 1}} \implies g'(x) = -\dfrac{2x}{3(x^2 + 1)^{4/3}}
     \]  
     Применяем формулу интегрирования по частям:  
     \[
     \int_{0}^{+\infty} \dfrac{\cos(Tx)}{\sqrt[3]{x^2 + 1}} dx = 
     \left. \dfrac{\sin(Tx)}{T \cdot \sqrt[3]{x^2 + 1}} \right|_{0}^{+\infty} +
      \dfrac{2}{3T} \int_{0}^{+\infty} \dfrac{x \sin(Tx)}{(x^2 + 1)^{4/3}} dx
     \]  
     Первое слагаемое обращается в ноль, так как:  
     \[
     \lim_{x \to +\infty} \dfrac{\sin(Tx)}{\sqrt[3]{x^2 + 1}} =
      0 \quad (\text{числитель ограничен, знаменатель растёт})
     \]
     Рассмотрим интеграл от модуля:  
     \[
     \int_{0}^{+\infty} \left| \dfrac{x \sin(Tx)}{(x^2 + 1)^{4/3}}
      \right| dx \leq \int_{0}^{+\infty} \dfrac{x}{(x^2 + 1)^{4/3}} dx
     \]  
     Для \(x \geq 1\):  
     \[
     \dfrac{x}{(x^2 + 1)^{4/3}} \sim \dfrac{x}{x^{8/3}} = \dfrac{1}{x^{5/3}}
     \]  
     Интеграл \(\int_{1}^{+\infty} \dfrac{1}{x^{5/3}} dx\) сходится, так как \(5/3 > 1\).
  
      Поскольку:  
     \[
     \dfrac{x}{(x^2 + 1)^{4/3}} \leq \dfrac{C}{x^{5/3}} \quad 
     \text{для некоторой константы } C,
     \]  
     интеграл \(\int_{0}^{+\infty} \dfrac{x}{(x^2 + 1)^{4/3}} dx\) сходится абсолютно. 
     Следовательно, исходный интеграл также сходится абсолютно
    
     \textbf{Ответ: } сходится\\

     \item[(b)]Рассмотрим интеграл как сумму двух интегралов:  
    \[
    \int_{0}^{+\infty} \frac{1}{\sqrt{e^x - 1}} \, dx = \int_{0}^{1} \frac{1}{\sqrt{e^x - 1}} \, dx + \int_{1}^{+\infty} \frac{1}{\sqrt{e^x - 1}} \, dx.
    \]
    Исследуем интеграл на \([1, +\infty)\):
    \begin{itemize}
      \item При \(x \to +\infty\):  
     \[
     \sqrt{e^x - 1} \sim \sqrt{e^x} = 
     e^{x/2} \implies \frac{1}{\sqrt{e^x - 1}} \sim e^{-x/2}
     \]  
     \item Сравним с \(g(x) = e^{-x/2}\):  
     \[
     \lim_{x \to +\infty} \frac{\frac{1}{\sqrt{e^x - 1}}}{e^{-x/2}} = 1
     \]  
     Так как \(\int_{1}^{+\infty} e^{-x/2} dx\) 
     сходится, по признаку сравнения (Факт 3) интеграл \(\int_{1}^{+\infty}
      \frac{1}{\sqrt{e^x - 1}} \, dx\) сходится.
    \end{itemize}

    Исследуем интеграл на \((0, 1]\):
    \begin{itemize}
      \item При \(x \to 0^+\):  
     \[
     e^x - 1 \approx x \implies \sqrt{e^x - 1} \approx 
     \sqrt{x} \implies \frac{1}{\sqrt{e^x - 1}} \approx \frac{1}{\sqrt{x}}
     \]  
     \item Сравним с \(g(x) = x^{-1/2}\):  
     \[
     \lim_{x \to 0^+} \frac{\frac{1}{\sqrt{e^x - 1}}}{x^{-1/2}} = 1
     \]  
     Так как \(\int_{0}^{1} x^{-1/2} dx\) сходится (степень \(x\) в знаменателе: 
     \(1/2 < 1\)), по признаку сравнения (Факт 3) интеграл 
     \(\int_{0}^{1} \frac{1}{\sqrt{e^x - 1}} \, dx\) сходится.
    \end{itemize}
    Оба интеграла сходятся, следовательно, исходный интеграл \(\int_{0}^{+\infty} \frac{1}{\sqrt{e^x - 1}} \, dx\) сходится.
    
    \textbf{Ответ: } сходится\\

    \item[(c)]При больших \(x\) величина \(\frac{1}{x}\) мала. Используем приближения:  
    \[
    \sin\left(\frac{1}{x}\right) \approx \frac{1}{x} - \frac{1}{6x^3} + \cdots
    \]  
    Тогда:  
    \[
    \ln\left(1 + \sin\left(\frac{1}{x}\right)\right) \approx
      \sin\left(\frac{1}{x}\right) - \frac{\sin^2\left(\frac{1}{x}\right)}{2} + 
      \cdots \approx \frac{1}{x} - \frac{1}{2x^2} + \cdots
    \]  
    Главный член разложения: \(\frac{1}{x}\).

    Выберем \(g(x) = \frac{1}{x}\) (n = 1). Находим предел:  
    \[
    \lim_{x \to +\infty} \frac{\ln\left(1 + \sin\left(\frac{1}{x}\right)\right)}{\frac{1}{x}} = \lim_{t \to 0^+} \frac{\ln(1 + t)}{t} = 1,
    \]  
    где \(t = \frac{1}{x}\). 
    Поскольку предел конечен и положителен, интегралы 
    \(\int_{1}^{+\infty} \ln\left(1 + \sin\left(\frac{1}{x}\right) \right)dx\) и 
    \(\int_{1}^{+\infty} \frac{1}{x} dx\) ведут себя одинаково.

    Так как \(\int_{1}^{+\infty} \frac{1}{x} dx\) 
    расходится, исходный интеграл \(\int_{1}^{+\infty}
    \ln\left(1 + \sin\left(\frac{1}{x}\right) \right) dx\) также расходится.

    \textbf{Ответ: } расходится\\

    \item[(d)]Исследование интеграла на \((0, 1]\):
    \begin{itemize}
      \item При \(x \to 0^+\):  
      \[
      \ln(1+x) \approx x \implies \frac{\ln(1+x)}{x^a} \approx \frac{x}{x^a} = x^{1 - a}
      \]  
      \item Сравним с \(g(x) = x^{1 - a}\):  
     \[
     \lim_{x \to 0^+} \frac{\ln(1+x)/x^a}{x^{1 - a}} = 1
     \]  
     Интеграл \(\int_{0}^{1} x^{1 - a} dx\) сходится при \(1 - a > -1 \implies a < 2\).

    \end{itemize}
    Исследование интеграла на \([1, +\infty)\):
    \begin{itemize}
      \item  При \(x \to +\infty\):  
      \[
      \ln(1+x) \approx \ln x \implies \frac{\ln(1+x)}{x^a} \approx \frac{\ln x}{x^a}.
      \]  
      \item  Рассмотрим два случая:
      \begin{itemize}
        \item Случай \(a > 1\):
        
         Выберем \(g(x) =
         \frac{1}{x^{a - \epsilon}}\) (\(\epsilon > 0\)).
         Так как \(\ln x = o(x^\epsilon)\), интеграл \(\int_{1}^{+\infty}
          \frac{\ln x}{x^a} dx\) сходится.  

        \item Случай \(a \leq 1\):
        
         Интеграл \(\int_{1}^{+\infty} \frac{\ln x}{x^a} dx\) расходится.
      \end{itemize}
    \end{itemize}

     Интеграл \(\int_{0}^{+\infty} \frac{\ln(1+x)}{x^a} dx\) сходится
      тогда и только тогда, когда:  
    \begin{itemize}
      \item 
      На \((0, 1]\): \(a < 2\),  
      \item 
      На \([1, +\infty)\): \(a > 1\).  
    \end{itemize}

    \textbf{Ответ: } сходится при \(1 < a < 2\)\\

    \item[(e)]Исследование интеграла на \((0, 1]\):
    \begin{itemize}
      \item При \(x \to 0^+\):  
     \[
     \arctg(ax) - \arctg(bx) \approx (a - b)x
     \]  
     Подынтегральная функция:  
     \[
     \frac{\arctg(ax) - \arctg(bx)}{x} \approx a - b
     \]  

     \item При \(a \neq b\):  
     \[
     \lim_{x \to 0^+} \frac{\arctg(ax) - \arctg(bx)}{x} = a - b \neq 0
     \]  
     Сравним с \(g(x) = 1\):  
     \[
     \int_{0}^{1} \frac{C}{x} dx \quad \text{расходится}
     \]  
    \end{itemize}
    Исследование интеграла на \([1, +\infty)\):

    При \(x \to +\infty\):  
      
    Используем тождество:  
    \[
    \arctg(ax) = \frac{\pi}{2} - \arctg\left(\frac{1}{ax}\right)
    \]  
    Разность:  
    $$
    \arctg(ax) - \arctg(bx) = \left(\frac{\pi}{2} -
     \arctg\left(\frac{1}{ax}\right)\right) - 
     \left(\frac{\pi}{2} - \arctg\left(\frac{1}{bx}\right)\right) =$$
     $$=
      \arctg\left(\frac{1}{bx}\right) - \arctg\left(\frac{1}{ax}\right)
    $$
    При \(x \to +\infty\):  
    \[
    \arctg\left(\frac{1}{ax}\right) \approx \frac{1}{ax},
     \quad \arctg\left(\frac{1}{bx}\right) \approx \frac{1}{bx}
    \]  
    Подынтегральная функция:  
    \[
    \frac{\arctg(ax) - \arctg(bx)}{x} \approx \frac{1}{x} \left(\frac{1}{bx}
     - \frac{1}{ax}\right) = \frac{a - b}{ab x^2}
    \]  
    Интеграл \(\int_{1}^{+\infty} \frac{C}{x^2} dx\) сходится.

    Получаем:
    \begin{itemize}
      \item На интервале \((0, 1]\) интеграл расходится при \(a \neq b\).  
      \item На интервале \([1, +\infty)\) интеграл сходится.  
    \end{itemize}
    \textbf{Ответ: } интеграл расходится при \(a \neq b\).\\

    \item[(f)]\begin{itemize}
      \item \(a = b\):
      
      Интеграл сводится к \(\int_{0}^{+\infty} \frac{1}{2x^a} dx\)
      \begin{itemize}
        \item 
        Сходится при \(a > 1\)
        \item
        Расходится при \(a \leq 1\)
      \end{itemize}  

      \item \(a < b\):
      \begin{itemize}
        \item Интервал \((0, 1]:\)
        Преобразуем знаменатель:  
        \[
        x^a + x^b = x^a(1 + x^{b-a}) \implies \frac{1}{x^a + x^b} 
        \approx \frac{1}{x^a} \quad \text{при } x \to 0^+
        \]  
        Интеграл \(\int_{0}^{1} \frac{1}{x^a} dx\) сходится при \(a < 1\).

        \item Интервал \([1, +\infty)\):
         Преобразуем знаменатель:  
        \[
        x^a + x^b = x^b(1 + x^{a-b}) \implies \frac{1}{x^a + x^b} 
        \approx \frac{1}{x^b} \quad \text{при } x \to +\infty
        \]  
        Интеграл \(\int_{1}^{+\infty} \frac{1}{x^b} dx\) сходится при \(b > 1\).
      \end{itemize}
    \end{itemize}

    Интеграл \(\int_{0}^{+\infty} \frac{1}{x^a + x^b} dx\) сходится
    тогда и только тогда, когда:
    \begin{itemize}
    \item 
    \(a < 1\) (для сходимости на \((0, 1]\)),
    \item
    \(b > 1\) (для сходимости на \([1, +\infty)\)).
    \end{itemize}

    \textbf{Ответ: } Интеграл сходится при  \(a < 1\)  и  \(b>1\) \\

    \item[(g)]\begin{itemize}
      \item Случай \(a = 0\):
      \[
      \int_{0}^{1} |\ln x|^0 \, dx = \int_{0}^{1} 1 \, dx = 1 \quad \text{(сходится)}
      \]
      \item Случай \(a > 0\):
      
       Особенность в точке \(x = 0\). Выполним замену \(x = e^{-t}\),
        тогда \(t = -\ln x\), \(dx = -e^{-t} dt\):  
      \[
      \int_{0}^{1} |\ln x|^a \, dx = \int_{0}^{+\infty} t^a e^{-t} \, dt
      \]  
      Это функция  сходится при
       \(a + 1 > 0 \implies a > -1\).  
      
       Значит, интеграл сходится при \(a > -1\).

      \item Случай \(a < 0\):
      
      Особенность в точке \(x = 1\). При \(x \to 1^-\):  
      \[
      |\ln x|^a \approx (1 - x)^a
      \]  
      Сравним с \(g(x) = (1 - x)^7\) (по Факту 3):  
      \[
      \lim_{x \to 1^-} \frac{(1 - x)^a}{(1 - x)^7} = \lim_{x \to 1^-} (1 - x)^{a - 7}
      \]  
      Интеграл \(\int_{0}^{1} (1 - x)^{a - 7} dx\) сходится при 
      \(a - 7 > -1 \implies a > 6\), что невозможно для \(a < 0\).  
      
      Значит, интеграл расходится при \(a < 0\). 
    \end{itemize}
    \textbf{Ответ: } сходится при всех $a > -1$\\

    \item[(h)]
  \end{enumerate}
  
  
\end{enumerate}
\end{document}