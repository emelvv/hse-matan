\documentclass[a4paper]{article}
\usepackage{setspace}
\usepackage[T2A]{fontenc} %
\usepackage[utf8]{inputenc} % подключение русского языка
\usepackage[russian]{babel} %
\usepackage[14pt]{extsizes}
\usepackage{mathtools}
\usepackage{graphicx}
\usepackage{fancyhdr}
\usepackage{amssymb}
\usepackage{amsmath, amsfonts, amssymb, amsthm, mathtools}


\setstretch{1.3}

\newcommand{\mat}[1]{\begin{pmatrix} #1 \end{pmatrix}}
\renewcommand{\f}[2]{\frac{#1}{#2}}
\newcommand{\dspace}{\space\space}
\newcommand{\s}[2]{\sum\limits_{#1}^{#2}}



\renewcommand{\geq}{\geqslant}
\renewcommand{\leq}{\leqslant}
\newcommand{\RR}{\mathbb{R}}
\newcommand{\CC}{\mathbb{C}}
\newcommand{\QQ}{\mathbb{Q}}
\newcommand{\ZZ}{\mathbb{Z}}
\newcommand{\VV}{\mathbb{V}}
\newcommand{\NN}{\mathbb{N}}


\begin{document}

\section*{Домашнее задание на 15.09.2024 (Математический Анализ)}
{\large Емельянов Владимир, ПМИ гр №247}\\\\
\begin{enumerate}
    \item[\textbf{1.}] \indent\\
    $\f{m}{n}=5,(2077) = 5+0.(2077)$\\\\
    $10^4*0.(2077)-0.(2077) = 2077,(2077)-0(2077)=2077=\\
    =(10^4-1)*0.(2077)=9999*0.(2077)\Rightarrow 0.(2077)=2077/9999 \Rightarrow\\
    \Rightarrow \f{m}{n}=5+\f{2077}{9999} = \f{5*9999+2077}{9999} = \f{52072}{9999}=\f{m}{n}$\\
    \\\textbf{Ответ:} m=52072, n=9999

    \item[\textbf{2.}] \indent
    \begin{enumerate}
        \item[(a)]
        $A_n = \s{k=0}{n} \f{5^k-2}{7^k}=\f{1-2}{1}+\f{5-2}{7}+\f{5^2-2}{7^2} + \dots + \f{5^n-2}{7^n}\Rightarrow\\
        \Rightarrow 7^nA_n = (5^0-2)*7^n+(5^1-2)*7^{n-1}+(5^2-2)*7^{n-2} + \dots + (5^{n-1}-2)*7^1 +(5^n-2)*7^0 = \\
        = 5^0*7^n-2*7^n + 5*7^{n-1}-2*7^{n-1}+5^2*7^{n-2} - 2*7^{n-2} + \dots + 5^{n-1}*7^1-2*7^2 + 5^n*7^1-2*7^0 = \\
        = -2*(7^n+7^{n-1}+7^{n-2}+\dots+7^{0}) + (5^0*7^{n}+5*7^{n-1}+5^2*7^{n-2} +\dots+5^{n-1}*7^2+5^n*7^1)$\\\\
        $(7^{0}+\dots+7^{n-2}+7^{n-1}+7^n)$ - сумма $n+1$ членов геометрической прогрессии с первым членом $7^0$ и разностью $7$ $\Rightarrow\\
        \Rightarrow (7^{0}+\dots+7^{n-2}+7^{n-1}+7^n) = \frac{7^0 \cdot (7^{n+1} - 1)}{7 - 1} = \f{7^{n+1} - 1}{6}$ \\
        $(5^0*7^{n}+5*7^{n-1}+5^2*7^{n-2} +\dots+5^{n-1}*7^2+5^n*7^1)$ - сумма $n+1$ членов геометрической прогрессии с первым членом $5^0*7^n$ и разностью $\f{5}{7}$ $\Rightarrow \\
        \Rightarrow (5^0*7^{n}+5*7^{n-1}+5^2*7^{n-2} +\dots+5^{n-1}*7^2+5^n*7^1) = \f{5^0*7^n*(\f{5}{7}^{n+1}-1)}{\f{5}{7}-1} = -\f{7^n\cdot(\f{5}{7}^{n+1}-1)}{\f{2}{7}} = -\f{7^{n+1}\cdot(\f{5}{7}^{n+1}-1)}{2}$\\\\
        $7^nA_n = -2*\f{7^{n+1} - 1}{6}-\f{7^{n+1}\cdot(\f{5}{7}^{n+1}-1)}{2}=\f{-2\cdot7^{n+1} + 2}{6}+\f{7^{n+1}\cdot(1-\f{5}{7}^{n+1})}{2}=\\
        =\f{-2\cdot7^{n+1} + 2}{6}+\f{3\cdot7^{n+1}\cdot(1-\f{5}{7}^{n+1})}{6}=\f{-2\cdot7^{n+1} + 2}{6}+\f{3\cdot7^{n+1}-3\cdot5^{n+1}}{6}=
        \f{7^{n+1}-3\cdot 5^{n+1}+2}{6} \Rightarrow\\\Rightarrow A_n = \f{7^{n+1}-3\cdot 5^{n+1}+2}{6\cdot7^n}$\dspace \textbf{Ответ: } $\f{7^{n+1}-3\cdot 5^{n+1}+2}{6\cdot7^n}$\\
        
        \item[(b)]
        $B_n=\s{k=1}{n}\f{1}{\sqrt{k+1}+\sqrt{k}}=\f{1}{\sqrt{2}+\sqrt{1}}+\f{1}{\sqrt{3}+\sqrt{2}}+\f{1}{\sqrt{4}+\sqrt{3}}+\dots+\f{1}{\sqrt{n+1}+\sqrt{n}}=\\
        = \s{k=1}{n}\f{(\sqrt{k+1}+\sqrt{k})(\sqrt{k+1}-\sqrt{k})}{\sqrt{k+1}+\sqrt{k}} = \s{k=1}{n}(\sqrt{k+1}-\sqrt{k})=\\
        = \sqrt{2}-\sqrt{1}+\sqrt{3}-\sqrt{2}+\dots+\sqrt{n}-\sqrt{n-1}+\sqrt{n+1}-\sqrt{n}=\\
        =\sqrt{n+1}-\sqrt{1}$\dspace \textbf{Ответ: } $\sqrt{n+1}-\sqrt{1}$\\

        \item[(c)]
        $C_n=\s{k=1}{n}\f{1}{k(k+1)(k+2)}=\f{1}{1(2)(3)}+\f{1}{2(3)(4)}+\f{1}{3(4)(5)}+\dots+\f{1}{n(n+1)(n+2)}$\\
        $\f{1}{k(k+1)(k+2)}=\f{A}{k}+\f{B}{k+1}+\f{C}{k+2}$, где $A, B, C \in \RR$\\
        $\f{A}{k}+\f{B}{k+1}+\f{C}{k+2}=\f{A(k+1)(k+2)+Bk(k+2)+Ck(k+1)}{k(k+1)(k+2)}=\f{1}{k(k+1)(k+2)} \Rightarrow\\
        \Rightarrow A(k+1)(k+2)+Bk(k+2)+Ck(k+1) = 1$\\
        Пусть $A=\f{1}{2}$ : $\f{1}{2}(k+1)(k+2)+Bk(k+2)+Ck(k+1)=\f{1}{2}k^2+\f{3}{2}k+1+B(k^2+2k)+C(k^2+k)\Rightarrow\\
        \Rightarrow \begin{cases}
            B+C=-\f{1}{2}\\
            2B+C=-\f{3}{2}
        \end{cases}\Rightarrow\begin{cases}
            C=-\f{1}{2}-B\\
            2B-\f{1}{2}-B=-\f{3}{2}
        \end{cases}\Rightarrow B=-\f{3}{2}+\f{1}{2}=-1 \Rightarrow C = -\f{1}{2}+1 = \f{1}{2}\Rightarrow\\
        \Rightarrow \f{1}{k(k+1)(k+2)}=\f{\f{1}{2}}{k}+\f{-1}{k+1}+\f{\f{1}{2}}{k+2}=\f{\f{1}{2}}{k}+\f{\f{1}{2}}{k+2}-\f{1}{k+1}$\\
        $\s{k=1}{n}\f{1}{k(k+1)(k+2)} = \s{k=1}{n} \f{\f{1}{2}}{k}+\f{\f{1}{2}}{k+2}-\f{1}{k+1} = \\
        =\f{\f{1}{2}}{1}+\f{\f{1}{2}-1}{2}+\f{\f{1}{2}+\f{1}{2}-1}{3}+\f{\f{1}{2}+\f{1}{2}-1}{4}+\f{\f{1}{2}+\f{1}{2}-1}{5}+\dots+\f{\f{1}{2}+\f{1}{2}-1}{n}+\f{\f{1}{2}-1}{n+1}+\f{\f{1}{2}}{n+2}=\\
        =\f{1}{2}-\f{1}{4}+0+0+0+\dots+0-\f{1}{2n+2}+\f{1}{2n+4}=\f{1}{4}-\f{1}{2n+2}+\f{1}{2n+4}=\f{n^2+3n}{4n^2+12n+8}$
        \dspace \textbf{Ответ: }$\f{n^2+3n}{4n^2+12n+8}$\\

        \item[(d)]
        $D_n=\s{k=1}{n}\f{k}{3^k}=\f{1}{3}+\f{2}{3^2}+\f{3}{3^3}+\dots+\f{n}{3^n}$\\
        $f(x)=1+x+x^2+x^3+\dots+x^n \Rightarrow\\ \Rightarrow f'(x)=0+1+2x+3x^2+\dots+nx^{n-1}\Rightarrow D_n=\f{1}{3}f'(\f{1}{3}) \\
        f(x)$ - сумма $n+1$ членов геометрической прогрессии с первым членом $1$ и разностью $x$ $\Rightarrow\\
        \Rightarrow f(x)=\f{x^{n+1}-1}{x-1} \Rightarrow f'(x)=\f{nx^{n+1}-nx^n-x^n+1}{(x-1)^2}=0+1+2x+3x^2+\dots+nx^{n-1} \Rightarrow f'(\f{1}{3})=\f{n\f{1}{3}^{n+1}-n\f{1}{3}^n-\f{1}{3}^n+1}{(\f{1}{3}-1)^2}
        =\f{n\f{1}{3}^{n+1}-n\f{1}{3}^n-\f{1}{3}^n+1}{\f{4}{9}}=\f{9n\f{1}{3}^{n+1}-9n\f{1}{3}^n-9\f{1}{3}^n+9}{4} \Rightarrow D_n=\f{1}{3}\cdot\f{9n\f{1}{3}^{n+1}-9n\f{1}{3}^n-9\f{1}{3}^n+9}{4} = \f{3n\f{1}{3}^{n+1}-3n\f{1}{3}^n-3\f{1}{3}^n+3}{4}=\\
        =\f{3-2n\f{1}{3}^n-3\f{1}{3}^n}{4}$\dspace \textbf{Ответ: }$\f{3-2n\f{1}{3}^n-3\f{1}{3}^n}{4}$\\

        \item[(e)]
        $E_n=\s{k=1}{n}a_k=a_1+a_2+a_3+\dots+a_n$, где $a_1=2, a_k=3a_{k-1}+1$ при $k=2$\\
        Докажем, что $a_k = \f{5*3^k-3}{6}$ методом математической индукции. Рассмотрим случай $k=1$ (база индукции):\\
        $a_1=\f{5*3^1-3}{6}=\f{12}{6}=2$ - верно\\
        Докажем что $a_k = \f{5*3^k-3}{6}$ верно при $a_{k-1} = \f{5*3^{k-1}-3}{6}$ (шаг индукции):\\
        $a_k=3a_{k-1}+1=3\cdot\f{5*3^{k-1}-3}{6}+1=3\cdot\f{5\cdot \f{1}{3}\cdot3^{k}-3}{6}+1= \f{5\cdot3^{k}-9}{6}+1= \f{5\cdot3^{k}-3}{6}$ - ч.т.д.\\
        Так как при k = 1 равенство верно и для любого k выполняется $a_k = \f{5*3^k-3}{6}$ при $a_{k-1} = \f{5*3^{k-1}-3}{6}$, то равенство $a_k = \f{5*3^k-3}{6}$ верно при любых $k \in \NN$. $\Rightarrow$\\
        $\Rightarrow E_n=\s{k=1}{n}a_k = \s{k=1}{n}\f{5\cdot 3^k-3}{6}=\s{k=1}{n}(\f{5}{6}\cdot 3^k - \f{1}{2})=\s{k=1}{n}(\f{5}{6}\cdot 3^k)-\s{k=1}{n}(\f{1}{2})$\\
        $\s{k=1}{n}(\f{5}{6}\cdot 3^k)$ - сумма $n$ членов геометрической прогрессии с первым членом $\f{5}{2}$ и разностью $3$ $\Rightarrow \s{k=1}{n}(\f{5}{6}\cdot 3^k) = \f{5}{2}\cdot \f{3^n-1}{3-1} = \f{5\cdot 3^n -5}{4}$\\
        $\s{k=1}{n}(\f{1}{2}) = \f{n}{2}$\\
        $E_n=\f{5\cdot 3^n -5}{4}-\f{n}{2}=\f{5\cdot 3^n -2n -5}{4}$ \dspace \textbf{Ответ: }$\f{5\cdot 3^n -2n -5}{4}$ 
        
    \end{enumerate}
    \item[\textbf{3.}] \indent
    \begin{enumerate}
        \item[(a)] Доказать: $1^2+2^2+\dots+n^2=\f{n(n+1)(2n+1)}{6}$\\
        Рассмотрим случай $n=1$ (база индукции):\\
        $1^2=\f{1(1+1)(2+1)}{6}=\f{2*3}{6}=1$ - верно\\
        Докажем что $1^2+2^2+\dots+n^2=\f{n(n+1)(2n+1)}{6}$ верно при $1^2+2^2+\dots+(n-1)^2=\f{(n-1)((n-1)+1)(2(n-1)+1)}{6}$ (шаг индукции):\\
        $1^2+2^2+\dots+(n-1)^2+n^2 = \f{(n-1)((n-1)+1)(2(n-1)+1)}{6}+n^2 = \f{(n-1)(n)(2n-1)}{6}+n^2 = \f{(n^2-n)(2n-1)}{6}+n^2 = \f{2n^3-2n^2-n^2+n}{6}+n^2 = \f{2n^3-3n^2+n}{6}+\f{6n^2}{6} = \f{2n^3+3n^2+n}{6} =  \f{n(2n^+3n+1)}{6} = \f{n(2n^+3n+1)}{6} = \f{n(n+1)(2n+1)}{6}$ - ч.т.д.\\
        Так как при n = 1 равенство верно и для любого n выполняется $1^2+2^2+\dots+n^2=\f{n(n+1)(2n+1)}{6}$ при $1^2+2^2+\dots+(n-1)^2=\f{(n-1)((n-1)+1)(2(n-1)+1)}{6}$, то равенство $1^2+2^2+\dots+n^2=\f{n(n+1)(2n+1)}{6}$ верно при любых $n \in \NN$.\\
    
        \item[(b)] Доказать: $\phi^n=\phi*F_n+F_{n-1}$\\
        Рассмотрим случай $n=1$: $\phi^1=\phi*F_1+F_0=\phi+0=\phi$ - верно\\
        Рассмотрим случай $n=2$ (база индукции): $\phi^2=\phi*F_2+F_1 = \phi(F_1+F_0)+F_1 = \phi+1$\\
        Проверим, что $\phi^2=\phi+1$: $\phi = \f{1\pm \sqrt{5}}{2}$\\
        $\begin{cases}
            \phi = \f{1+\sqrt{5}}{2}: \phi^2 = (\f{1+\sqrt{5}}{2})^2 = \f{1+2\sqrt{5}+5}{2}= \f{6+2\sqrt{5}}{2} = 1+\f{1+\sqrt{5}}{2} = 1+ \phi \text{ - верно} \\
            \phi = \f{1-\sqrt{5}}{2}: \phi^2 = (\f{1-\sqrt{5}}{2})^2 = \f{1-2\sqrt{5}+5}{2} = \f{6-2\sqrt{5}}{2} = 1+\f{1-\sqrt{5}}{2} = 1+ \phi \text{ - верно}
        \end{cases}\\ \Rightarrow $ случай n=2 - выполняется\\
        Докажем что $\phi^n=\phi*F_n+F_{n-1}$ верно при $\phi^{n-1}=\phi*F_{n-1}+F_{n-2}$ для $n \geq 2$ (шаг индукции):\\
        $F_n = F_{n-2} + F_{n-1} \Rightarrow F_{n-2} = F_n-F_{n-1}$\\
        $\phi^2 = \phi+1$ (Из второго случая) $\Rightarrow \phi = \phi^2 - 1$\\
        $\phi^{n-1} = \phi F_{n-1}+F_{n-2} \Rightarrow \phi^{n} = \phi^2 F_{n-1}+ \phi F_{n-2} = \phi^2 F_{n-1}+ \phi (F_n-F_{n-1}) = \phi^2 F_{n-1}+ \phi F_n- \phi F_{n-1} = \phi^2 F_{n-1}+ \phi F_n- (\phi^2 - 1)F_{n-1} = \phi^2 F_{n-1}+ \phi F_n- \phi^2 F_{n-1} + F_{n-1} = \phi F_n + F_{n-1}$ - ч.т.д.\\
        Так как при n = 2 равенство верно и для любого n выполняется $\phi^n=\phi*F_n+F_{n-1}$ при $\phi^{n-1}=\phi*F_{n-1}+F_{n-2}$, то равенство $\phi^n=\phi*F_n+F_{n-1}$ верно при любых $n \geq 2$, а так как мы проверили случай при $n=1$, то равенство верно при любых $n \in \NN$.\\

        \item[(c)] Доказать: $F_n < 2^n$\\
        Рассмотрим случай $n=1$: $F_1 < 2 \Leftrightarrow 1 < 2$ - верно \\
        Рассмотрим случай $n=2$: $F_2 < 4 \Leftrightarrow 1+0 < 2$ - верно \\
        Шаг индукции: докажем что $F_n < 2^n$ верно, если известно, что: \\
        $\left[ 
        \begin{gathered}
            F_{n-1} < 2^{n-1} \\
            F_{n-2} < 2^{n-2}
        \end{gathered}
        \right. \Rightarrow F_{n-1}+F_{n-2} < 2^{n-1}+2^{n-2}$\\
        $F_n = (F_{n-1}+F_{n-2}) < 2^{n-1}+2^{n-2} = \f{1}{2}\cdot 2^n + \f{1}{4}\cdot 2^n = \f{3}{4}\cdot 2^n < 2^n$ - ч.т.д.\\
        Так как при n = 1 и n = 2 равенство верно и для любого n выполняется $F_n < 2^n$ при $F_{n-1} < 2^{n-1}$ и $F_{n-2} < 2^{n-2}$, то равенство $F_n < 2^n$ верно при любых $n \in \NN$.\\


    \end{enumerate}
\end{enumerate}
\end{document}