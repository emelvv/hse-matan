\documentclass[a4paper]{article}
\usepackage{setspace}
\usepackage[T2A]{fontenc} %
\usepackage[utf8]{inputenc} % подключение русского языка
\usepackage[russian]{babel} %
\usepackage[12pt]{extsizes}
\usepackage{mathtools}
\usepackage{graphicx}
\usepackage{fancyhdr}
\usepackage{amssymb}
\usepackage{amsmath, amsfonts, amssymb, amsthm, mathtools}
\usepackage{tikz}

\usetikzlibrary{positioning}
\setstretch{1.3}

\newcommand{\mat}[1]{\begin{pmatrix} #1 \end{pmatrix}}
\renewcommand{\det}[1]{\begin{vmatrix} #1 \end{vmatrix}}
\renewcommand{\f}[2]{\frac{#1}{#2}}
\newcommand{\dspace}{\space\space}
\newcommand{\s}[2]{\sum\limits_{#1}^{#2}}
\newcommand{\sq}[1]{\left[ {#1} \right]}
\newcommand{\gath}[1]{\left[ \begin{array}{@{}l@{}} #1 \end{array} \right.}
\newcommand{\case}[1]{\begin{cases} #1 \end{cases}}
\newcommand{\ts}{\text{\space}}
\newcommand{\lm}[1]{\underset{#1}{\lim}}
\newcommand{\suplm}[1]{\underset{#1}{\overline{\lim}}}
\newcommand{\inflm}[1]{\underset{#1}{\underline{\lim}}}

\newcommand{\lr}{\Leftrightarrow}
\renewcommand{\r}{\Rightarrow}
\newcommand{\rr}{\rightarrow}
\renewcommand{\geq}{\geqslant}
\renewcommand{\leq}{\leqslant}
\newcommand{\RR}{\mathbb{R}}
\newcommand{\CC}{\mathbb{C}}
\newcommand{\QQ}{\mathbb{Q}}
\newcommand{\ZZ}{\mathbb{Z}}
\newcommand{\VV}{\mathbb{V}}
\newcommand{\NN}{\mathbb{N}}

\DeclarePairedDelimiter\abs{\lvert}{\rvert} %
\makeatletter                               % \abs{}
\let\oldabs\abs                             %
\def\abs{\@ifstar{\oldabs}{\oldabs*}}       %

\begin{document}

\section*{Домашнее задание на 22.10 (Математический анализ)}
{\large Емельянов Владимир, ПМИ гр №247}\\\\
\begin{enumerate}
    \item[\textbf{1.}]
    \begin{enumerate}
        \item[(a)]
        $\s{k=1}{\infty}\frac{2+(-1)^k\mathrm{arctan}\,k}{\sqrt[3]{k^3+1}}$
        $$\frac{2+(-1)^k\mathrm{arctan}\,k}{\sqrt[3]{k^3+1}} \geq \frac{2-(-1)^k\frac{\pi}{2}}{\sqrt[3]{k^3+k^3}} = \frac{2-(-1)^k\frac{\pi}{2}}{k\sqrt[3]{2}} \geq $$
        $$\geq \f{2-\f{\pi}{2}}{k\sqrt[3]{2}} = \f{2-\f{\pi}{2}}{\sqrt[3]{2}}\cdot \f{1}{k} $$
        $$C = \f{2-\f{\pi}{2}}{\sqrt[3]{2}} > 0$$
        Так как $\s{k=1}{\infty} \f{C}{k} = C\s{k=1}{\infty} \f{1}{k}$ - расходится (гармонический ряд), то и $\s{k=1}{\infty}\frac{2+(-1)^k\mathrm{arctan}\,k}{\sqrt[3]{k^3+1}}$ тоже расходится по признаку сравнения, так как $$\frac{2+(-1)^k\mathrm{arctan}\,k}{\sqrt[3]{k^3+1}} \geq \f{C}{k}$$
        \textbf{Ответ: } расходится \\

        \item[(b)]
        $\s{k=2}{\infty}\f{1}{(\ln{k})^{\ln{k}}}$
        $$\ln{k}^{\ln{k}} = e^{\ln{((\ln{k})^{\ln{k}})}} = e^{\ln{k}\ln{(\ln{k})}}  = k^{\ln{(\ln{k})}}$$
        Узнаем когда $k^2 <k^{\ln{(\ln{k})}} $:
        $$2 < \ln{(\ln{k})} \r e^2< \ln{k} \r e^{e^2} < k \r $$
        $\r $ при $k \to \infty$: $k^2 < k^{\ln{(\ln{k})}} $, следовательно:
        $$\f{1}{(\ln{k})^{\ln{k}}} = \f{1}{k^{\ln{(\ln{k})}}} < \f{1}{k^2} \text{ при } k > e^{e^2} \r$$
        $\r$ по признаку сравнения т.к. ряд $\s{k=2}{\infty}\f{1}{k^2}$ сходится, то и $\s{k=2}{\infty}\f{1}{(\ln{k})^{\ln{k}}}$ тоже сходится\\
        \textbf{Ответ:} сходится\\
        
        \item[(c)]
        $\s{k=2}{\infty} \f{1}{k\ln{k}}$\\
        Рассмотрим ряд: 
        $$\s{k=2}{\infty} \f{2^k}{2^k\ln{2^k}} = \s{k=2}{\infty} \f{1}{\ln{2^k}} = \s{k=2}{\infty} \f{1}{k\ln{2}} \text{ - расходится}$$
        Так как $\s{k=2}{\infty} \f{1}{k\ln{2}} \text{ - расходится}$ (гармонический ряд), $\s{k=2}{\infty} \f{1}{k\ln{k}}$ - не возрастает, и $\f{1}{k\ln{k}} > 0$, то по признаку Лобачевского-Коши ряд $\s{k=2}{\infty} \f{1}{k\ln{k}}$ тоже расходится\\
        \textbf{Ответ: } расходится\\

        \item[(d)]
        $\s{k=1}{\infty}\f{2^k}{\sqrt[k]{k}}\sin{\f{1}{3^k}}$
        $$\f{1}{3^k} \to 0 \text{ при } k \to +\infty \r \lim_{k\to \infty}\f{2^k}{\sqrt[k]{k}}\sin{\f{1}{3^k}} = \lim_{k\to \infty}\f{2^k}{\sqrt[k]{k}} \cdot  \f{1}{3^k} \text{ т.к. } \sin{\f{1}{3^k}} \sim \f{1}{3^k}  $$
        Пусть $a_k = \f{2^k}{3^k \sqrt[k]{k}}$, а $b_k = \f{2^k}{\sqrt[k]{k}} \r $
        $$\r \f{a_k}{b_k} = \f{1}{3^k}\r 0 <\f{1}{3^k} \leq \f{1}{3} \r $$
        $\r$ $\s{k=1}{\infty}a_k$ и  $\s{k=1}{\infty}b_k$ эквивалентны по сходимости по факту 2.\\
        Следовательно, так как $\s{k=1}{\infty}\f{2^k}{\sqrt[k]{k}}$ - расходится ($2^k \to \infty$, $\sqrt[k]{k} \to 1$), то и $\s{k=1}{\infty}\f{2^k}{3^k \sqrt[k]{k}}$ тоже расходится,
        следовательно, и  $\s{k=1}{\infty}\f{2^k}{\sqrt[k]{k}}\sin{\f{1}{3^k}}$ - расходится.\\
        \textbf{Ответ: } расходится\\

        \item[(e)]$\s{k=1}{\infty}\f{(k!)^2}{(2k)!}$\\
        Пусть $a_k = \f{(k!)^2}{(2k)!}$, тогда рассмотрим:
        $$\lim_{k \to \infty}\f{a_{k+1}}{a_k} = \lim_{k \to \infty}\f{((k+1)!)^2 (2k)!}{(k!)^2 (2k+2)!} = \lim_{k \to \infty}\f{(k+1)^2}{(2k+2)(2k+1)} = $$ 
        $$= \lim_{k \to \infty}\f{k+1}{4k+2} = \lim_{k \to \infty}\f{1+\f{1}{k}}{4+\f{2}{k}} = \f{1}{4}$$
        $$\lim_{k \to \infty}\f{a_{k+1}}{a_k} = \overline{\lim_{k \to \infty}}\f{a_{k+1}}{a_k} = \lim_{\overline{k \to \infty}}\f{a_{k+1}}{a_k} = \f{1}{4} \r$$
        $\r $ по факту 4 ряд $\s{k=1}{\infty}\f{(k!)^2}{(2k)!}$ сходится\\
        \textbf{Ответ: } сходится\\

        \item[(f)]$\s{k=1}{\infty}k^a b^k, \ts a, b \in \RR, \ts b>0$\\
        Пусть $c_k = k^a b^k$, рассмотрим при $b \neq 1$:
        $$\lim_{k \to \infty} \sqrt[k]{c_k} = \lim_{k \to \infty} \sqrt[k]{ k^a b^k} = \lim_{k \to \infty} b \cdot \sqrt[k]{ k^a} =b$$
        Следовательно, по факту 3 ряд $\s{k=1}{\infty}k^a b^k$ сходится при $b<1$ и расходится при $b>1$.\\
        При $b=1$:
        $$\s{k=1}{\infty}k^a b^k = \s{k=1}{\infty}k^a = \s{k=1}{\infty}\f{1}{k^{-a}} \r $$
        $\r $ ряд сходится при $-a > 1 \r a < -1$ и расходится при $-a \leq 1 \r a \geq -1$\\
        \textbf{Ответ: } $\case{
            \text{при $b < 1, a \in R$ - сходится} \\
            \text{при $b > 1, a \in R$ - расходится} \\
            \text{при $b = 1, a < -1 $ - сходится} \\
            \text{при $b = 1, a \geq -1 $ - сходится} \\
        }$\\

        \item[(g)]$\s{k=1}{\infty}\f{k^{42}(\sqrt{2}+(-1)^k)^k}{3^k}$\\
        Пусть $a_k = \f{k^{42}(\sqrt{2}+(-1)^k)^k}{3^k}$, рассмотрим:
        $$\lim_{k \to \infty} \sqrt[k]{a_k} = \lim_{k \to \infty} \sqrt[k]{\f{k^{42}(\sqrt{2}+(-1)^k)^k}{3^k}} = \lim_{k \to \infty} \f{\sqrt{2}+(-1)^k}{3}\sqrt[k]{k^{42}} \r $$
        $$\r\case{
            \suplm{k \to \infty} \f{\sqrt{2}+1}{3}\sqrt[k]{k^{42}} = \f{\sqrt{2}+1}{3} < 1\\
            \inflm{k \to \infty} \f{\sqrt{2}-1}{3}\sqrt[k]{k^{42}} = \f{\sqrt{2}-1}{3} < 1
        } \r $$
        $\r$ по факту 3 ряд $\s{k=1}{\infty}\f{k^{42}(\sqrt{2}+(-1)^k)^k}{3^k}$ сходится\\
        \textbf{Ответ: } сходится\\

        \item[(h)]$\s{k=1}{\infty} \f{k!}{(a+1)(a+2)\dots(a+k)}, \ts a \in \RR, \ts a > 0$\\
        Пусть $b_k = \f{k!}{(a+1)(a+2)\dots(a+k)}$, рассмотрим:
        $$B = \lm{k \to \infty}\ln{k}(k(\f{b_k}{b_{k+1}} - 1) - 1)=$$
        $$= \lm{k \to \infty}\ln{k}(k(\f{k!(a+1)(a+2)\dots(a+k+1)}{(a+1)(a+2)\dots(a+k)(k+1)!} - 1) - 1) = $$
        $$ = \lm{k \to \infty}\ln{k}(k(\f{a+k+1}{k+1} - 1) - 1) = \lm{k \to \infty}\ln{k}(\f{ak}{k+1} - 1) = $$
        $$ = \lm{k \to \infty}\ln{k}(\f{ak- k - 1}{k+1}) = \lm{k \to \infty}\f{(ak- k - 1)\ln{k}}{k+1} = $$
        $$= \lm{k \to \infty}\f{(a- 1 - \f{1}{k})\ln{k}}{1+\f{1}{k}} = \lm{k \to \infty}(a- 1)\ln{k}$$
        Получилось, что при $a = 1: B = 0$, а при $a \neq 1: B = +\infty$, следовательно, по факту 5, ряд сходится при $a \neq 1$ и расходится при $a = 1$.\\
        \textbf{Ответ: } $\case{ a = 1: \text{ расходится} \\ a \neq 1: \text{ сходится}}$\\
    \end{enumerate}
    
    \item[\textbf{2.}]Пусть $a_k = \f{(-1)^k}{\sqrt[3]{k}}$, тогда $b_k = \f{1}{\sqrt[3]{k}}$, рассмотрим:
    $$\f{b_{k}}{b_{k+1}} = \f{\sqrt[3]{k+1}}{\sqrt[3]{k}} = \sqrt[3]{\f{k+1}{k}} = \sqrt[3]{1+\f{1}{k}} \geq 1 \r $$
    $$\r k_{n} \geq b_{k+1} \r b_k \text{ - не возрастает}$$
    Так как $b_k \to 0$ и $b_k$ не возрастает, то по признаку Лейбница ряд $\s{k=1}{\infty}a_k = \s{k=1}{\infty}\f{(-1)^k}{\sqrt[3]{k}}$ сходится\\
    Рассмотрим ряд $\s{k=1}{\infty}(a_k)^3$:
    $$\s{k=1}{\infty}(a_k)^3 = \s{k=1}{\infty}\f{(-1)^{3k}}{k} - \text{расходится (гармонический ряд)}$$
\end{enumerate}
\end{document}