\documentclass[a4paper]{article}
\usepackage{setspace}
\usepackage[T2A]{fontenc} %
\usepackage[utf8]{inputenc} % подключение русского языка
\usepackage[russian]{babel} %
\usepackage[12pt]{extsizes}
\usepackage{mathtools}
\usepackage{graphicx}
\usepackage{fancyhdr}
\usepackage{amssymb}
\usepackage{amsmath, amsfonts, amssymb, amsthm, mathtools}
\usepackage{tikz}

\usetikzlibrary{positioning}
\setstretch{1.3}

\newcommand{\mat}[1]{\begin{pmatrix} #1 \end{pmatrix}}
\newcommand{\matsq}[1]{\begin{bmatrix} #1 \end{bmatrix}}
\newcommand{\vmat}[1]{\begin{vmatrix} #1 \end{vmatrix}}
\renewcommand{\f}[2]{\frac{#1}{#2}}
\newcommand{\dspace}{\space\space}
\newcommand{\s}[2]{\sum\limits_{#1}^{#2}}
\newcommand{\mul}[2]{\prod_{#1}^{#2}}
\newcommand{\sq}[1]{\left[ {#1} \right]}
\newcommand{\gath}[1]{\left[ \begin{array}{@{}l@{}} #1 \end{array} \right.}
\newcommand{\case}[1]{\begin{cases} #1 \end{cases}}
\newcommand{\ts}{\text{\space}}
\newcommand{\lm}[1]{\underset{#1}{\lim}}
\newcommand{\suplm}[1]{\underset{#1}{\overline{\lim}}}
\newcommand{\inflm}[1]{\underset{#1}{\underline{\lim}}}
\newcommand{\Ker}[1]{\operatorname{Ker}}

\renewcommand{\phi}{\varphi}
\newcommand{\lr}{\Leftrightarrow}
\renewcommand{\l}{\left(}
\renewcommand{\r}{\right)}
\newcommand{\rr}{\rightarrow}
\renewcommand{\geq}{\geqslant}
\renewcommand{\leq}{\leqslant}
\newcommand{\RR}{\mathbb{R}}
\newcommand{\CC}{\mathbb{C}}
\newcommand{\QQ}{\mathbb{Q}}
\newcommand{\ZZ}{\mathbb{Z}}
\newcommand{\VV}{\mathbb{V}}
\newcommand{\NN}{\mathbb{N}}
\newcommand{\OO}{\underline{O}}
\newcommand{\oo}{\overline{o}}
\renewcommand{\Ker}{\operatorname{Ker}}
\renewcommand{\Im}{\operatorname{Im}}
\newcommand{\vol}{\text{vol}}
\newcommand{\Vol}{\text{Vol}}
\renewcommand{\d}{\;\text{d}}
\newcommand{\dx}{\;\text{dx}}

\DeclarePairedDelimiter\abs{\lvert}{\rvert} %
\makeatletter                               % \abs{}
\let\oldabs\abs                             %
\def\abs{\@ifstar{\oldabs}{\oldabs*}}       %

\begin{document}

\section*{Домашнее задание на 04.05 (Математический анализ)}
 {\large Емельянов Владимир, ПМИ гр №247}\\\\
\begin{enumerate}
    \item[\textbf{№1}]Найдём:
    \[
    \int \frac{x^8 + x^7 - 7x^6 + 26x^5 - 32x^4 + 34x^3 - 20x^2 + 48x - 33}{(x+1)(x-4)^2(x^2+1)^3} \, dx
    \]
    Разложим интеграл как:
    \[
    \int \frac{P(x)}{Q(x)} dx = \frac{P_1(x)}{Q_1(x)} + \int \frac{P_2(x)}{Q_2(x)} dx
    \]
    где \( Q(x) = Q_1(x) Q_2(x) \).
    \[ Q(x) = (x+1)(x-4)^2(x^2+1)^3 \]
    \[
    Q'(x) = (x-4)(x^2+1)^2(-2 - 21x - 20x^2 + 9x^3)
    \]
    Следовательно:
    \[
    Q_1(x) = (x-4)(x^2+1)^2
    \]
    Выразим \(Q_2(x) = \frac{Q(x)}{Q_1(x)}\):
    
    \[
    Q(x) = (x+1)(x-4)^2(x^2+1)^3
    \]
    \[
    Q_1(x) = (x-4)(x^2+1)^2
    \]
    Тогда:
    \[
    Q_2(x) = \frac{(x+1)(x-4)^2(x^2+1)^3}{(x-4)(x^2+1)^2} = (x+1)(x-4)(x^2+1)
    \]
    Искомый интеграл превращается в:
    \[
    \int \frac{P(x)}{Q(x)} \, dx = \frac{P_1(x)}{Q_1(x)} + \int \frac{P_2(x)}{Q_2(x)} \, dx
    \]
    Найдём  многочлены \(P_1(x)\) и \(P_2(x)\):
    \[
    P(x) = P_1(x) \cdot Q_2(x) + P_2(x) \cdot Q_1(x)
    \]
    Пусть:
    \[
    P_1(x) = a_4 x^4 + a_3 x^3 + a_2 x^2 + a_1 x + a_0
    \quad \text{(многочлен степени $\leq$ 4)}
    \]

    \[
    P_2(x) = b_3 x^3 + b_2 x^2 + b_1 x + b_0
    \quad\text{(многочлен степени $\leq$ 3)}
    \]
    Получаем:
    \[
    P(x) = (a_4 x^4 + a_3 x^3 + a_2 x^2 + a_1 x + a_0)(x+1)(x-4)(x^2+1) $$
    $$ + (b_3 x^3 + b_2 x^2 + b_1 x + b_0)(x-4)(x^2+1)^2 = 
    \]
    \[
    -4a_0 - 4b_0
    \]
    \[
    + (-3a_0 - 4a_1 + b_0 - 4b_1)x
    \]
    \[
    + (-3a_0 - 3a_1 - 4a_2 - 8b_0 + b_1 - 4b_2)x^2
    \]
    \[
    + (-3a_0 - 3a_1 - 3a_2 - 4a_3 + 2b_0 - 8b_1 + b_2 - 4b_3)x^3
    \]
    \[
    + (a_0 - 3a_1 - 3a_2 - 3a_3 - 4a_4 - 4b_0 + 2b_1 - 8b_2 + b_3)x^4
    \]
    \[
    + (a_1 - 3a_2 - 3a_3 - 3a_4 + b_0 - 4b_1 + 2b_2 - 8b_3)x^5
    \]
    \[
    + (a_2 - 3a_3 - 3a_4 + b_1 - 4b_2 + 2b_3)x^6
    \]
    \[
    + (a_3 - 3a_4 + b_2 - 4b_3)x^7
    \]
    \[
    + (a_4 + b_3)x^8
    \]

    С другой стороны:
    \[
    P(x) = x^8 + x^7 - 7x^6 + 26x^5 - 32x^4 + 34x^3 - 20x^2 + 48x - 33
    \]
    Получаем систему:
    \[
    \begin{cases}
    a_4 + b_3 = 1,\\
    a_3 - 3a_4 + b_2 - 4b_3 = 1,\\
    a_2 - 3a_3 - 3a_4 + b_1 - 4b_2 + 2b_3 = -7,\\
    a_1 - 3a_2 - 3a_3 - 3a_4 + b_0 - 4b_1 + 2b_2 - 8b_3 = 26,\\
    a_0 - 3a_1 - 3a_2 - 3a_3 - 4a_4 - 4b_0 + 2b_1 - 8b_2 + b_3 = -32,\\
    -3a_0 - 3a_1 - 3a_2 - 4a_3 + 2b_0 - 8b_1 + b_2 - 4b_3 = 34,\\
    -3a_0 - 3a_1 - 4a_2 - 8b_0 + b_1 - 4b_2 = -20,\\
    -3a_0 - 4a_1 + b_0 - 4b_1 = 48,\\
    -4a_0 - 4b_0 = -33.
    \end{cases}
    \]
    \textbf{У системы нет решений}\\

    \item[\textbf{№2}]\begin{enumerate}
        \item[(a)] Для интеграла \(\int_a^b f(a + b - x) \, dx\) выполним замену \(u = a + b - x\). Тогда \(du = -dx\), пределы интегрирования меняются местами:  
        \[
        \int_a^b f(a + b - x) \, dx = \int_b^a f(u) (-du) = \int_a^b f(u) \, du = \int_a^b f(x) \, dx.
        \]  
        Таким образом, \(\int_a^b f(x) \, dx = \int_a^b f(a + b - x) \, dx\).\\

        \item[(b)]Для интеграла \(\int_0^{\pi/2} f(\cos x) \, dx\) выполним замену \(x = \frac{\pi}{2} - t\). Тогда \(dx = -dt\), пределы интегрирования меняются:  
        \[
        \int_0^{\pi/2} f(\sin x) \, dx = \int_{\pi/2}^0 f\left(\sin\left(\frac{\pi}{2} - t\right)\right) (-dt) = \int_0^{\pi/2} f(\cos t) \, dt.
        \]  
        Следовательно, \(\int_0^{\pi/2} f(\sin x) \, dx = \int_0^{\pi/2} f(\cos x) \, dx\).\\
        
        \item[(c)]Интеграл \(\int_0^{\pi} f(\sin x) \, dx\) разобьем на два:  
        \[
        \int_0^{\pi} f(\sin x) \, dx = \int_0^{\pi/2} f(\sin x) \, dx + \int_{\pi/2}^{\pi} f(\sin x) \, dx.
        \]  
        Во втором интеграле выполним замену \(x = \pi - t\):  
        \[
        \int_{\pi/2}^{\pi} f(\sin x) \, dx = \int_0^{\pi/2} f(\sin(\pi - t)) \, dt = \int_0^{\pi/2} f(\sin t) \, dt.
        \]  
        Таким образом:  
        \[
        \int_0^{\pi} f(\sin x) \, dx = 2 \int_0^{\pi/2} f(\sin x) \, dx.
        \]\\

        \item[(d)]Для периодической функции \(f\) с периодом \(T\) интеграл по интервалу длины \(T\) не зависит от выбора начала интервала:  
        \[
        \int_a^b f(x) \, dx = \int_0^T f(x) \, dx, \quad \text{где } b - a = T.
        \]
        Следовательно:
        $$\int_a^b f(x) \, dx = \int_0^T f(x) \, dx$$\\
    \end{enumerate}

    \item[\textbf{№3}]\begin{enumerate}
        \item[(a)]Используем гиперболическую замену \(x = \sqrt{5} \, \text{sh} \, t\). Тогда:  
        \[
        dx = \sqrt{5} \, \text{ch} \, t \, dt, \quad \sqrt{5 + x^2} = \sqrt{5} \, \text{ch} \, t.
        \]  
        Интеграл преобразуется:  
        \[
        \int_0^2 \frac{\sqrt{5} \, \text{ch} \, t}{2 + 5 \, \text{sh}^2 t} \cdot \sqrt{5} \, \text{ch} \, t \, dt = 5 \int \frac{\text{ch}^2 t}{2 + 5 \, \text{sh}^2 t} \, dt.
        \]  
        Используя тождество \(\text{ch}^2 t = 1 + \text{sh}^2 t\):  
        \[
        5 \int \frac{1 + \text{sh}^2 t}{2 + 5 \, \text{sh}^2 t} \, dt = 5 \left( \int \frac{1}{2 + 5 \, \text{sh}^2 t} \, dt + \int \frac{\text{sh}^2 t}{2 + 5 \, \text{sh}^2 t} \, dt \right).
        \]  
        После вычислений (подстановка \(u = \text{sh} \, t\)) и возврата к \(x\):  
        \[
        \sqrt{5} \, \text{arsh}\left(\frac{x}{\sqrt{5}}\right) - 
        \arctan\left(\frac{x}{\sqrt{2}}\right) + C
        \]\\

        \item[(b)]Применяем универсальную замену \(t = \tg \frac{x}{2}\):  
        \[
        \int_{-\pi}^\pi \frac{2 \, dt}{(1 + t^2)(6t^2 + 4t + 4)}.
        \]  
        После упрощения:  
        \[
        \frac{1}{\sqrt{5}} \arctan\left(\frac{3t + 1}{\sqrt{5}}\right) \Bigg|_{-\infty}^{+\infty} = \frac{\pi}{\sqrt{5}}
        \]  \\

        \item[(c)]Пусть \(I = \int \frac{\sqrt{2} + 1}{\sqrt{2} - 1} \sin x \, dx\). Замена \(x = \frac{1}{t}\) даёт:  
        \[
        J = \int (\sqrt{2} + 1) \sin\left(\frac{1}{t}\right) \cdot \frac{-dt}{t^2}.
        \]  
        Суммируя \(I + J\) и упрощая:  
        \[
        2I = \int \left(\frac{\sqrt{2} + 1}{\sqrt{2} - 1} \sin x + (\sqrt{2} + 1) \sin\left(\frac{1}{x}\right)\right) dx = 0 \implies I = 0.
        \]  \\

        \item[(d)]Из условия \(h(g(x)) = x\):  
        \[
        h'(g(x)) = \frac{1}{g'(x)} = e^x + xe^{3x}.
        \]  
        Интегрируем:  
        \[
        \int_0^T (e^x + xe^{3x}) \, dx = e^T + \frac{T e^{3T}}{3} - \frac{e^{3T}}{9} + \frac{1}{9}.
        \]  

    \end{enumerate}

\end{enumerate}
\end{document}