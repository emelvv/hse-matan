\documentclass[a4paper]{article}
\usepackage{setspace}
\usepackage[T2A]{fontenc} %
\usepackage[utf8]{inputenc} % подключение русского языка
\usepackage[russian]{babel} %
\usepackage[12pt]{extsizes}
\usepackage{mathtools}
\usepackage{graphicx}
\usepackage{fancyhdr}
\usepackage{amssymb}
\usepackage{amsmath, amsfonts, amssymb, amsthm, mathtools}
\usepackage{tikz}

\usetikzlibrary{positioning}
\setstretch{1.3}

\newcommand{\mat}[1]{\begin{pmatrix} #1 \end{pmatrix}}
\renewcommand{\det}[1]{\begin{vmatrix} #1 \end{vmatrix}}
\renewcommand{\f}[2]{\frac{#1}{#2}}
\newcommand{\dspace}{\space\space}
\newcommand{\s}[2]{\sum\limits_{#1}^{#2}}
\newcommand{\mul}[2]{\prod_{#1}^{#2}}
\newcommand{\sq}[1]{\left[ {#1} \right]}
\newcommand{\gath}[1]{\left[ \begin{array}{@{}l@{}} #1 \end{array} \right.}
\newcommand{\case}[1]{\begin{cases} #1 \end{cases}}
\newcommand{\ts}{\text{\space}}
\newcommand{\lm}[1]{\underset{#1}{\lim}}
\newcommand{\suplm}[1]{\underset{#1}{\overline{\lim}}}
\newcommand{\inflm}[1]{\underset{#1}{\underline{\lim}}}

\renewcommand{\phi}{\varphi}
\newcommand{\lr}{\Leftrightarrow}
\renewcommand{\r}{\Rightarrow}
\newcommand{\rr}{\rightarrow}
\renewcommand{\geq}{\geqslant}
\renewcommand{\leq}{\leqslant}
\newcommand{\RR}{\mathbb{R}}
\newcommand{\CC}{\mathbb{C}}
\newcommand{\QQ}{\mathbb{Q}}
\newcommand{\ZZ}{\mathbb{Z}}
\newcommand{\VV}{\mathbb{V}}
\newcommand{\NN}{\mathbb{N}}
\newcommand{\OO}{\underline{O}}
\newcommand{\oo}{\overline{o}}


\DeclarePairedDelimiter\abs{\lvert}{\rvert} %
\makeatletter                               % \abs{}
\let\oldabs\abs                             %
\def\abs{\@ifstar{\oldabs}{\oldabs*}}       %

\begin{document}

\section*{Домашнее задание на 23.01 (Математический анализ)}
 {\large Емельянов Владимир, ПМИ гр №247}\\\\
\begin{enumerate}
    \item[\textbf{№1}]Найдем промежутки монотонности и экстремумы для каждой из заданных функций.
    \begin{enumerate}
        \item[(a)]$ f(x) = \frac{3x - 7}{(x^2 - 1)^2} $
        Найдём производную:
        $$
        f'(x) = \frac{(3)(x^2 - 1)^2 - (3x - 7)(2(x^2 - 1)(2x))}{(x^2 - 1)^4} =
        $$
        $$
        = \frac{3(x^2 - 1)^2 - 4x(3x - 7)(x^2 - 1)}{(x^2 - 1)^4} = \frac{(x^2 - 1)(3(x^2 - 1) - 4x(3x - 7))}{(x^2 - 1)^4} =
        $$
        $$
        =\frac{-9x^2 +28x-3}{(x^2 - 1)^3} = \frac{(x-3)(1-9x)}{(x^2 - 1)^3}
        $$

        Найдём на каких промежутках функция $f$ возрастает, а на каких убывает:
        $$
        f'(x) = \frac{(x-3)(1-9x)}{(x^2 - 1)^3} \leq 0 \implies  x \in (-\infty, -1) \cup \left[\frac{1}{9}, 1\right) \cup [3;+\infty)
        $$
        $$
        f'(x) = \frac{(x-3)(1-9x)}{(x^2 - 1)^3} \geq 0 \implies x \in \left(\left.-1, \frac{1}{9}\right]\right. \cup (1, 3]
        $$
        Следовательно, возрастает на:
        $$x \in \left(\left.-1, \frac{1}{9}\right]\right. \cup (1, 3]$$
        И убывает на:
        $$x \in (-\infty, -1) \cup \left[\frac{1}{9}, 1\right) \cup [3;+\infty)$$
        Точки экстремумов:
        $$\{\f{1}{9}, 3\}$$

        \item[(b)] $ f(x) = \begin{cases} x^{x \ln x}, & x > 0 \\ 1, & x = 0 \end{cases} $\\
        
        Найдём производную $f$ при $x>0$:
        $$f'(x) = (e^{x\ln^2(x)})' = (x\ln^2(x))'e^{x\ln^2(x)} = (\ln^2(x)+2\ln(x))e^{x\ln^2(x)} = $$
        $$= \ln(x)(\ln(x)+2)e^{x\ln^2(x)} = (\ln(x)-\ln(1))(\ln(x)-\ln(e^{-2}))e^{x\ln^2(x)}$$
        Найдём промежутки монотонности функции $f$:
        $$f'(x) = (\ln(x)-\ln(1))(\ln(x)-\ln(e^{-2}))e^{x\ln^2(x)} \geq 0 \lr$$
        $$\lr(x-1)(x-e^{-2})e^{x\ln^2(x)}\geq 0\text{ по м. рационализации} \r$$
        $$\r x\in (0,e^{-2}] \cup [1; +\infty)$$
        $$f'(x) = (\ln(x)-\ln(1))(\ln(x)-\ln(e^{-2}))e^{x\ln^2(x)} \leq 0 \lr$$
        $$\lr(x-1)(x-e^{-2})e^{x\ln^2(x)}\leq 0\text{ по м. рационализации} \r$$
        $$\r x\in [e^{-2};1]$$
        Следовательно, функция убывает при:
        $$x\in [e^{-2};1]$$
        и возрастает при:
        $$x\in (0,e^{-2}] \cup [1; +\infty)$$
        Точки экстремумов:
        $$\{0, e^{-2}, 1\}$$\\
    \end{enumerate}

    \item[\textbf{№2}]Для функции 
    $$
    f(x)=\left\{\begin{array}{ll}
    \sin x, & x \geqslant 0 \\
    x^{2}, & x<0
    \end{array}\right.
    $$
    \begin{enumerate}
        \item[(a)]Мы знаем, что $\sin x$ возрастает на:
        $$x \in [\pi k; \f{\pi (2k+1)}{2}] \quad \forall k\in \ZZ$$
        А убывает на:
        $$x \in [\f{\pi (2k-1)}{2}; \pi k] \quad \forall k\in \ZZ$$
        Также мы знаем, что $x^2$ возрастает на $$x\geq 0$$
        и убывает на:
        $$x \leq 0$$
        Следовательно, $f$ возрастает на:
        $$x \in [\pi k; \f{\pi (2k+1)}{2}]\quad  \forall k\geq0, k\in \ZZ$$
        и убывает на:
        $$x \in (-\infty;0]\cup[\f{\pi (2k-1)}{2}; \pi k] \quad \forall k>0,k \in \ZZ$$
        Точки экстремумов:
        $$\left\{\pi k \right\} k\geq0, k \in \ZZ$$\\
        
        \item[(b)]$\sin x$ выпуклый вниз при:
        $$x \in [\pi (2k-1); \pi 2k] \quad \forall k\in \ZZ$$
        и выпуклый вверх при:
        $$x \in [\pi 2k; \pi (2k+1)] \quad \forall k\in \ZZ$$
        $x^2$ всегда выпуклый вниз.

        Следовательно, $f$ выпуклая вниз при:
        $$x\in (-\infty;0]\cup[\pi (2k-1); \pi 2k]  \quad \forall k>0, k\in \ZZ$$
        и выпуклая вверх при:
        $$x \in [\pi 2k; \pi (2k+1)] \quad \forall k\geq0, k\in \ZZ$$\\

    \end{enumerate}

    \item[\textbf{№3}]Докажем неравенства:
    \begin{enumerate}
        \item[(a)]$1+\frac{x}{2}-\frac{x^{2}}{8}<\sqrt{1+x}<1+\frac{x}{2}$ для всех $x>0$
        
        Разложим $\sqrt{1+x}$ по формуле Тейлора до 0 степени:
        $$f(x) = \sqrt{1+x} = f(0)+f'(c)x, \text{ где $c \in (0, x)$} $$
        $$f(x) = \sqrt{1+x} = 1+\f{x}{2\sqrt{1+c}} < 1+\f{x}{2}$$
        Разложим $\sqrt{1+x}$ по формуле Тейлора до 1 степени:
        $$f(x) = \sqrt{1+x} = f(0)+f'(0)x+\f{f''(c)}{2}x^2 = $$
        $$=1+\f{x}{2}-\f{1}{8(c+1)^{\f{3}{2}}}x^2 > 1+\frac{x}{2}-\frac{x^{2}}{8}$$
        Следовательно,
        $$1+\frac{x}{2}-\frac{x^{2}}{8}<\sqrt{1+x}<1+\frac{x}{2} \text{ для всех $x>0$}$$

        \item[(b)]$\f{b-a}{b} < \ln \f{b}{a} < \f{b-a}{a}$ для всех $0 < a < b$
        
        Разложим $f(x) = \ln(x)$ по формуле Тейлора:
        $$f(x) = \ln(x) = f(1) + f'(c)(x-1) = \f{x-1}{c}, \text{ где $c \in (0;x)$}$$
        Сравним $\f{b-a}{b}$ и $\ln \f{b}{a}$:
        $$\ln \f{b}{a} = \ln b - \ln a = \f{b-1}{c_1}-\f{a-1}{c_2} < \f{b-1}{a}-\f{a-1}{a} = \f{b-a}{a}$$
        Сравним $\ln \f{b}{a}$ и $\f{b-a}{a}$:
        $$\ln \f{b}{a} = \f{b-1}{c_1}-\f{a-1}{c_2} > \f{b-1}{b}-\f{a-1}{b} = \f{b-a}{b}$$
    \end{enumerate}

    \item[\textbf{№4}]\begin{enumerate}
        \item[(a)] Докажем, что $f(x) = 3x^x = 3e^{x\ln x}$ - выпуклая.
        $$f'(x) = 3\,{x}^{x}\,\ln\left(x\right)+3\,{x}^{x} \implies $$
        $$\implies f''(x) = 3\,{x}^{x}\,\ln^{2}\left(x\right)+6\,{x}^{x}\,\ln\left(x\right)+3\,{x}^{x-1}\,\left(x+1\right) \geq 0 \text{ при } x > 0 $$
        Воспользуемся неравенством Йенсена для выпуклой при $x>0$ функции $f(x) = 3x^x$, положив $n=3$, $\lambda_i = \f{1}{3}$:
        $$x^x+y^y+z^z = \f{1}{3}f(x)+\f{1}{3}f(y)+\f{1}{3}f(z) \geq f(\f{1}{3}(x+y+z)) = f(\pi)$$
        $$f(\pi)= 3\pi^\pi > 36 \cdot 3 = 108$$

        \item[(b)] Докажем, что $f(x) = e^x$ - выпуклая.
        $$f'(x) = e^x \implies f''(x) = e^x \geq 0 \implies f(x) \text{ - выпуклая}$$
        Воспользуемся неравенством Йенсена для выпуклой функции $f(x) = e^x$, положив $\lambda_i = \f{1}{n}$:
        $$e^{\f{\ln x_1 + \dots + \ln x_n}{n}} = \sqrt[n]{x_1 x_2 \dots x_n} \leq \f{x_1+x_2+\dots + x_n}{n}$$

    \end{enumerate}
    \item[\textbf{№5}]Для нахождения максимально возможного значения суммы $\sin A + \sin B + \sin C$, где $A$, $B$ и $C$ — углы треугольника, воспользуемся неравенством Йенсена.
    
    Углы треугольника $A$, $B$ и $C$ удовлетворяют условию:
    $$
    A + B + C = \pi
    $$
    Функция $\sin x$ является выпуклой на интервале $[0, \pi]$. По неравенству Йенсена для выпуклой функции имеем:
    $$
    \sin A + \sin B + \sin C \leq 3 \sin\left(\frac{A + B + C}{3}\right).
    $$
    Подставим значение суммы углов:
    $$
    \sin A + \sin B + \sin C \leq 3 \sin\left(\frac{\pi}{3}\right) = \frac{3\sqrt{3}}{2}
    $$
    Равенство в неравенстве Йенсена достигается, когда $A = B = C$. В случае треугольника это возможно, когда $A = B = C = 60^\circ$.

    \textbf{Ответ: } $\frac{3\sqrt{3}}{2}$
\end{enumerate}
\end{document}