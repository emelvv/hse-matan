\documentclass[a4paper]{article}
\usepackage{setspace}
\usepackage[utf8]{inputenc}
\usepackage[russian]{babel}
\usepackage[12pt]{extsizes}
\usepackage{mathtools}
\usepackage{graphicx}
\usepackage{fancyhdr}
\usepackage{amssymb}
\usepackage{amsmath, amsfonts, amssymb, amsthm, mathtools}
\usepackage{tikz}

\usetikzlibrary{positioning}
\setstretch{1.3}

\newcommand{\mat}[1]{\begin{pmatrix} #1 \end{pmatrix}}
\newcommand{\vmat}[1]{\begin{vmatrix} #1 \end{vmatrix}}
\renewcommand{\f}[2]{\frac{#1}{#2}} 
\newcommand{\dspace}{\space\space}
\newcommand{\s}[2]{\sum\limits_{#1}^{#2}}
\newcommand{\mul}[2]{\prod_{#1}^{#2}}
\newcommand{\sq}[1]{\left[ {#1} \right]}
\newcommand{\gath}[1]{\left[ \begin{array}{@{}l@{}} #1 \end{array} \right.}
\newcommand{\case}[1]{\begin{cases} #1 \end{cases}}
\newcommand{\ts}{\text{\space}}
\newcommand{\lm}[1]{\underset{#1}{\lim}}
\newcommand{\suplm}[1]{\underset{#1}{\overline{\lim}}}
\newcommand{\inflm}[1]{\underset{#1}{\underline{\lim}}}
\newcommand{\Ker}[1]{\operatorname{Ker}}

\renewcommand{\phi}{\varphi}
\newcommand{\lr}{\Leftrightarrow}
\renewcommand{\l}{\left(}
\renewcommand{\r}{\right)}
\newcommand{\rr}{\rightarrow}
\renewcommand{\geq}{\geqslant}
\renewcommand{\leq}{\leqslant}
\newcommand{\RR}{\mathbb{R}}
\newcommand{\CC}{\mathbb{C}}
\newcommand{\QQ}{\mathbb{Q}}
\newcommand{\ZZ}{\mathbb{Z}}
\newcommand{\VV}{\mathbb{V}}
\newcommand{\NN}{\mathbb{N}}
\newcommand{\OO}{\underline{O}}
\newcommand{\oo}{\overline{o}}
\renewcommand{\Ker}{\operatorname{Ker}}
\renewcommand{\Im}{\operatorname{Im}}
\newcommand{\vol}{\text{vol}}
\newcommand{\Vol}{\text{Vol}}

\DeclarePairedDelimiter\abs{\lvert}{\rvert} %
\makeatletter                               % \abs{}
\let\oldabs\abs                             %
\def\abs{\@ifstar{\oldabs}{\oldabs*}}       %

\begin{document}

\section*{Домашнее задание на 15.05 (Математический анализ)}
{\large Емельянов Владимир, ПМИ гр №247}\\\\
\begin{enumerate}
  \item[\textbf{№1}]
  Найдём площадь фигуры, ограниченной кривыми
  $$
  y^2 = 2\lambda x,
  \qquad
  x^2 = 2\lambda y,
  $$
  Ищем общие точки решений системы
  $$
  \begin{cases}
  y^2 = 2\lambda x,\\
  x^2 = 2\lambda y,
  \end{cases}
  \qquad x\ge0,\;y\ge0.
  \quad \implies (x, y) = (0, 0), (2\lambda, 2\lambda)$$
  
  В первой четверти на отрезке $x\in[0,2\lambda]$ верхней ветвью фигуры является
  $$
  y=f(x)=\sqrt{2\lambda x},
  $$
  нижней:
  $$
  y=g(x)=\frac{x^2}{2\lambda}.
  $$
  По Факту 1 Формула (1) для площади $S$ зоны между графиками $y=f(x)$ и $y=g(x)$ на $[a,b]$ имеет вид
  $$
  S=\int_a^b\bigl(f(x)-g(x)\bigr)\,dx.
  $$
  Подставляем $a=0$, $b=2\lambda$:
  $$
  S=\int_0^{2\lambda}\!\Bigl(\sqrt{2\lambda x}-\frac{x^2}{2\lambda}\Bigr)\,dx
  =\int_0^{2\lambda}(2\lambda x)^{1/2}\,dx\;-\;\frac1{2\lambda}\int_0^{2\lambda}x^2\,dx.
  $$
  Следовательно
  $$
  S=\frac{8}{3}\lambda^2-\frac{4}{3}\lambda^2
  =\frac{4}{3}\,\lambda^2
  $$
  \textbf{Ответ: }$\frac{4}{3}\,\lambda^2$\\

  \item[\textbf{№2}]
  Найдём площадь фигуры, ограниченной:
  $$
  r^2 \;=\; 2\,\lambda^2\cos2\varphi.
  $$
  Лемниската задана уравнением 
  $$r^2\ge0$$
  то есть 
  $$\cos2\varphi\ge0$$
  Это выполняется при
  $$
  -\,\frac\pi4\;\le\;\varphi\;\le\;\frac\pi4
  \quad\text{и}\quad
  \frac{3\pi}4\;\le\;\varphi\;\le\;\frac{5\pi}4
  $$
  По факту 2 для кривой $r=f(\varphi)$, непер­ерывной на $[\alpha,\beta]$, площадь сектора

  $$
  S_{\rm sector}
  =\frac12\int_{\alpha}^{\beta}f(\varphi)^2\,d\varphi
  $$
  Для одного лепестка \(r^2=2\lambda^2\cos2\varphi\), \(\varphi\in[-\pi/4,\pi/4]\):  
  $$
  S
  =\frac12\int_{-\pi/4}^{\pi/4}2\lambda^2\cos2\varphi\,d\varphi
  =\lambda^2\int_{-\pi/4}^{\pi/4}\cos2\varphi\,d\varphi
  $$
  Вычислим интеграл:
  $$
  \int_{-\pi/4}^{\pi/4}\cos2\varphi\,d\varphi
  =\Bigl[\tfrac12\sin2\varphi\Bigr]_{-\pi/4}^{\pi/4}
  =\tfrac12\bigl(\sin\tfrac\pi2-\sin(-\tfrac\pi2)\bigr)
  =\tfrac12(1-(-1))=1
  $$
  Значит
  $$
  S=\lambda^2\cdot1=\lambda^2
  $$
  Лемниската состоит из двух равных лепестков, поэтому
  $$
  S_{\rm total}=2\,S=2\lambda^2
  $$
  \textbf{Ответ: } $2\lambda^2$\\

  \item[\textbf{№3}]Найдём длину окружности
  $$
  x^2 + y^2 = \lambda^2.
  $$
  В полярных координатах окружность радиуса $\lambda$ задаётся просто:
  $$
  r( \phi)=\lambda,\qquad  \phi\in[0,2\pi].
  $$
  По Факту 4 длина кривой
  $$
  r=f( \phi),\; \phi\in[\alpha,\beta]
  $$
  в полярных координатах равна
  $$
  L=\int_{\alpha}^{\beta}\sqrt{\bigl(f'( \phi)\bigr)^2+\bigl(f( \phi)\bigr)^2}\;d \phi.
  $$
  Здесь \(f( \phi)=\lambda\) константа, поэтому \(f'( \phi)=0\). Подставляем в формулу с \(\alpha=0\), \(\beta=2\pi\):  
  $$
  L
  =\int_{0}^{2\pi}\sqrt{0^2+\lambda^2}\;d \phi
  =\int_{0}^{2\pi}\lambda\;d \phi
  =\lambda\bigl[ \phi\bigr]_{0}^{2\pi}
  =2\pi\lambda
  $$
  \textbf{Ответ: }$2\pi\lambda$\\

  \item[\textbf{№4}]Найдём длину винтовой линии
  \[
  \begin{aligned}
  x(t) &= \lambda \cos t, & y(t) &= \lambda \sin t, & z(t) &= 7\lambda t, 
  \quad t\in[0,2\pi], \\[6pt]
  x'(t) &= -\lambda \sin t, & y'(t) &= \lambda \cos t, & z'(t) &= 7\lambda.
  \end{aligned}
  \]
  По формуле 3:
  \[
  \begin{aligned}
  L &= \int_{0}^{2\pi} \sqrt{(x'(t))^2 + (y'(t))^2 + (z'(t))^2}\,dt
    = \int_{0}^{2\pi} \sqrt{\lambda^2\sin^2t + \lambda^2\cos^2t + (7\lambda)^2}\,dt \\[6pt]
    &= \int_{0}^{2\pi} \sqrt{\lambda^2(\sin^2t + \cos^2t) + 49\lambda^2}\,dt
    = \int_{0}^{2\pi} \sqrt{50\,\lambda^2}\,dt
    = \int_{0}^{2\pi} 5\sqrt{2}\,\lambda \,dt \\[6pt]
    &= 5\sqrt{2}\,\lambda \,\bigl[t\bigr]_{0}^{2\pi}
    = 5\sqrt{2}\,\lambda \cdot 2\pi
    = 10\pi\sqrt{2}\,\lambda
  \end{aligned}
  \]
  \textbf{Ответ: } $10\pi\sqrt{2}\,\lambda$\\

  \item[\textbf{№5}]Объём тела вращения по Формуле 6:
    \[
  \begin{aligned}
  V &= \pi \int_{0}^{7} \bigl(\sqrt{x e^{-x}}\bigr)^2 \,dx 
        = \pi \int_{0}^{7} x e^{-x} \,dx, \\[6pt]
  \int_{0}^{7} x e^{-x}\,dx 
    &= \Bigl[-(x+1)e^{-x}\Bigr]_{0}^{7}
      = \bigl[-(7+1)e^{-7}\bigr] - \bigl[-(0+1)e^{0}\bigr] \\[4pt]
    &= -8e^{-7} + 1
  \end{aligned}
  \]
  \textbf{Ответ: } $-8e^{-7} + 1$\\

  \item[\textbf{№6}]По общей формуле объёма через поперечные сечения:
  \[
  V \;=\;\int_{-a}^{a} S(x)\,dx,
  \]
  По Формуле 1 площадь эллипса:
  \[
  S(x) \;=\;\pi\,b\,c\;\bigl(1 - \tfrac{x^2}{a^2}\bigr).
  \]
  Интегрируем:
  \[
  \begin{aligned}
  V &= \pi\,b\,c \int_{-a}^{a} \Bigl(1 - \frac{x^2}{a^2}\Bigr)\,dx
      = \pi\,b\,c \Bigl[\,x - \frac{x^3}{3\,a^2}\Bigr]_{-a}^{a} \\[6pt]
    &= \pi\,b\,c \Bigl(\bigl(a - \tfrac{a^3}{3a^2}\bigr) - \bigl(-a + \tfrac{(-a)^3}{3a^2}\bigr)\Bigr)
      = \pi\,b\,c \Bigl(2a - \tfrac{2a}{3}\Bigr)
      = \frac{4}{3}\,\pi\,a\,b\,c
  \end{aligned}
  \]
  \textbf{Ответ: } $\frac{4}{3}\,\pi\,a\,b\,c$\\

  \item[\textbf{№7}]Площадь поверхности вращения кривой
  \[
  S \;=\; 2\pi \int_{0}^{3} y\,\sqrt{1 + \bigl(y'\bigr)^2}\,dx
  \]
  Подставляем заданную функцию y и ее производную
  \[
  y = \sqrt{2\lambda x}, 
  \quad
  y' = \frac{d}{dx}\sqrt{2\lambda x}
        = \frac{\lambda}{\sqrt{2\lambda x}}
  \]
  Преобразуем 
  \[
  \sqrt{1 + (y')^2}
  = \sqrt{1 + \frac{\lambda^2}{2\lambda x}}
  = \sqrt{\frac{2x + \lambda}{2x}}
  \]
  \[
  y\,\sqrt{1+(y')^2}
  = \sqrt{2\lambda x}\;\sqrt{\frac{2x + \lambda}{2x}}
  = \sqrt{\lambda\,(2x + \lambda)}
  \]
  Выносим множитель
  \[
  S = 2\pi \int_{0}^{3} \sqrt{\lambda\,(2x + \lambda)}\,dx
      = 2\pi\sqrt{\lambda}\,\int_{0}^{3} \sqrt{2x + \lambda}\,dx
  \]
  Выполняем замену
  \[
  \int_{0}^{3} \sqrt{2x + \lambda}\,dx
  = \frac{1}{2}\int_{\lambda}^{\lambda+6} u^{1/2}\,du
  = \frac{1}{3}\Bigl[(\lambda+6)^{3/2} - \lambda^{3/2}\Bigr]
  \]
  \textbf{Ответ: } $S = \frac{2\pi\sqrt{\lambda}}{3}\,\bigl[(\lambda + 6)^{3/2} - \lambda^{3/2}\bigr]$\\

  \item[\textbf{№8}]Используем Формулу 7 для площади поверхности вращения в пространстве 
  \[
  S \;=\; 2\pi \int_{-\,\lambda}^{\lambda} y(x)\,\sqrt{1 + \bigl(y'(x)\bigr)^2}\,dx
  \]
  Здесь сечение сферой плоскостью $y-z$ при фиксированном 
  x есть круг радиуса $y(x)=\sqrt{\lambda^2 - x^2}$, его производная:
  \[
  y(x) = \sqrt{\lambda^2 - x^2}, 
  \quad
  y'(x) = \frac{d}{dx}\sqrt{\lambda^2 - x^2} = -\,\frac{x}{\sqrt{\lambda^2 - x^2}}
  \]
  Поэтому  
  \[
  \sqrt{1 + \bigl(y'(x)\bigr)^2}
  = \sqrt{1 + \frac{x^2}{\lambda^2 - x^2}}
  = \sqrt{\frac{\lambda^2}{\lambda^2 - x^2}}
  = \frac{\lambda}{\sqrt{\lambda^2 - x^2}}
  \]
  Подставляем в интеграл 
  \[
  y\,\sqrt{1 + (y')^2}
  = \sqrt{\lambda^2 - x^2}\;\frac{\lambda}{\sqrt{\lambda^2 - x^2}}
  = \lambda
  \]
  \[
  S = 2\pi \int_{-\,\lambda}^{\lambda} \lambda \,dx
      = 2\pi\lambda \bigl[x\bigr]_{-\,\lambda}^{\lambda}
      = 2\pi\lambda\,(2\lambda)
      = 4\pi\,\lambda^2
  \]
  \textbf{Ответ: } $S = 4\pi\,\lambda^2$\\

  \item[\textbf{№9}]Формула 3 для длины дуги кривой $y = f(x)$ и формула 8 для координат центра масс кривой при плотности $\rho=const $
  \[
  \begin{aligned}
  &y = \sqrt{\lambda^2 - x^2}, 
  \quad y' = -\frac{x}{\sqrt{\lambda^2 - x^2}}, 
  \\
  &ds = \sqrt{1 + (y')^2}\,dx
      = \sqrt{1 + \frac{x^2}{\lambda^2 - x^2}}\,dx
      = \frac{\lambda}{\sqrt{\lambda^2 - x^2}}\,dx,
  \\
  &L = \int_{-\lambda}^{\;\lambda} ds
      = \lambda \int_{-\lambda}^{\;\lambda} \frac{dx}{\sqrt{\lambda^2 - x^2}}
      = \lambda \,\pi,
  \\
  &x_c = \frac{\displaystyle\int_{-\lambda}^{\;\lambda} x\,ds}{\displaystyle\int_{-\lambda}^{\;\lambda} ds}
        = 0 \quad\text{(по симметрии)}, 
  \\
  &y_c = \frac{\displaystyle\int_{-\lambda}^{\;\lambda} y\,ds}{\displaystyle\int_{-\lambda}^{\;\lambda} ds}
        = \frac{\displaystyle\int_{-\lambda}^{\;\lambda} \sqrt{\lambda^2 - x^2}\,\frac{\lambda\,dx}{\sqrt{\lambda^2 - x^2}}}
              {\lambda\,\pi}
        = \frac{\displaystyle\lambda\int_{-\lambda}^{\;\lambda} dx}{\lambda\,\pi}
        = \frac{2\lambda^2}{\lambda\,\pi}
        = \frac{2\lambda}{\pi}
  \end{aligned}
  \]
  \textbf{Ответ: } $(x_c, y_c) = \Bigl(0,\;\frac{2\lambda}{\pi}\Bigr)$\\

  \item[\textbf{№10}]По Формуле 9 для центра масс плоской фигуры при однородной плотности:
  $$x_c = \frac{\iint_D x\,dA}{\iint_D dA},   y_c = \frac{\iint_D y\,dA}{\iint_D dA}$$
  \[
  \text{Область }D: 0 \le x \le \lambda,\quad -\sqrt{\lambda x}\,\le y \le \sqrt{\lambda x}.
  \]
  Площадь D:
  \[
  A = \int_{x=0}^{\lambda}\bigl(2\sqrt{\lambda x}\bigr)\,dx
      = 2\sqrt{\lambda}\int_{0}^{\lambda} x^{1/2}\,dx
      = 2\sqrt{\lambda}\,\frac{2}{3}\,\lambda^{3/2}
      = \frac{4}{3}\,\lambda^2.
  \]
  Координата $x_c$:
  \[
  \iint_D x\,dA
  = \int_{0}^{\lambda} x\bigl(2\sqrt{\lambda x}\bigr)\,dx
  = 2\sqrt{\lambda}\int_{0}^{\lambda} x^{3/2}\,dx
  = 2\sqrt{\lambda}\,\frac{2}{5}\,\lambda^{5/2}
  = \frac{4}{5}\,\lambda^3,
  \]
  \[
  x_c
  = \frac{\displaystyle\iint_D x\,dA}{\displaystyle A}
  = \frac{\tfrac{4}{5}\,\lambda^3}{\tfrac{4}{3}\,\lambda^2}
  = \frac{3}{5}\,\lambda.
  \]
  По симметрии
  \[
  y_c = 0.
  \]
  \textbf{Ответ: } $(x_c, y_c) = \Bigl(\tfrac{3}{5}\,\lambda,\;0\Bigr)$\\

  \item[\textbf{№11}]
  \begin{enumerate}
    \item[(a)]Определяем $I_n(x)$ по Формуле 10:
    \[
    I_n(x) \;=\;\int_{-1}^{1}(1 - t^2)^n\cos(xt)\,dt 
    \]
    \item[(b)]По методу полной мат. индукции имеем
    \[
    x^{2n+1}I_n(x) \;=\; n!\,\bigl(P_n(x)\sin x + Q_n(x)\cos x\bigr)
    \]
    \item[(c)]Предположим, что \(\displaystyle\frac\pi2=\frac ab\), \(a,b\in\mathbb N\). 
    Подставляя $x=\pi/2$ в предыдущее равенство, получаем Формулу 11:
    \[
    \frac{a^{2n+1}}{n!}\,I_n\Bigl(\frac\pi2\Bigr)
    \;=\;b^{2n+1}P_n\Bigl(\frac\pi2\Bigr)
    \]
    \item[(d)]Покажем, что $0< I_n(\pi/2)<2$. Заметим, что на $[-1,1]$
    $$\text{$0\leq (1-t^2)^n\leq 1$ и $|\cos(\f{t\pi}{2})|\leq 1$}$$

    \item[(e)]По представлению Пуянкара правая часть $b^{2n+1}P_n(\f{\pi}{2})$ --- целое число.

    \item[(f)]Переход к пределу $n \to \infty$ в равенстве 11 даёт
    \[
    \lim_{n\to\infty}\frac{a^{2n+1}}{n!}\,I_n\Bigl(\frac\pi2\Bigr)=0,
    \]
    но это целое (по пункту (e)) и строго положительное (по (d)) число — противоречие.
  \end{enumerate}
  \textbf{Вывод:} допущение рациональности $\f{\pi}{2}$ ведёт к противоречию, значит $\pi$ иррационально.
\end{enumerate}
\end{document}